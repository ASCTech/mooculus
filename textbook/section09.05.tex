\section{Work}{}{}
\nobreak
A fundamental concept in classical physics is {\dfont
work\index{work}\/}: If an object is moved in a straight line against
a force $F$ for a distance $s$ the work done is $W=Fs$.

\begin{example}
How much work is done in lifting a 10 pound weight vertically a
distance of 5 feet? The force due to gravity on a 10 pound weight is
10 pounds at the surface of the earth, and it does not change
appreciably over 5 feet. The work done is $W=10\cdot 5=50$ foot-pounds.
\end{example}

In reality few situations are so simple. The force might not be
constant over the range of motion, as in the next example.

\begin{example} How much work is done in lifting a 10 pound weight from the
surface of the earth to an orbit 100 miles above the surface? Over 100
miles the force due to gravity does change significantly, so we need
to take this into account. The force exerted on a 10 pound weight at a
distance $r$ from the center of the earth is $\ds F=k/r^2$ and by
definition it is 10 when $r$ is the radius of the earth (we assume the
earth is a sphere). How can we approximate the work done? We divide
the path from the surface to orbit into $n$ small subpaths. On each
subpath the force due to gravity is roughly constant, with value
$\ds k/r_i^2$ at distance $\ds r_i$. The work to raise the object from
$\ds r_i$ to $\ds r_{i+1}$ is thus approximately $\ds k/r_i^2\Delta r$ and the
total work is approximately
$$\sum_{i=0}^{n-1} {k\over r_i^2}\Delta r,$$
or in the limit
$$W=\int_{r_0}^{r_1} {k\over r^2}\,dr,$$
where $\ds r_0$ is the radius of the earth and $\ds r_1$ is $\ds r_0$ plus 100
miles. The work is
$$W=\int_{r_0}^{r_1} {k\over r^2}\,dr=
-\left.{k\over r}\right|_{r_0}^{r_1}=-{k\over r_1}+{k\over r_0}.$$
Using $\ds r_0=20925525$ feet we have $\ds r_1=21453525$. The force on the 10
pound weight at the surface of the earth is 10 pounds, so 
$\ds 10=k/20925525^2$, giving $k=4378775965256250$. Then
$$-{k\over r_1}+{k\over r_0}={491052320000\over 95349}\approx 5150052
\quad\hbox{foot-pounds}.$$
Note that if we assume the force due to gravity is 10 pounds over the
whole distance we would calculate the work as $\ds 10(r_1-r_0)=10\cdot100\cdot
5280=5280000$, somewhat higher since we don't account for the
weakening of the gravitational force.
\end{example}

\begin{example} How much work is done in lifting a 10 kilogram object from the
surface of the earth to a distance $D$ from the center of the earth?
This is the same problem as before in different units, and we are not
specifying a value for $D$. As before
$$W=\int_{r_0}^{D} {k\over r^2}\,dr= -\left.{k\over
  r}\right|_{r_0}^{D}=-{k\over D}+{k\over r_0}.$$ 
While ``weight in pounds'' is a measure of force, ``weight in
kilograms'' is a measure of mass. To convert to force we need to use
Newton's law $F=ma$. At the surface of the earth the acceleration due
to gravity is approximately 9.8 meters per second squared, so the
force is $F=10\cdot 9.8=98$. The units here are ``kilogram-meters per
second squared'' or ``kg m/s$^2$'', also known as a
Newton\index{Newton} (N), so $F=98$~N.  The radius of the earth is
approximately 6378.1 kilometers or 6378100 meters.
Now the problem proceeds as before. From
$\ds F=k/r^2$ we compute $k$:
$\ds 98=k/6378100^2$, $\ds k= 3.986655642\cdot 10^{15}$. Then the work is:
$$W=-{k\over D}+6.250538000\cdot 10^8\quad\hbox{Newton-meters.}$$
As $D$ increases $W$ of course gets larger, since the quantity being
subtracted, $-k/D$, gets smaller. But note that the work $W$ will
never exceed $\ds 6.250538000\cdot 10^8$, and in fact will approach this
value as $D$ gets larger. In short, with a finite amount of work, namely
$\ds 6.250538000\cdot 10^8$ N-m, we can lift the 10 kilogram object as far
as we wish from earth.
\end{example}
\label{example:object to infinity}

Next is an example in which the force is constant, but there are many
objects moving different distances.

\begin{example} Suppose that a water tank is shaped like a right circular
cone with the tip at the bottom, and has height 10 meters and radius 2
meters at the top. If the tank is full, how much work is required
to pump all the water out over the top? Here we have a large number of
atoms of water that must be lifted different distances to get to the
top of the tank. Fortunately, we don't really have to deal with
individual atoms---we can consider all the atoms at a given depth
together. 

To approximate the work, we can divide the water in the tank into
horizontal sections, approximate the volume of water in a section by a
thin disk, and compute the amount of work required to lift each disk
to the top of the tank. As usual, we take the limit as the sections
get thinner and thinner to get the total work.

\figure
\vbox{\beginpicture
\normalgraphs
\ninepoint
\setcoordinatesystem units <0.5truecm,0.5truecm>
\setplotarea x from -2 to 2, y from 0 to 10
\axis left shiftedto x=0 /
\plot -2 10 0 0 2 10 -2 10 /
\setdashes
\putrule from -1.2 6 to 1.2 6
\betweenarrows {$h$} from -0.4 6 to -0.4 10
\betweenarrows {10} from 2.3 0 to 2.3 10
\setplotsymbol ({\teeny.})
\plotsymbolspacing=.2pt
\put {2} [b] <0pt,2pt> at 1 10
\arrow <2pt> [0.7, 2] from 0.5 10.4 to 0 10.4
\arrow <2pt> [0.7, 2] from 1.5 10.4 to 2 10.4
\endpicture}
\figrdef{fig:conical water tank}
\endfigure{Cross-section of a conical water tank.}

At depth $h$ the circular cross-section through the tank has radius
$r=(10-h)/5$, by similar triangles,
 and area $\ds \pi(10-h)^2/25$. A section of the tank at depth
$h$ thus has volume approximately $\ds \pi(10-h)^2/25\Delta h$ and so
contains $\ds \sigma\pi(10-h)^2/25\Delta h$ kilograms of water, where
$\sigma$ is the density of water in kilograms per cubic meter;
$\sigma\approx 1000$. The force due to gravity on this much water is
$\ds 9.8\sigma\pi(10-h)^2/25\Delta h$, and finally, this section of water
must be lifted a distance $h$, which requires
$\ds h9.8\sigma\pi(10-h)^2/25\Delta h$ Newton-meters of work. The total
work is therefore
$$W={9.8\sigma\pi\over 25} \int_0^{10} h(10-h)^2\,dh={980000\over3}\pi\approx
1026254\quad\hbox{Newton-meters.}$$
\end{example}

A spring has a ``natural length,'' its length if nothing is stretching
or compressing it. If the spring is either stretched or compressed the
spring provides an opposing force; according to {\dfont Hooke's
Law\index{Hooke's Law}\/} the magnitude of this force is proportional to the
distance the spring has been stretched or compressed: $F=kx$.
The constant of proportionality, $k$, of course depends on the
spring. Note that $x$ here represents the {\em change\/} in length from the
natural length.

\begin{example} Suppose $k=5$ for a given spring that has a natural length of
$0.1$ meters. Suppose a force is applied that compresses the spring to
length $0.08$. What is the magnitude of the force? Assuming that the
constant $k$ has appropriate dimensions (namely, kg/s$^2$), the force is
$5(0.1-0.08)=5(0.02)=0.1$ Newtons.
\end{example}

\begin{example} How much work is done in compressing the spring in the
previous example from its natural length to $0.08$ meters? From $0.08$
meters to $0.05$ meters? How much work is done to stretch the spring
from $0.1$ meters to $0.15$ meters?  We can approximate the work by
dividing the distance that the spring is compressed (or stretched)
into small subintervals. Then the force exerted by the spring is
approximately constant over the subinterval, so the work required to
compress the spring from $\ds x_i$ to $\ds x_{i+1}$ is approximately
$\ds 5(x_i-0.1)\Delta x$.  The total work is approximately
$$\sum_{i=0}^{n-1} 5(x_i-0.1)\Delta x$$
and in the limit
$$W=\int_{0.1}^{0.08} 5(x-0.1)\,dx=\left.{5(x-0.1)^2\over2}\right|_{0.1}^{0.08}=
{5(0.08-0.1)^2\over2}-{5(0.1-0.1)^2\over2}={1\over1000}\,\hbox{N-m}.$$
The other values we seek simply use different limits. To compress the
spring from $0.08$
meters to $0.05$ meters takes
$$W=\int_{0.08}^{0.05} 5(x-0.1)\,dx=\left.{5x^2\over2}\right|_{0.08}^{0.05}=
{5(0.05-0.1)^2\over2}-{5(0.08-0.1)^2\over2}={21\over4000}\quad\hbox{N-m}$$
and to stretch the spring
from $0.1$ meters to $0.15$ meters requires
$$W=\int_{0.1}^{0.15} 5(x-0.1)\,dx=\left.{5x^2\over2}\right|_{0.1}^{0.15}=
{5(0.15-0.1)^2\over2}-{5(0.1-0.1)^2\over2}={1\over160}\quad\hbox{N-m}.$$
\end{example}

\begin{exercises}

\begin{exercise} How much work is done in lifting a 100 kilogram weight from
the surface of the earth to an orbit 35,786 kilometers above the
surface of the earth?
\begin{answer} $\approx 5,305,028,516$ N-m
\end{answer}\end{exercise}

\begin{exercise} How much work is done in lifting a 100 kilogram weight from
an orbit 1000 kilometers above the surface of the earth to an orbit
35,786 kilometers above the surface of the earth?
\begin{answer} $\approx 4,457,854,041$ N-m
\end{answer}\end{exercise}

\begin{exercise} A water tank has the shape of an upright cylinder with radius $r=1$
meter and height 10 meters. If the depth of the water is 5 meters, how
much work is required to pump all the water out the top of the tank?
\begin{answer} $367,500 \pi$ N-m
\end{answer}\end{exercise}

\begin{exercise} Suppose the tank of the previous problem is lying on its
side, so that the circular ends are vertical, and that it has the same
amount of water as before. How much work is required to pump the water
out the top of the tank (which is now 2 meters above the bottom of the
tank)?
\begin{answer} $49000\pi + 196000/3$ N-m
\end{answer}\end{exercise}

\begin{exercise} A water tank has the shape of the bottom half of a sphere
with radius $r=1$ meter. If the tank is full,
how much work is required to pump all the water out
the top of the tank?
\begin{answer} $2450\pi$ N-m
\end{answer}\end{exercise}

\begin{exercise} A spring has constant $k=10$ kg/s$^2$. How much work is done
in compressing it $1/10$ meter from its natural length?
\begin{answer} $0.05$ N-m
\end{answer}\end{exercise}

\begin{exercise} A force of 2 Newtons will compress a spring from 1 meter
(its natural length) to
0.8 meters. How much work is required to stretch the spring from 
1.1 meters to 1.5 meters?
\begin{answer} $6/5$ N-m
\end{answer}\end{exercise}

\begin{exercise} A 20 meter long steel cable has density 2 kilograms per
meter, and is hanging straight down. How much work is required to lift
the entire cable to the height of its top end?
\begin{answer} $3920$ N-m
\end{answer}\end{exercise}

\begin{exercise} The cable in the previous problem has a 100 kilogram bucket
of concrete attached to its lower end. How much work is required to lift
the entire cable and bucket to the height of its top end?
\begin{answer} $23520$ N-m
\end{answer}\end{exercise}

\begin{exercise} Consider again the cable and bucket of the previous problem.
How much work is required to lift the bucket 10 meters by raising the
cable 10 meters? (The top half of the cable ends up at the height of
the top end of the cable, while the bottom half of the cable is lifted
10 meters.)
\begin{answer} $12740$ N-m
\end{answer}\end{exercise}

\end{exercises}

