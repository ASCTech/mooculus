\section{Newton's Method}{}{}
\nobreak
Suppose you have a function $f(x)$, and you want to find as accurately as
possible where it crosses the $x$-axis; in other words, you want to solve
$f(x)=0$.  Suppose you know of no way to find an exact solution by any
algebraic procedure, but you are able to use an approximation,
provided it can be made quite close to the true value.
Newton's method is a way to find a solution to the
equation to as many decimal places as you want.  It is what is called an
``iterative procedure,'' meaning that it can be repeated again and again to
get an answer of greater and greater accuracy.  Iterative procedures like
Newton's method are well suited to programming for a
computer. Newton's method uses the fact that the tangent line to a
curve is a good approximation to the curve near the point of tangency.

\begin{example} Approximate $\ds \sqrt{3}$. Since $\ds \sqrt{3}$ is a solution to
$\ds x^2=3$ or $\ds x^2-3=0$, we use $\ds f(x)=x^2-3$. We start by guessing
something reasonably close to the true value; this is usually easy to
do; let's use $\ds \sqrt3\approx2$. Now use the tangent line to the curve
when $x=2$ as an approximation to the curve, as shown in
figure~\xrefn{fig:newtons method}. Since $f'(x)=2x$, the slope of this
tangent line is 4 and its equation is $y=4x-7$. The tangent line is
quite close to $f(x)$, so it crosses the $x$-axis near the point at
which $f(x)$ crosses, that is, near $\ds \sqrt3$. It is easy to find where
the tangent line crosses the $x$-axis: solve $0=4x-7$ to get
$x=7/4=1.75$. This is certainly a better approximation than 2, but let
us say not close enough. We can improve it by doing the same thing
again: find the tangent line at $x=1.75$, find where this new tangent
line crosses the $x$-axis, and use that value as a better
approximation. We can continue this indefinitely, though it gets a bit
tedious. Lets see if we can shortcut the process. Suppose the best
approximation to the intercept we have so far is $\ds x_i$. To find a
better approximation we will always
do the same thing: find the slope of the tangent line at $\ds x_i$, find the
equation of the tangent line, find the $x$-intercept. The slope is
$\ds 2x_i$. The tangent line is $\ds y=(2x_i)(x-x_i)+(x_i^2-3)$, using
the point-slope formula for a line. Finally, the intercept is found by
solving $\ds 0 =(2x_i)(x-x_i)+(x_i^2-3)$. With a little algebra this turns
into $\ds x=(x_i^2+3)/(2x_i)$; this is the next approximation, which we
naturally call $\ds x_{i+1}$. Instead of doing the whole tangent line
computation every time we can simply use this formula to get as many
approximations as we want. Starting with $\ds x_0=2$, we get
$\ds x_1=(x_0^2+3)/(2x_0)=(2^2+3)/4=7/4$ (the same approximation we got
above, of course), $\ds x_2=(x_1^2+3)/(2x_1)=
((7/4)^2+3)/(7/2)=97/56\approx 1.73214$, $\ds x_3\approx 1.73205$, and so
on. This is still a bit tedious by hand, but with a calculator or,
even better, a good computer program, it is quite easy to get many,
many approximations. We might guess already that $1.73205$ is accurate
to two decimal places, and in fact it turns out that it is accurate to
5 places.  \end{example}

%% BADBAD
%% \figure
%% \vbox{\beginpicture
%% \normalgraphs
%% \sevenpoint
%% \setcoordinatesystem units <1.5truecm,0.6truecm>
%% \setplotarea x from 0 to 4, y from -4 to 7
%% \axis left /
%% \axis bottom shiftedto y=0 ticks numbered from 1 to 2 by 1 /
%% \plot 0.000 -3.000 0.150 -2.978 0.300 -2.910 0.450 -2.798 0.600 -2.640 
%% 0.750 -2.438 0.900 -2.190 1.050 -1.898 1.200 -1.560 1.350 -1.178 
%% 1.500 -0.750 1.650 -0.278 1.800 0.240 1.950 0.802 2.100 1.410 
%% 2.250 2.062 2.400 2.760 2.550 3.502 2.700 4.290 2.850 5.122 
%% 3.000 6.000 /
%% \setlinear
%% \plot 1 -3 3.5 7 /
%% \endpicture}
%% \figrdef{fig:newtons method}
%% \endfigure{Newton's method.
%% (\expandafter\url\expandafter{\liveurl jsxgraph/newtons_method.html}%
%% AP\endurl)}

Let's think about this process in more general terms. We want to
approximate a solution to $f(x)=0$.
We start with a rough guess, which we call $\ds x_0$.  We use the tangent
line to $f(x)$ to get a new approximation that we hope will be closer
to the true value. What is the equation of the tangent line when
$\ds x=x_0$? The slope is $\ds f'(x_0)$ and the line goes through
$\ds(x_0,f(x_0))$, so the equation of the line is
$$ y=f'(x_0)(x-x_0)+f(x_0).$$
Now we find where this crosses the $x$-axis by substituting $y=0$ and
solving for $x$:
$$x={x_0f'(x_0)-f(x_0)\over f'(x_0)} = x_0 - {f(x_0)\over f'(x_0)}.$$
We will typically want to compute more than one of these improved
approximations, so we number them consecutively; from $\ds x_0$ we have
computed $\ds x_1$:
$$x_1={x_0f'(x_0)-f(x_0)\over f'(x_0)} = x_0 - {f(x_0)\over f'(x_0)},$$
and in general from $\ds x_i$ we compute $\ds x_{i+1}$:
$$x_{i+1}={x_if'(x_i)-f(x_i)\over f'(x_i)} = x_i - {f(x_i)\over f'(x_i)}.$$

\begin{example}
Returning to the previous example, $\ds f(x)=x^2-3$, $f'(x)=2x$, and the
formula becomes $\ds x_{i+1}=x_i - (x_i^2-3)/(2x_i)=(x_i^2+3)/(2x_i)$, as
before.
\end{example}

In practice, which is to say, if you need to approximate a value in
the course of designing a bridge or a building or an airframe, you
will need to have some confidence that the approximation you settle on
is accurate enough. As a rule of thumb, once a certain number of
decimal places stop changing from one approximation to the next it is
likely that those decimal places are correct. Still, this may not be
enough assurance, in which case we can test the result for accuracy.

\begin{example} Find the $x$ coordinate of the intersection of the curves $y=2x$
and $y=\tan x$, accurate to three decimal places. To put this in the
context of Newton's method, we note that we want to know where $2x=\tan
x$ or $f(x)=\tan x-2x=0$. We compute $\ds f'(x)=\sec^2 x - 2$ and set up the
formula:
$$x_{i+1} = x_i-{\tan x_i -2x_i\over \sec^2 x_i - 2}.$$ From the graph
in figure~\xrefn{fig:tanx vs x} 
we guess $\ds x_0=1$ as a starting point, then using the formula we
compute $\ds x_1=1.310478030$, $\ds x_2=1.223929096$, $\ds x_3=1.176050900$,
$\ds x_4=1.165926508$, $\ds x_5=1.165561636$.  So we guess that the first
three places are correct, but that is not the same as saying $1.165$
is correct to three decimal places---$1.166$ might be the correct,
rounded approximation. How can we tell? We can substitute $1.165$,
$1.1655$ and $1.166$ into $\tan x - 2x$; this
gives $-0.002483652$, $-0.000271247$, $0.001948654$. Since the first
two are negative and the third is positive, $\tan x - 2x$ crosses the
$x$ axis between $1.1655$ and $1.166$, so the correct value to three
places is $1.166$.
\end{example}

%% BADBAD
%% \figure
%% \vbox{\beginpicture
%% \normalgraphs
%% \sevenpoint
%% \setcoordinatesystem units <3truecm,0.25truecm>
%% \setplotarea x from 0 to 1.5, y from 0 to 15
%% \axis left ticks numbered from 0 to 15 by 5 /
%% \axis bottom shiftedto y=0 ticks length <2pt> withvalues {$1$} {$1.5$} / at
%% 1 1.5 / /
%% \plot 0.000 0.000 0.038 0.038 0.075 0.075 0.112 0.113 0.150 0.151 
%% 0.188 0.190 0.225 0.229 0.262 0.269 0.300 0.309 0.338 0.351 
%% 0.375 0.394 0.412 0.438 0.450 0.483 0.488 0.530 0.525 0.579 
%% 0.562 0.630 0.600 0.684 0.638 0.741 0.675 0.800 0.712 0.864 
%% 0.750 0.932 0.788 1.004 0.825 1.083 0.862 1.167 0.900 1.260 
%% 0.938 1.362 0.975 1.475 1.012 1.601 1.050 1.743 1.088 1.905 
%% 1.125 2.093 1.162 2.312 1.200 2.572 1.238 2.888 1.275 3.282 
%% 1.312 3.785 1.350 4.455 1.388 5.394 1.425 6.810 1.462 9.198 
%% 1.500 14.101  /
%% \setlinear
%% \plot 0 0 1.5 3 /
%% \setcoordinatesystem units <3truecm,0.25truecm> point at -2.5 0
%% \setplotarea x from 0 to 1.5, y from 0 to 15
%% \axis left ticks numbered from 0 to 15 by 5 /
%% \axis bottom shiftedto y=0 ticks length <2pt> withvalues {$1$} {$1.5$} / at
%% 1 1.5 / /
%% \plot 0.000 -0.000 0.037 -0.037 0.075 -0.075 0.113 -0.112 0.150 -0.149
%% 0.188 -0.185 0.225 -0.221 0.263 -0.256 0.300 -0.291 0.338 -0.324
%% 0.375 -0.356 0.412 -0.387 0.450 -0.417 0.487 -0.445 0.525 -0.471
%% 0.562 -0.495 0.600 -0.516 0.637 -0.534 0.675 -0.550 0.713 -0.561
%% 0.750 -0.568 0.787 -0.571 0.825 -0.567 0.863 -0.558 0.900 -0.540
%% 0.938 -0.513 0.975 -0.475 1.012 -0.424 1.050 -0.357 1.087 -0.270
%% 1.125 -0.157 1.163 -0.013 1.200 0.172 1.238 0.413 1.275 0.732
%% 1.312 1.160 1.350 1.755 1.387 2.619 1.425 3.960 1.462 6.273
%% 1.500 11.101 /
%% \endpicture}
%% \figrdef{fig:tanx vs x}
%% \endfigure{$y=\tan x$ and $y=2x$ on the left, $y=\tan x-2x$ on the right.}

\begin{exercises}

\begin{exercise} Approximate the fifth root of 7, using $\ds x_0=1.5$ as a
first guess. Use Newton's method to find $\ds x_3$ as your
approximation.  
\begin{answer} $\ds x_3=1.475773162$ 
\end{answer}\end{exercise}

\begin{exercise} Use Newton's Method to approximate the cube root of 10 to
two decimal places.
\begin{answer} $2.15$
\end{answer}\end{exercise}

\begin{exercise} The function 
$\ds f(x)=x^3-3x^2-3x+6$ has a root between 3 and 4, because
$f(3)=-3$ and $f(4)=10$. Approximate the root to two decimal places.
\begin{answer}  $3.36$
\end{answer}\end{exercise}

\begin{exercise} A rectangular piece of cardboard of dimensions $8\times 17$
is used to make an open-top box by cutting out a small square of side
$x$ from each corner and bending up the sides.  (See
exercise~\xrefn{sec:optimization}.\xrefn{exercise: cardboard box}.)
If $x=2$, then the volume of the box is $2\cdot 4\cdot 13=104$.  Use
Newton's method to find a value of $x$ for which the box has volume
100, accurate to 3 significant figures.  
\begin{answer} $2.19$ or $1.26$
\end{answer}\end{exercise}

% Mike Wills stuff
% \iflatetranscendentals
% 
% \begin{remark} {Further investigation} Newton's method does not always
% work. The following result gives sufficient conditions for when it
% does.  \end{remark}
% 
% \begin{theorem} (Newton) Let $f:[a,b] \rightarrow \R$ be continuous. Suppose that
% $f(c) =0$ for some $c$ in $(a,b)$.  Suppose that $f''$ exists and is
% bounded on $(a,b)$; that is, there exists $M>0$ such that
% $$f''(x) \leq M $$ 
% for every $x\in (a,b)$.
% 
% Suppose that $f'$ is bounded away from zero; that is, there exists
% $\epsilon >0$ such that
% $$f'(x) \geq \epsilon $$ for every $x \in (a,b)$. Then there is a
% closed interval $I:=[d,e] \subset (a,b)$ containing $c$ such that
% the sequence given recursively by
% $$x_n = x_{n-1} -f(x_{n-1})/f'(x_{n -1}) $$ converges to
% $c$. Moreover, $x_n$ is in $I$ for each $n$.
% \end{proof}
% 
% \noindent
% For this to make sense it helps if we have a notion of convergence of
% sequences. We will do this in second semester calculus. For now, we
% will just use the intuitive statement ``as $n$ gets large, $x_n$ gets
% arbitrarily close to $c$" and write 
% $\displaystyle{\lim _{n\rightarrow\infty } x_n }= c$.
% 
% \noindent
% Let us now analyze the various conditions in the statement.
% 
% \noindent
% Bounding $f'$ away from zero should seem reasonable from the
% definition of $\ds x_n$ but it is not necessary. The following
% exercise shows that Newton's method may work even if $f'$ is not
% bounded away from zero.
% 
% \begin{exercise} Let $\ds f(x) = x^2$. Adopt the notation of Newton's
% theorem. Show that if $\ds x_{n-1 } \neq 0$ then $\ds x_n
% =x_{n-1}/2$. Suppose that $\ds x_0 = 1.$ What is $\ds x_1$? How about
% $\ds x_2$? How about $\ds x_n$? Argue that $\ds x_n$ converges to zero
% as $n$ goes to $\infty$ when $\ds x_0 =1$.
% 
% \noindent
% In the above example, any $\ds x_0$ except zero will do.  The next
% exercise illustrates what can go wrong if we do not bound $f'$ away
% from zero.
% 
% \begin{exercise} Let $\ds f(x) = x^4 - 4x - 11$. Then $f(0)<0 < f(3)$ so by
% the intermediate value theorem there exists a root between $x=0$ and
% $x=3$. What happens if we apply Newton's method when we start at $\ds
% x_0 =1$?  Sketch the graph of $f$ on the interval $[0,3]$.
%  
% \noindent
% We also require a bound on the second derivative. The following
% exercise shows what can go wrong.
% 
% \begin{exercise} Let $\ds f(x) = x^{1/3}$. Observe that $x=0$ is a root of
% $f$.  Compute $f''(x)$ and explain why $f''$ is unbounded near
% $x=0$. Attempt to use Newton's method with $\ds x_0 = 1$.  What
% happens? Illustrate your conclusion with a diagram.
%  
% \noindent
% Newton's method, when it works, is extremely fast. The number of
% decimal places of accuracy approximately doubles with each successive
% iteration.
%  
% \noindent
% The statement of Newton's theorem does not tell us how to how to find
% the closed interval $I$. The proof gives some indication but is beyond
% the scope of the course.
%  
% \noindent
% For our purposes, we can use a graphing calculator to give us an idea
% of where to look for the roots of a given function and then apply
% Newton's method to get a good decimal approximation.
%  
% \begin{exercise} Use Newton's method to find the coordinates of the
% inflection point of $\ds y=x^3/12 + \sin x$ correct to six decimal
% places.
%  
% \begin{exercise} Let $a$ be a non-zero number. Suppose we want to compute a
% decimal expansion of $1/a$.  This could be difficult if $a$ is big or
% has a large number of decimal places. Newton's method gives us a way
% of computing $a$ without division.  Let $f(x) = 1/x - a$. Notice that
% $f$ has a root at $x=1/a$.  Using the notation of Newton's theorem,
% show that $\ds x_{n} =2x_{n-1} - ax_{n-1}^2$.
%  
% \fi

\end{exercises}
