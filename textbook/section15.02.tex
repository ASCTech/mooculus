\section{Double Integrals in Cylindrical Coordinates}{}{}
\label{sec:Double Integrals in Cylindrical Coordinates}
\nobreak
Suppose we have a surface given in cylindrical coordinates as
$z=f(r,\theta)$ and we wish to find the integral over some region. We
could attempt to translate into rectangular coordinates and do the
integration there, but it is often easier to stay in cylindrical
coordinates.

How might we approximate the volume under such a surface in a way that
uses cylindrical coordinates directly? The basic idea is the same as
before: we divide the region into many small regions, multiply the
area of each small region by the height of the surface somewhere in
that little region, and add them up. What changes is the shape of the
small regions; in order to have a nice representation in terms of $r$
and $\theta$, we use small pieces of ring-shaped areas, as shown in
figure~\xrefn{fig:cylindrical coordinates regions}. Each small region
is roughly rectangular, except that two sides are segments of a circle
and the other two sides are not quite parallel. Near a point
$(r,\theta)$, the length of either circular arc is about
$r\Delta\theta$ and the length of each straight side is simply $\Delta
r$. When $\Delta r$ and $\Delta \theta$ are very small, the region is
nearly a rectangle with area $r\Delta r\Delta\theta$, and the volume
under the surface is approximately
$$\sum\sum f(r_i,\theta_j)r_i\Delta r\Delta\theta.$$
In the limit, this turns into a double integral
$$\int_{\theta_0}^{\theta_1}\int_{r_0}^{r_1} f(r,\theta)r\,dr\,d\theta.$$

\figure
\vbox{\beginpicture
\normalgraphs
\ninepoint
\setcoordinatesystem units <1.5truecm,1.5truecm>
\setplotarea x from 0 to 4.1, y from 0 to 4.1
\axis left  /
\axis bottom  /
\circulararc 70 degrees from 3.5 0.5 center at 0 0
\circulararc 70 degrees from 2.5 0.5 center at 0 0
\circulararc 60 degrees from 1.5 0.5 center at 0 0
\setlinear
\plot 3.3 1.166 0 0 2.8 2.1 /
\plot 2 2.87 0 0 /
\put {$\Delta r$} [tl] <0pt,-3pt> at 2.9 1
\put {$r\Delta \theta$} [bl] <2pt,2pt> at 3.1 1.62
\endpicture}
\figrdef{fig:cylindrical coordinates regions}
\endfigure{A cylindrical coordinates ``grid''.}

\begin{example} Find the volume under $z=\sqrt{4-r^2}$ 
above the quarter circle bounded by
the two axes and the circle $x^2+y^2=4$ in the first quadrant.

In terms of $r$ and $\theta$, this region is described by the
restrictions $0\le r\le 2$ and $0\le\theta\le\pi/2$, so we have
$$\eqalign{
\int_{0}^{\pi/2}\int_{0}^{2} \sqrt{4-r^2}\;r\,dr\,d\theta
&=\int_{0}^{\pi/2}\left. -{1\over3}(4-r^2)^{3/2}\right|_0^2\,d\theta \\
&=\int_{0}^{\pi/2} {8\over3}\,d\theta \\
&={4\pi\over3}. \\
}$$
The surface is a portion of the sphere of radius 2 centered at the
origin, in fact exactly one-eighth of the sphere. We know the formula
for volume of a sphere is $(4/3)\pi r^3$, so the volume we have
computed is $(1/8)(4/3)\pi 2^3=(4/3)\pi$, in agreement with our
answer.
\end{example}
\label{example:integration in polar coordinates}

This example is much like a simple one in rectangular coordinates: the region
of interest may be described exactly by a constant range for
each of the variables. As with rectangular coordinates, we can adapt
the method to deal with more complicated regions.

\begin{example} Find the volume under $z=\sqrt{4-r^2}$ above the region enclosed by the
curve $r=2\cos\theta$, $-\pi/2\le\theta\le\pi/2$; see
figure~\xrefn{fig:volume with variable limits}.
The region is described in polar coordinates by the inequalities
$-\pi/2\le\theta\le\pi/2$ and $0\le r\le2\cos\theta$, so
the double integral is
$$
\int_{-\pi/2}^{\pi/2}\int_{0}^{2\cos\theta} \sqrt{4-r^2}\;r\,dr\,d\theta
=2\int_{0}^{\pi/2}\int_{0}^{2\cos\theta} \sqrt{4-r^2}\;r\,dr\,d\theta.
$$
We can rewrite the integral as shown because of the symmetry of the
volume; this avoids a complication during the evaluation.
Proceeding:
$$\eqalign{
2\int_{0}^{\pi/2}\int_{0}^{2\cos\theta} \sqrt{4-r^2}\;r\,dr\,d\theta
&=2\int_{0}^{\pi/2}-{1\over3}\left.(4-r^2)^{3/2}\right|_0^{2\cos\theta}\,d\theta \\
&=2\int_{0}^{\pi/2}-{8\over3}\sin^3\theta+{8\over3}\,d\theta \\
&=2\left(-{8\over3}{\cos^3\theta\over3}-\cos\theta+{8\over3}\theta\right|_0^{\pi/2} \\
&={8\over3}\pi-{32\over9}. \\
}$$
\vskip-10pt\end{example}

\figure
\vbox{\beginpicture
\normalgraphs
\ninepoint
\setcoordinatesystem units <1.5truecm,1.5truecm>
\setplotarea x from 0 to 2.1, y from -1.1 to 1.1
\axis left  /
\axis bottom shiftedto y=0 /
\circulararc 360 degrees from 2 0 center at 1 0
\put {\hbox{\epsfxsize6cm\epsfbox{cylindrical_volume.eps}}} at 5 0
\endpicture}
\figrdef{fig:volume with variable limits}
\endfigure{Volume over a region with non-constant limits.}

You might have learned a formula for computing areas in polar
coordinates. It is possible to
compute areas as volumes, so that you need only remember one
technique. Consider the surface $z=1$, a horizontal plane. The volume
under this surface and above a region in the $x$-$y$ plane is 
simply $1\cdot(\hbox{area of the region})$, so computing the volume
really just computes the area of the region.

\begin{example} Find the area outside the circle $r=2$ and inside
$r=4\sin\theta$; see figure~\xrefn{fig:area by volume}.
The region is described by $\pi/6\le\theta\le5\pi/6$ and
$2\le r\le4\sin\theta$, so the integral is
$$\eqalign{
\int_{\pi/6}^{5\pi/6}\int_2^{4\sin\theta}1\,r\,dr\,d\theta
&=\int_{\pi/6}^{5\pi/6}\left. {1\over2}r^2\right|_2^{4\sin\theta}\,d\theta \\
&=\int_{\pi/6}^{5\pi/6}8\sin^2\theta-2\,d\theta \\
&={4\over3}\pi+2\sqrt3. \\
}$$
\end{example}

\figure
\hbox{\hfill\tikzpicture[domain=-2:2,x=6mm,y=6mm]
\draw[->] (-2.1,0) -- (2.1,0) ;
\draw[->] (0,-2.1) -- (0,4.1) ;
\draw[color=black] (0,0) circle (2);
\draw[color=black] (0,2) circle (2);
\gpad
\fill[opacity=0.5,fill=red!20]
plot[parametric,id=\the\gpnum,domain=0.5236:2.618]
function{4*sin(t)*cos(t),4*sin(t)*sin(t)} node {\gpad}
plot[parametric,id=\the\gpnum,domain=0.5236:2.618]
function{2*cos(3.1416-t),2*sin(3.1416-t)} -- cycle;
\endtikzpicture\hfill}
\figrdef{fig:area by volume}
\endfigure{Finding area by computing volume.}

\begin{exercises}

\begin{exercise} Find the volume above the $x$-$y$ plane, under the surface
$r^2=2z$, and inside $r=2$.
\begin{answer} $4\pi$
\end{answer}\end{exercise}

\begin{exercise} Find the volume inside both $r=1$ and $r^2+z^2=4$.
\begin{answer} $32\pi/3-4\sqrt3\pi$
\end{answer}\end{exercise}

\begin{exercise} Find the volume below $z=\sqrt{1-r^2}$ and above
the top half of the cone $z=r$.
\begin{answer} $(2-\sqrt2)\pi/3$
\end{answer}\end{exercise}

\begin{exercise} Find the volume below  $z=r$, above the $x$-$y$ plane, and
inside $r=\cos\theta$.
\begin{answer} $4/9$
\end{answer}\end{exercise}

\begin{exercise} Find the volume below  $z=r$, above the $x$-$y$ plane, and
inside $r=1+\cos\theta$.
\begin{answer} $5\pi/3$
\end{answer}\end{exercise}

\begin{exercise} Find the volume between $x^2+y^2=z^2$ and $x^2+y^2=z$.
\begin{answer} $\pi/6$
\end{answer}\end{exercise}

\begin{exercise} Find the area inside $r=1+\sin\theta$ and outside
$r=2\sin\theta$. 
\begin{answer} $\pi/2$
\end{answer}\end{exercise}

\begin{exercise} Find the area inside both
$r=2\sin\theta$ and $r=2\cos\theta$. 
\begin{answer} $\pi/2-1$
\end{answer}\end{exercise}

\begin{exercise} Find the area inside the four-leaf rose $r=\cos(2\theta)$
and outside $r=1/2$.
\begin{answer} $\sqrt3/4+\pi/6$
\end{answer}\end{exercise}

% Albert
\begin{exercise} Find the area inside the cardioid $r=2(1+\cos\theta)$
and outside $r=2$.
\begin{answer} $8+\pi$
\end{answer}\end{exercise}


\begin{exercise}  \relax\label{exer:area of three-leaf rose loop}
Find the area of one loop of the three-leaf rose
 $r=\cos(3\theta)$.
\begin{answer} $\pi/12$\
\end{answer}\end{exercise}

\begin{exercise} Compute $\ds \int_{-3}^3\int_0^{\sqrt{9-x^2}}
\sin(x^2+y^2)\,dy\,dx$ by converting to cylindrical coordinates.
\begin{answer} $(1-\cos(9))\pi/2$
\end{answer}\end{exercise}

\begin{exercise} Compute $\ds \int_{0}^a\int_{-\sqrt{a^2-x^2}}^0 x^2y\,dy\,dx$ 
by converting to cylindrical coordinates.
\begin{answer} $-a^5/15$
\end{answer}\end{exercise}

%/Albert

%% \begin{exercise} Investigate and describe the differences between the graphs of
%%   $r=\cos(2\theta)$ and $r=\sin(2\theta)$.
%% 
%% \begin{exercise} Investigate and describe the differences between the graphs of
%%   $r=\cos(2k\theta)$ and $r=\cos((2k+1)\theta)$

\begin{exercise} Find the volume under $z=y^2+x+2$ above
the region $x^2+y^2\le 4$
\begin{answer} $12\pi$
\end{answer}\end{exercise}

\begin{exercise} Find the volume between
$z=x^2y^3$ and $z=1$ above
the region $x^2+y^2\le 1$
\begin{answer} $\pi$
\end{answer}\end{exercise}

\begin{exercise} Find the volume inside
$x^2+y^2=1$ and $x^2+z^2=1$.
\begin{answer} $16/3$
\end{answer}\end{exercise}


\begin{exercise} Find the volume under $z=r$ above $r=3+\cos\theta$.
\begin{answer} $21\pi$
\end{answer}\end{exercise}

\begin{exercise} Compute $\ds
\int_0^\pi\int_0^{\pi/2}\int_0^1 z\sin x+z\cos y\,dz\,dy\,dx$.
\begin{answer} $\pi$
\end{answer}\end{exercise}

\begin{exercise} Compute $\ds
\tint{R} x+y+z\,dV$, where $R$ is the region inside
$x^2+y^2+z^2\le 1$ in the first octant.
\begin{answer} $\ds {3\pi\over16}$
\end{answer}\end{exercise}

\begin{exercise} Figure \xrefn{fig:double flower} shows the plot of
$r=1+4\sin(5\theta)$.

\figure
\vbox{\beginpicture
\normalgraphs
\ninepoint
\setcoordinatesystem units <2truecm,2truecm>
\setplotarea x from -1 to 1, y from -1 to 1
\put {\hbox{\epsfxsize4cm\epsfbox{double_flower.eps}}} at 0 0
\endpicture}
\figrdef{fig:double flower}
\endfigure{$r=1+4\sin(5\theta)$}

\begin{itemize} % BADBAD

\item{a.} Describe the behavior of the graph in terms of the given
  equation.  Specifically, explain maximum and minimum values, number
  of leaves, and the 'leaves within leaves'.
 
\item{b.} Give an integral or integrals to determine the area outside a
  smaller leaf but inside a larger leaf.


\item{c.} How would changing the value of $a$ in the equation
  $r=1+a\cos(5\theta)$ change the relative sizes of the inner and
  outer leaves? Focus on values $a\geq 1$.  (Hint: How would we change
  the maximum and minimum values?)
 
\end{itemize}

\begin{exercise} Consider the integral $\ds\dint{D} {1\over\sqrt{x^2+y^2}} \;
dA$, where $D$ is the unit disk centered at the origin. (See the graph
\expandafter\url\expandafter{\liveurl jmol_infinite_trumpet}%
here\endurl.)
%\expandafter\url\expandafter{\sageurl 2273}%
%here\endurl.)

\begin{itemize} % BADBAD

\item{a.} Why might this integral be considered improper?

\item{b.} Calculate the value of the integral of the same function
  $\ds 1/\sqrt{x^2+y^2}$ over the annulus with outer radius 1 and
  inner radius $\delta$.

\item{c.} Obtain a value for the integral on the whole disk by letting
  $\delta$ approach 0.
\begin{answer} $2\pi$
\end{answer}\end{exercise}

\item{d.} For which values $\lambda$ can we replace the denominator with
  $(x^2+y^2)^\lambda$ in the original integral?

\end{itemize}

\end{exercises}
