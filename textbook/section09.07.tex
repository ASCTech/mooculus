\section{Kinetic energy; improper integrals}{}{}
\label{sec:improper integrals}
\nobreak
Recall example~\xrefn{example:object to infinity} in which we computed
the work required to lift an object from the surface of the earth to
some large distance $D$ away. Since $\ds F=k/x^2$ we computed
$$\int_{r_0}^D {k\over x^2}\,dx=-{k\over D}+{k\over r_0}.$$
We noticed that as $D$ increases, $k/D$ decreases to zero so that the
amount of work increases to $\ds k/r_0$. More precisely,
$$
  \lim_{D\to\infty}\int_{r_0}^D {k\over x^2}\,dx= 
  \lim_{D\to\infty}-{k\over D}+{k\over r_0}={k\over r_0}.
$$
We might reasonably describe this calculation as computing the amount
of work required to lift the object ``to infinity,'' and abbreviate
the limit as
$$
  \lim_{D\to\infty}\int_{r_0}^D {k\over x^2}\,dx= \int_{r_0}^\infty
  {k\over x^2}\,dx.
$$ 
Such an integral, with a limit of infinity, is called an {\dfont
improper integral\index{improper integral}\index{integral!improper}\/}.
This is a bit unfortunate, since it's not really ``improper'' to do
this, nor is it really ``an integral''---it is an abbreviation for the
limit of a particular sort of integral. Nevertheless, we're stuck with
the term, and the operation itself is perfectly legitimate. It may at
first seem odd that a finite amount of work is sufficient to lift an
object to ``infinity'', but sometimes surprising things are
nevertheless true, and this is such a case. If the value of an
improper integral is a finite number, as in this example, we say that
the integral {\dfont converges\index{improper integral!convergent}},
and if not we say that the integral 
{\dfont diverges\index{improper integral!diverges}}.

Here's another way, perhaps even more surprising, to interpret this
calculation. We know that one interpretation of
$$\int_{1}^D {1\over x^2}\,dx$$
is the area under $\ds y=1/x^2$ from $x=1$ to $x=D$. Of course, as $D$
increases this area increases. But since
$$\int_{1}^D {1\over x^2}\,dx=-{1\over D}+{1\over1},$$
while the area increases, it never exceeds 1, that is
$$\int_{1}^\infty {1\over x^2}\,dx= 1.$$
The area of the infinite region under $\ds y=1/x^2$ from $x=1$ to infinity
is finite.

Consider a slightly different sort of improper integral:
$\ds \int_{-\infty}^\infty xe^{-x^2}\,dx$. There are two ways we might
try to compute this. First, we could break it up into two more
familiar integrals:
$$
  \int_{-\infty}^\infty xe^{-x^2}\,dx=
  \int_{-\infty}^0 xe^{-x^2}\,dx+\int_{0}^\infty xe^{-x^2}\,dx.
$$
Now we do these as before:
$$
  \int_{-\infty}^0 xe^{-x^2}\,dx=\lim_{D\to\infty}
  \left.-{e^{-x^2}\over2}\right|_D^0=-{1\over2},
$$
and
$$
  \int_0^\infty xe^{-x^2}\,dx=\lim_{D\to\infty}
  \left.-{e^{-x^2}\over2}\right|_0^D={1\over2},
$$
so 
$$\ds \int_{-\infty}^\infty xe^{-x^2}\,dx=-{1\over2}+{1\over2}=0.$$
Alternately, we might try
$$
  \int_{-\infty}^\infty xe^{-x^2}\,dx=
  \lim_{D\to\infty}\int_{-D}^D xe^{-x^2}\,dx=
  \lim_{D\to\infty}\left.-{e^{-x^2}\over2}\right|_{-D}^D=
  \lim_{D\to\infty} -{e^{-D^2}\over2}+{e^{-D^2}\over2}=0.
$$
So we get the same answer either way. This does not always happen;
sometimes the second approach gives a finite number, while the first
approach does not; the exercises provide examples. In general, we
interpret the integral $\ds\int_{-\infty}^\infty f(x)\,dx$ according
to the first method: both integrals $\ds\int_{-\infty}^a
f(x)\,dx$ and $\ds\int_{a}^\infty f(x)\,dx$ must converge for the
original integral to converge. The second approach does turn out to be
useful; when $\ds\lim_{D\to\infty}\int_{-D}^D f(x)\,dx=L$, and $L$ is
finite, then $L$ is called the 
{\dfont Cauchy Principal Value\index{Cauchy Principal Value}\/} of
$\ds\int_{-\infty}^\infty f(x)\,dx$.

Here's a more concrete application of these ideas. We know that
in general
$$W=\int_{x_0}^{x_1} F\,dx$$ 
is the work done against the force $F$ in moving from $\ds x_0$ to
$\ds x_1$. In the case that $F$ is the force of gravity exerted by the
earth, it is customary to make $F<0$ since the force is ``downward.''
This makes the work $W$ negative when it should be positive, so
typically the work in this case is defined as
$$W=-\int_{x_0}^{x_1} F\,dx.$$
Also, by Newton's Law, $F=ma(t)$. This means that 
$$W=-\int_{x_0}^{x_1} ma(t)\,dx.$$
Unfortunately this integral is a bit problematic: $a(t)$ is in terms
of $t$, while the limits and the ``$dx$'' are in terms of $x$. But $x$
and $t$ are certainly related here: $x=x(t)$ is the function that
gives the position of the object at time $t$, so $v=v(t)=dx/dt=x'(t)$
is its velocity and $a(t)=v'(t)=x''(t)$. We can use $v=x'(t)$ as a
substitution to convert the integral from ``$dx$'' to ``$dv$'' in the
usual way, with a bit of cleverness along the way:
$$\eqalign{
  dv&=x''(t)\,dt=a(t)\,dt=a(t){dt\over dx}\,dx \\
  {dx\over dt}\,dv&=a(t)\,dx \\
  v\,dv&=a(t)\,dx. \\
}$$ 
Substituting in the integral:
$$
  W=-\int_{x_0}^{x_1} ma(t)\,dx=-\int_{v_0}^{v_1} mv\,dv=
  -\left.{mv^2\over2}\right|_{v_0}^{v_1}=-{mv_1^2\over2}+{mv_0^2\over2}.
$$
You may recall seeing the expression $\ds mv^2/2$ in a physics course---it
is called the {\dfont kinetic energy\index{kinetic energy}\/} of the
object. We have shown here that the work done in moving the object
from one place to another is the same as the change in kinetic energy.

We know that the work required to move an object from the surface
of the earth to infinity is
$$W=\int_{r_0}^\infty {k\over r^2}\,dr={k\over r_0}.$$ 
At the surface of the earth the acceleration due
to gravity is approximately 9.8 meters per second squared, so the
force on an object of mass $m$ is $F=9.8m$. The radius of the earth is
approximately 6378.1 kilometers or 6378100 meters. Since the force due
to gravity obeys an inverse square law, $\ds F=k/r^2$ and
$\ds 9.8m=k/6378100^2$, $k= 398665564178000m$ and
$W=62505380 m$.

Now suppose that the initial velocity of the object, $\ds v_0$, is just
enough to get it to infinity, that is, just enough so that the object
never slows to a stop, but so that its speed decreases to zero, i.e., so
that $\ds v_1=0$. Then 
$$62505380 m=W=-{mv_1^2\over2}+{mv_0^2\over2}={mv_0^2\over2}$$
so
$$v_0=\sqrt{125010760}\approx 11181\quad\hbox{meters per second},$$
or about 40251 kilometers per hour. This speed is called the {\dfont
escape velocity\index{escape velocity}\/}. Notice that the mass of
the object, $m$, canceled out at the last step; the escape velocity
is the same for all objects. Of course, it takes considerably more
energy to get a large object up to 40251 kph than a small one, so it
is certainly more difficult to get a large object into deep space than
a small one. Also, note that while we have computed the escape
velocity for the earth, this speed would not in fact get an object
``to infinity'' because of the large mass in our neighborhood called
the sun. Escape velocity for the sun {\em starting at the distance of
the earth from the sun\/} is nearly 4 times the escape velocity we
have calculated.

\begin{exercises}

\begin{exercise} Is the area under $y=1/x$ from 1 to infinity finite or
infinite? If finite, compute the area.
\begin{answer} $\infty$
\end{answer}\end{exercise}

\begin{exercise} Is the area under $\ds y=1/x^3$ from 1 to infinity finite or
infinite? If finite, compute the area.
\begin{answer} $1/2$
\end{answer}\end{exercise}

\begin{exercise} Does $\ds\int_0^\infty x^2+2x-1\,dx$ converge or diverge? If
it converges, find the value.
\begin{answer} diverges 
\end{answer}\end{exercise}

\begin{exercise} Does $\ds\int_1^\infty 1/\sqrt{x}\,dx$ converge or diverge? If
it converges, find the value.
\begin{answer} diverges
\end{answer}\end{exercise}

%wills
\begin{exercise} Does $\ds\int_0^\infty e^{-x }\,dx$ converge or diverge? If
it converges, find the value.
\begin{answer} 1
\end{answer}\end{exercise}

\begin{exercise} $\ds\int_0^{1/2} (2x-1)^{-3}\,dx$ is an improper integral of
a slightly different sort. Express it as a limit and determine whether
it converges or diverges; if 
it converges, find the value.
\begin{answer} diverges
\end{answer}\end{exercise}

\begin{exercise} Does $\ds\int_0^1 1/\sqrt{x}\,dx$ converge or diverge? If
it converges, find the value.
\begin{answer} $2$
\end{answer}\end{exercise}

%wills
\begin{exercise} Does $\ds\int_0^{\pi/2} \sec^2x\,dx$ converge or diverge? If
it converges, find the value.
\begin{answer} diverges
\end{answer}\end{exercise}

%wills
\begin{exercise} Does $\ds\int_{-\infty}^\infty{x^2\over 4+x^6}\,dx$ 
converge or diverge? If
it converges, find the value.
\begin{answer} $\pi/6$
\end{answer}\end{exercise}

%wills
\begin{exercise} Does $\ds\int_{-\infty}^\infty x\,dx$ 
converge or diverge? If
it converges, find the value. Also find the Cauchy Principal Value, if
it exists.
\begin{answer} diverges, $0$
\end{answer}\end{exercise}

%wills
\begin{exercise} Does $\ds\int_{-\infty}^\infty \sin x\,dx$ 
converge or diverge? If
it converges, find the value. Also find the Cauchy Principal Value, if
it exists.
\begin{answer} diverges, $0$
\end{answer}\end{exercise}

\begin{exercise} Does $\ds\int_{-\infty}^\infty \cos x\,dx$ 
converge or diverge? If
it converges, find the value. Also find the Cauchy Principal Value, if
it exists.
\begin{answer} diverges, no CPV
\end{answer}\end{exercise}

\begin{exercise} \relax
\label{exer:gabriels horn}
Suppose the curve $y=1/x$ is rotated around the $x$-axis
generating a sort of funnel or horn shape, called {\dfont
  Gabriel's horn\index{Gabriel's horn}\/} or {\dfont Toricelli's
  trumpet\index{Toricelli's trumpet}}. Is the volume of this funnel
from $x=1$ to infinity finite or infinite? If finite, compute the
volume.  
\begin{answer} $\pi$ 
\end{answer}\end{exercise}

\begin{exercise} An officially sanctioned baseball must be between 142 and
149 grams. How much work, in Newton-meters, does it take to throw a
ball at 80 miles per hour? At 90 mph? At 100.9 mph?  (According to the
Guinness Book of World Records, at
\url{http://www.baseball-almanac.com/recbooks/rb_guin.shtml}
{\vb|http://www.baseball-almanac.com/recbooks/rb_guin.shtml|}\endurl, ``The
greatest reliably recorded speed at which a baseball has been pitched
is 100.9 mph by Lynn Nolan Ryan (California Angels) at Anaheim Stadium
in California on August 20, 1974.'')
\begin{answer} 80 mph: $90.8$ to $95.3$ N\hfill\break
90 mph: $114.9$ to $120.6$ N\hfill\break
$100.9$ mph: $144.5$ to $151.6$ N
\end{answer}\end{exercise}

\end{exercises}
