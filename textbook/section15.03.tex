\section{Moment and Center of Mass}{}{}

Using a single integral we were able to compute the center of mass for
a one-dimensional object with variable density, and a two dimensional
object with constant density. With a double integral we can handle two
dimensions and variable density.

Just as before, the coordinates of the center of mass are
$$\bar x={M_y\over M} \qquad \bar y={M_x\over M},$$
where $M$ is the total mass, $M_y$ is the moment\index{moment}
 around the $y$-axis,
and $M_x$ is the moment around the $x$-axis. (You may want to review
the concepts in section~\xrefn{sec:center of mass}.) 

The key to the computation, just as before, is the approximation of
mass. In the two-dimensional case, we treat density $\sigma$ as mass
per square area, so when density is constant, mass is 
$(\hbox{density})(\hbox{area})$. If we have a two-dimensional region
with varying density given by $\sigma(x,y)$, and we divide the region
into small subregions with area $\Delta A$, then the mass of one
subregion is approximately $\sigma(x_i,y_j)\Delta A$, the total mass
is approximately the sum of many of these, 
and as usual the sum
turns into an integral in the limit:
$$M=\int_{x_0}^{x_1}\int_{y_0}^{y_1} \sigma(x,y)\,dy\,dx,$$
and similarly for computations in cylindrical coordinates.
Then as before
$$\eqalign{
M_x &= \int_{x_0}^{x_1}\int_{y_0}^{y_1} y\sigma(x,y)\,dy\,dx \\
M_y &= \int_{x_0}^{x_1}\int_{y_0}^{y_1} x\sigma(x,y)\,dy\,dx. \\
}$$

\begin{example} Find the center of mass of a thin, uniform plate whose shape
is the region between $y=\cos x$ and the $x$-axis between $x=-\pi/2$
and $x=\pi/2$. Since the density is constant, we may take
$\sigma(x,y)=1$. 

It is clear that $\bar x=0$, but for practice let's
compute it anyway. First we compute the mass:
$$
M=\int_{-\pi/2}^{\pi/2} \int_0^{\cos x} 1\,dy\,dx
=\int_{-\pi/2}^{\pi/2} \cos x\,dx
=\left.\sin x\right|_{-\pi/2}^{\pi/2}=2.
$$
Next,
$$
M_x=\int_{-\pi/2}^{\pi/2} \int_0^{\cos x} y\,dy\,dx
=\int_{-\pi/2}^{\pi/2} {1\over2}\cos^2 x\,dx={\pi\over4}.
$$
Finally,
$$
M_y=\int_{-\pi/2}^{\pi/2} \int_0^{\cos x} x\,dy\,dx
=\int_{-\pi/2}^{\pi/2} x\cos x\,dx=0.
$$
So $\bar x=0$ as expected, and $\bar y=\pi/4/2=\pi/8$. 
This is the same problem as in example~\xrefn{exam:center of mass
  under cos}; it may be helpful to compare the two solutions.
\end{example}

\begin{example} Find the center of mass of a two-dimensional plate 
that occupies the quarter circle $x^2+y^2\le1$ in the
first quadrant and has density
$k(x^2+y^2)$. It seems clear that because of the symmetry of both the
region and the density function (both are important!), $\bar x=\bar
y$. We'll do both to check our work.

Jumping right in:
$$
M=\int_0^1 \int_0^{\sqrt{1-x^2}} k(x^2+y^2)\,dy\,dx
=k\int_0^1 x^2\sqrt{1-x^2}+{(1-x^2)^{3/2}\over3}\,dx.
$$
This integral is something we can do, but it's a bit unpleasant. Since
everything in sight is related to a circle, let's back up and try
polar coordinates. Then $x^2+y^2=r^2$ and
$$M=\int_0^{\pi/2} \int_0^{1} k(r^2)\,r\,dr\,d\theta
=k\int_0^{\pi/2}\left.{r^4\over4}\right|_0^1\,d\theta
=k\int_0^{\pi/2} {1\over4}\,d\theta
=k{\pi\over8}.
$$
Much better. Next, since $y=r\sin\theta$,
$$M_x=k\int_0^{\pi/2} \int_0^{1} r^4\sin\theta\,dr\,d\theta
=k\int_0^{\pi/2} {1\over5}\sin\theta\,d\theta
=k\left.-{1\over5}\cos\theta\right|_0^{\pi/2}={k\over5}.
$$
Similarly,
$$M_y=k\int_0^{\pi/2} \int_0^{1} r^4\cos\theta\,dr\,d\theta
=k\int_0^{\pi/2} {1\over5}\cos\theta\,d\theta
=k\left.{1\over5}\sin\theta\right|_0^{\pi/2}={k\over5}.
$$
Finally, $\ds\bar x = \bar y = {8\over5\pi}$.
\end{example}

\begin{exercises}

\begin{exercise} Find the center of mass of a two-dimensional plate 
that occupies the square $[0,1]\times[0,1]$
and has density
function $xy$.
\begin{answer} $\bar x=\bar y=2/3$
\end{answer}\end{exercise}

\begin{exercise} Find the center of mass of a two-dimensional plate 
that occupies the triangle $0\le x\le1$, $0\le y\le x$,
and has density
function $xy$.
\begin{answer} $\bar x=4/5$, $\bar y=8/15$
\end{answer}\end{exercise}

\begin{exercise} Find the center of mass of a two-dimensional plate 
that occupies the upper unit semicircle centered at $(0,0)$
and has density
function $y$.
\begin{answer} $\bar x=0$, $\bar y=3\pi/16$
\end{answer}\end{exercise}

\begin{exercise} Find the center of mass of a two-dimensional plate 
that occupies the upper unit semicircle centered at $(0,0)$
and has density
function $x^2$.
\begin{answer} $\bar x=0$, $\bar y=16/(15\pi)$
\end{answer}\end{exercise}

% Albert
\begin{exercise} Find the center of mass of a two-dimensional plate 
that occupies the triangle formed by $x=2$, $y=x$, and $y=2x$
and has density
function $2x$.
\begin{answer} $\bar x=3/2$, $\bar y=9/4$
\end{answer}\end{exercise}

\begin{exercise} Find the center of mass of a two-dimensional plate 
that occupies the triangle formed by $x=0$, $y=x$, and $2x+y=6$
and has density
function $\ds x^2$.
\begin{answer} $\bar x=6/5$, $\bar y=12/5$
\end{answer}\end{exercise}

\begin{exercise} Find the center of mass of a two-dimensional plate 
that occupies the region enclosed by the parabolas $x=y^2$, $y=x^2$
and has density
function $\ds\sqrt{x}$.
\begin{answer} $\bar x=14/27$, $\bar y=28/55$
\end{answer}\end{exercise}

\begin{exercise} Find the centroid of the area in the first quadrant bounded by
 $x^2-8y+4=0$, $x^2=4y$, and $x=0$. (Recall that the centroid\index{centroid}
is the center of mass when the density is 1 everywhere.)
\begin{answer} $(3/4,2/5)$
\end{answer}\end{exercise}

\begin{exercise} Find the centroid of one loop of the three-leaf rose
$r=\cos(3\theta)$.  (Recall that the centroid\index{centroid} is the
center of mass when the density is 1 everywhere, and that the mass in
this case is the same as the area, which was the subject of
exercise~\xrefn{exer:area of three-leaf rose loop} in
section~\xrefn{sec:Double Integrals in Cylindrical Coordinates}.)  The
computations of the integrals for the moments $M_x$ and $M_y$ are
elementary but quite long; Sage can help.
\begin{answer} $\ds\left({81\sqrt3\over80\pi},0\right)$
\end{answer}\end{exercise}

%/Albert

\begin{exercise} Find the center of mass of a two dimensional
object that occupies the region $0\le x\le \pi$, $0\le y\le \sin x$,
with density $\sigma=1$.
\begin{answer} $\bar x=\pi/2$, $\bar y=\pi/8$ 
\end{answer}\end{exercise}

\begin{exercise} A two-dimensional object has shape given by 
$r=1+\cos\theta$ and density $\sigma(r,\theta)=2+\cos\theta$. Set up
the three integrals required to compute the center of mass.
\begin{answer} $\ds M=\int_0^{2\pi} \int_0^{1+\cos\theta} (2+\cos\theta)r\,dr\,d\theta$,
\hfill\break
$\ds M_x=\int_0^{2\pi} \int_0^{1+\cos\theta} \sin\theta(2+\cos\theta)r^2\,dr\,d\theta$,
\hfill\break
$\ds M_y=\int_0^{2\pi} \int_0^{1+\cos\theta} \cos\theta(2+\cos\theta)r^2\,dr\,d\theta$.
\end{answer}\end{exercise}

\begin{exercise} A two-dimensional object has shape given by 
$r=\cos\theta$ and density $\sigma(r,\theta)=r+1$. Set up
the three integrals required to compute the center of mass.
\begin{answer} $\ds M=\int_{-\pi/2}^{\pi/2} \int_0^{\cos\theta} (r+1)r\,dr\,d\theta$,
\hfill\break
$\ds M_x=\int_{-\pi/2}^{\pi/2} \int_0^{\cos\theta} \sin\theta(r+1)r^2\,dr\,d\theta$,
\hfill\break
$\ds M_y=\int_{-\pi/2}^{\pi/2} \int_0^{\cos\theta} \cos\theta(r+1)r^2\,dr\,d\theta$.
\end{answer}\end{exercise}

\begin{exercise} A two-dimensional object sits inside $r=1+\cos\theta$
and outside $r=\cos\theta$, and has density $1$ everywhere.
Set up
the integrals required to compute the center of mass.
\begin{answer} $\ds M= \int_{-\pi/2}^{\pi/2}\int_{\cos\theta}^{1+\cos\theta}
r\,dr\,d\theta + \int_{\pi/2}^{3\pi/2}\int_0^{1+\cos\theta}r\,dr\,d\theta$,
\hfill\break
$\ds M_x=\int_{-\pi/2}^{\pi/2}\int_{\cos\theta}^{1+\cos\theta}
r^2\sin\theta\,dr\,d\theta + \int_{\pi/2}^{3\pi/2}\int_0^{1+\cos\theta}r^2\sin\theta\,dr\,d\theta$,
\hfill\break
$\ds M_y=\int_{-\pi/2}^{\pi/2}\int_{\cos\theta}^{1+\cos\theta}
r^2\cos\theta\,dr\,d\theta + \int_{\pi/2}^{3\pi/2}\int_0^{1+\cos\theta}r^2\cos\theta\,dr\,d\theta$.
\end{answer}\end{exercise}

\end{exercises}
