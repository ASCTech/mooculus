\section{Area between curves}{}{}
\nobreak
We have seen how integration can be used to find an area between a
curve and the $x$-axis. With very little change we can find some areas
between curves; indeed, the area between a curve and the $x$-axis may
be interpreted as the area between the curve and a second ``curve''
with equation $y=0$. In the simplest of cases, the idea is quite easy
to understand.

\begin{example} Find the area below $\ds f(x)= -x^2+4x+3$ and above
$\ds g(x)=-x^3+7x^2-10x+5$ over the interval $1\le x\le2$. In
figure~\xrefn{fig:area between curves} we show the two curves together, with
the desired area shaded, then $f$ alone with the area under $f$
shaded, and then $g$ alone with the area under $g$ shaded. 
\figure
\hbox to \hsize{\hfill
\def\yarrow{-- +(-1.5pt,-3pt) +(0pt,0pt) -- +(1.5pt,-3pt) +(0pt,0pt)}
\def\xarrow{-- +(-3pt,-1.5pt) +(0pt,0pt) -- +(-3pt,1.5pt) +(0pt,0pt) }
\tikzpicture[domain=0:3,y=3mm]
\draw[->] (0,0) -- (3.2,0) \xarrow node [right] {$x$};
\draw[->] (0,0) -- (0,11.2) \yarrow node [above] {$y$};
\gpad
\draw[color=black] plot[id=\the\gpnum,domain=0:3] function{-x**3+7*x**2-10*x+5};
\gpad
\draw[color=black] plot[id=\the\gpnum,domain=0:3] function{-x**2+4*x+3};
\draw[dashed] (1,0) -- (1,6);
\draw[dashed] (2,0) -- (2,7);
\foreach \x in {0,1,2,3} \draw (\x,0) -- (\x,-2pt) node[anchor=north] {\eightpoint $\x$};
\foreach \y in {0,5,10} \draw (0,\y) -- (-2pt,\y) node[anchor=east]
         {\eightpoint $\y$};
\gpad
\fill[opacity=0.5,fill=red!20] plot[id=\the\gpnum,domain=1:2]
function{-x**3+7*x**2-10*x+5} -- (2,7) node {\gpad}
plot[parametric,id=\the\gpnum,domain=0:1] function{2-t,-t**2+7} -- (1,1);
\endtikzpicture

\tikzpicture[domain=0:3,y=3mm]
\draw[->] (0,0) -- (3.2,0) \xarrow node [right] {$x$};
\draw[->] (0,0) -- (0,11.2) \yarrow node [above] {$y$};
\gpad
\draw[color=black] plot[id=\the\gpnum,domain=0:3] function{-x**2+4*x+3};
\draw[dashed] (1,0) -- (1,6);
\draw[dashed] (2,0) -- (2,7);
\foreach \x in {0,1,2,3} \draw (\x,0) -- (\x,-2pt) node[anchor=north] {\eightpoint $\x$};
\foreach \y in {0,5,10} \draw (0,\y) -- (-2pt,\y) node[anchor=east] {\eightpoint $\y$};
\gpad\fill[opacity=0.5,fill=red!20] (2,0) -- (2,7)
plot[parametric,id=\the\gpnum,domain=0:1] function{2-t,-t**2+7} -- (1,0) -- (2,0);
\endtikzpicture

\tikzpicture[domain=0:3,y=3mm]
\draw[angle 90] (0,0) -- (3.2,0) \xarrow node [right] {$x$};
\draw[->] (0,0) -- (0,11.2) \yarrow node [above] {$y$};
\gpad\draw[color=black] plot[id=\the\gpnum,domain=0:3] function{-x**3+7*x**2-10*x+5};
\draw[dashed] (1,0) -- (1,1);
\draw[dashed] (2,0) -- (2,5);
\foreach \x in {0,1,2,3} \draw (\x,0) -- (\x,-2pt) node[anchor=north] {\eightpoint $\x$};
\foreach \y in {0,5,10} \draw (0,\y) -- (-2pt,\y) node[anchor=east]
         {\eightpoint $\y$};
\gpad
\fill[opacity=0.5,fill=red!20] (1,0) -- (1,1) plot[id=\the\gpnum,domain=1:2]
function{-x**3+7*x**2-10*x+5} -- (2,0) -- (1,0);
\endtikzpicture
\hfill}
\figrdef{fig:area between curves}
\endfigure{Area between curves as a difference of areas.}

It is clear
from the figure that the area we want is the area under $f$ minus the
area under $g$, which is to say
$$\int_1^2 f(x)\,dx-\int_1^2 g(x)\,dx = \int_1^2 f(x)-g(x)\,dx.$$
It doesn't matter whether we compute the two integrals on the left and
then subtract or compute the single integral on the right. In this
case, the latter is perhaps a bit easier:
$$\eqalign{
  \int_1^2 f(x)-g(x)\,dx&=\int_1^2 -x^2+4x+3-(-x^3+7x^2-10x+5)\,dx \\
  &=\int_1^2 x^3-8x^2+14x-2\,dx \\
  &=\left.{x^4\over4}-{8x^3\over3}+7x^2-2x\right|_1^2 \\
  &={16\over4}-{64\over3}+28-4-({1\over4}-{8\over3}+7-2) \\
  &=23-{56\over3}-{1\over4}={49\over12}. \\
}$$
\vskip-10pt\end{example}

It is worth examining this problem a bit more. We have seen one way to
look at it, by viewing the desired area as a big area minus a small
area, which leads naturally to the difference between two
integrals. But it is instructive to consider how we might find the
desired area directly. We can approximate the area by dividing the
area into thin sections and approximating the area of each section by
a rectangle, as indicated in 
figure~\xrefn{fig:rectangles between curves}. 
The area of a typical rectangle is 
$\Delta x(f(x_i)-g(x_i))$, so the total area is approximately
$$\sum_{i=0}^{n-1} (f(x_i)-g(x_i))\Delta x.$$
This is exactly the sort of sum that turns into an integral in the
limit, namely the integral
$$\int_1^2 f(x)-g(x)\,dx.$$
Of course, this is the integral we actually computed above, but we
have now arrived at it directly rather than as a modification of the
difference between two other integrals. In that example it really
doesn't matter which approach we take, but in some cases this second
approach is better.

\figure
\vbox{\beginpicture
\normalgraphs
\ninepoint
\setcoordinatesystem units <3truecm,0.4truecm>
\setplotarea x from 0 to 3, y from 0 to 10
\axis left ticks numbered from 0 to 10 by 5 /
\axis bottom  ticks numbered from 0 to 3 by 1 /
\setquadratic
%\putrule from 1.52 2.707 to 1.52 6.819
%\putrule from 1.63 2.707 to 1.63 6.819
%\putrule from 1.52 2.707 to 1.63 2.707
%\putrule from 1.52 6.819 to 1.63 6.819
\putrule from 1.465 2.229 to 1.465 6.714
\putrule from 1.575 2.229 to 1.575 6.714
\putrule from 1.575 2.229 to 1.465 2.229
\putrule from 1.575 6.714 to 1.465 6.714
%
\putrule from 1.575 2.707 to 1.575 6.819
\putrule from 1.685 2.707 to 1.685 6.819
\putrule from 1.575 2.707 to 1.685 2.707
\putrule from 1.575 6.819 to 1.685 6.819
\plot
0.000 3.000 0.075 3.294 0.150 3.578 0.225 3.849 0.300 4.110 
0.375 4.359 0.450 4.598 0.525 4.824 0.600 5.040 0.675 5.244 
0.750 5.438 0.825 5.619 0.900 5.790 0.975 5.949 1.050 6.098 
1.125 6.234 1.200 6.360 1.275 6.474 1.350 6.578 1.425 6.669 
1.500 6.750 1.575 6.819 1.650 6.878 1.725 6.924 1.800 6.960 
1.875 6.984 1.950 6.998 2.025 6.999 2.100 6.990 2.175 6.969 
2.250 6.938 2.325 6.894 2.400 6.840 2.475 6.774 2.550 6.698 
2.625 6.609 2.700 6.510 2.775 6.399 2.850 6.278 2.925 6.144 
3.000 6.000 /
\plot
0.000 5.000 0.075 4.289 0.150 3.654 0.225 3.093 0.300 2.603 
0.375 2.182 0.450 1.826 0.525 1.535 0.600 1.304 0.675 1.132 
0.750 1.016 0.825 0.953 0.900 0.941 0.975 0.978 1.050 1.060 
1.125 1.186 1.200 1.352 1.275 1.557 1.350 1.797 1.425 2.071 
1.500 2.375 1.575 2.707 1.650 3.065 1.725 3.446 1.800 3.848 
1.875 4.268 1.950 4.703 2.025 5.151 2.100 5.609 2.175 6.075 
2.250 6.547 2.325 7.021 2.400 7.496 2.475 7.968 2.550 8.436 
2.625 8.896 2.700 9.347 2.775 9.785 2.850 10.208 2.925 10.614 
3.000 11.000 /
\setdashes
\putrule from 1 0 to 1 6
\putrule from 2 0 to 2 7
\setsolid
\endpicture}
\figrdef{fig:rectangles between curves}
\endfigure{Approximating area between curves with rectangles.}

\begin{example} Find the area below $\ds f(x)= -x^2+4x+1$ and above
$\ds g(x)=-x^3+7x^2-10x+3$ over the interval $1\le x\le2$; these are the
same curves as before but lowered by 2. In
figure~\xrefn{fig:area between curves too} we show the two curves
together. Note that the lower curve now dips below the $x$-axis. This
makes it somewhat tricky to view the desired area as a big area minus
a smaller area, but it is just as easy as before to think of
approximating the area by rectangles. The height of a typical
rectangle will still be $f(x_i)-g(x_i)$, even if $g(x_i)$ is
negative. Thus the area is 
$$
  \int_1^2 -x^2+4x+1-(-x^3+7x^2-10x+3)\,dx
  =\int_1^2 x^3-8x^2+14x-2\,dx.
$$
This is of course the same integral as before, because the region
between the curves is identical to the former region---it has just
been moved down by 2.
\end{example}

\figure
\hbox{\hfill
\tikzpicture[domain=0:3,y=3mm]
\draw[->] (0,0) -- (3.2,0) ;
\draw[->] (0,0) -- (0,10.2) ;
\gpad
\draw[color=black] plot[id=\the\gpnum,domain=0:3] function{-x**3+7*x**2-10*x+3};
\gpad
\draw[color=black] plot[id=\the\gpnum,domain=0:3] function{-x**2+4*x+1};
\draw[dashed] (1,0) -- (1,4);
\draw[dashed] (2,0) -- (2,5);
\foreach \x in {0,1,2,3} \draw (\x,0) -- (\x,-2pt) node[anchor=north] {\eightpoint $\x$};
\foreach \y in {0,5,10} \draw (0,\y) -- (-2pt,\y) node[anchor=east] {\eightpoint $\y$};
\gpad
\fill[opacity=0.5,fill=red!20] plot[id=\the\gpnum,domain=1:2]
function{-x**3+7*x**2-10*x+3} -- (2,5) node {\gpad}
plot[parametric,id=\the\gpnum,domain=0:1] function{2-t,-t**2+5} -- (1,-1);
\endtikzpicture
\hfill}
\figrdef{fig:area between curves too}
\endfigure{Area between curves.}

\begin{example} Find the area between $\ds f(x)= -x^2+4x$ and
$\ds g(x)=x^2-6x+5$ over the interval $0\le x\le 1$; the
curves are shown in figure~\xrefn{fig:curves cross}. Generally we
should interpret ``area'' in the usual sense, as a necessarily
positive quantity. Since the two curves cross, we need to compute two 
areas and add them. First we find the intersection point of the
curves:
$$\eqalign{
  -x^2+4x&=x^2-6x+5 \\
  0&=2x^2-10x+5 \\
  x&={10\pm\sqrt{100-40}\over4}={5\pm\sqrt{15}\over2}. \\
}$$
The intersection point we want is $\ds x=a=(5-\sqrt{15})/2$. Then
the total area is 
$$\eqalign{
  \int_0^a x^2-6x+5-(-x^2+4x)\,dx&+\int_a^1 -x^2+4x-(x^2-6x+5)\,dx \\
  &=\int_0^a 2x^2-10x+5\,dx+\int_a^1 -2x^2+10x-5\,dx \\
  &=\left.{2x^3\over3}-5x^2+5x\right|_0^a + 
    \left.-{2x^3\over3}+5x^2-5x\right|_a^1 \\
  &=-{52\over3}+5\sqrt{15},
}$$
after a bit of simplification.
\end{example}

\figure
\hbox{\hfill\tikzpicture[domain=0:1,x=3cm,y=1cm]
\draw[->] (0,0) -- (1.1,0) ;
\draw[->] (0,0) -- (0,5.2) ;
\gpad
\draw[color=black] plot[id=\the\gpnum,domain=0:1] function{x**2-6*x+5};
\gpad
\draw[color=black] plot[id=\the\gpnum,domain=0:1] function{-x**2+4*x};
\draw[dashed] (1,0) -- (1,3);
\foreach \x in {0,1} \draw (\x,0) -- (\x,-2pt) node[anchor=north] {\eightpoint $\x$};
\foreach \y in {0,1,2,3,4,5} \draw (0,\y) -- (-2pt,\y)
node[anchor=east] {\eightpoint $\y$};
\gpad
\fill[opacity=0.5,fill=red!20] plot[id=\the\gpnum,domain=0:0.564]
function{-x**2+4*x} node {\gpad}
plot[parametric,id=\the\gpnum,domain=0.564:0]
function{t,(t)**2-6*(t)+5} -- (0,0);
\gpad
\fill[opacity=0.5,fill=red!20] plot[id=\the\gpnum,domain=0.564:1]
function{-x**2+4*x} node {\gpad}
-- (1,0) plot[parametric,id=\the\gpnum,domain=1:0.564] function{t,(t)**2-6*(t)+5};
\endtikzpicture\hfill}
\figrdef{fig:curves cross}
\endfigure{Area between curves that cross.}

\begin{example} Find the area between $\ds f(x)= -x^2+4x$ and
$\ds g(x)=x^2-6x+5$; the
curves are shown in figure~\xrefn{fig:area bounded by curves}. Here we
are not given a specific interval, so it must be the case that there
is a ``natural'' region involved. Since the curves are both parabolas,
the only reasonable interpretation is the region between the two
intersection points, which we found in the previous example:
$${5\pm\sqrt{15}\over2}.$$
If we let $\ds a=(5-\sqrt{15})/2$ and $\ds b=(5+\sqrt{15})/2$,
the total area is 
$$\eqalign{
  \int_a^b -x^2+4x-(x^2-6x+5)\,dx
  &=\int_a^b -2x^2+10x-5\,dx \\
  &=\left.-{2x^3\over3}+5x^2-5x\right|_a^b \\
  &=5\sqrt{15}. \\
}$$
after a bit of simplification.
\end{example}

\figure
\vbox{\beginpicture
\normalgraphs
\ninepoint
\setcoordinatesystem units <1.7truecm,0.4truecm>
\setplotarea x from 0 to 5, y from -5 to 5
\axis left ticks numbered from -5 to 5 by 5 /
\axis bottom shiftedto y=0 ticks numbered from 1 to 5 by 1 /
\setquadratic
\plot
0.000 0.000 0.250 0.938 0.500 1.750 0.750 2.438 1.000 3.000 
1.250 3.438 1.500 3.750 1.750 3.938 2.000 4.000 2.250 3.938 
2.500 3.750 2.750 3.438 3.000 3.000 3.250 2.438 3.500 1.750 
3.750 0.938 4.000 0.000 4.250 -1.062 4.500 -2.250 4.750 -3.562 
5.000 -5.000 /
\plot
0.000 5.000 0.250 3.562 0.500 2.250 0.750 1.062 1.000 0.000 
1.250 -0.938 1.500 -1.750 1.750 -2.438 2.000 -3.000 2.250 -3.438 
2.500 -3.750 2.750 -3.938 3.000 -4.000 3.250 -3.938 3.500 -3.750 
3.750 -3.438 4.000 -3.000 4.250 -2.438 4.500 -1.750 4.750 -0.938 
5.000 0.000 /
\endpicture}
\figrdef{fig:area bounded by curves}
\endfigure{Area bounded by two curves.}

\begin{exercises}

Find the area bounded by the curves.
\nobreak
\begin{exercise} $\ds y=x^4-x^2$ and $\ds y=x^2$ (the part to the right of the $y$-axis)
\begin{answer} $\ds 8\sqrt2/15$
\end{answer}\end{exercise}

\begin{exercise} $\ds x=y^3$ and $\ds x=y^2$
\begin{answer} $1/12$
\end{answer}\end{exercise}

\begin{exercise} $\ds x=1-y^2$ and $y=-x-1$
\begin{answer} $9/2$
\end{answer}\end{exercise}

\begin{exercise} $\ds x=3y-y^2$ and $x+y=3$
\begin{answer} $4/3$
\end{answer}\end{exercise}

\begin{exercise} $y=\cos(\pi x/2)$ and $\ds y=1- x^2$ (in the first quadrant)
\begin{answer} $2/3-2/\pi$
\end{answer}\end{exercise}

\begin{exercise} $y=\sin(\pi x/3)$ and $y=x$ (in the first quadrant)
\begin{answer} $\ds 3/\pi - 3\sqrt3/(2\pi)-1/8$
\end{answer}\end{exercise}

\begin{exercise} $\ds y=\sqrt{x}$ and $\ds y=x^2$
\begin{answer} $1/3$
\end{answer}\end{exercise}

\begin{exercise} $\ds y=\sqrt x$ and $\ds y=\sqrt{x+1}$, $0\le x\le 4$
\begin{answer} $\ds 10\sqrt{5}/3-6$
\end{answer}\end{exercise}

\begin{exercise} $x=0$ and $\ds x=25-y^2$
\begin{answer} $500/3$
\end{answer}\end{exercise}

\begin{exercise} $y=\sin x\cos x$ and $y=\sin x$, $0\le x\le \pi$
\begin{answer} $2$
\end{answer}\end{exercise}

\begin{exercise} $\ds y=x^{3/2}$ and $\ds y=x^{2/3}$
\begin{answer} $1/5$
\end{answer}\end{exercise}

\begin{exercise} $\ds y=x^2-2x$ and $y=x-2$
\begin{answer} $1/6$
\end{answer}\end{exercise}

\iflatetranscendentals
\else
\vskip15pt\noindent
The following three exercises expand on the geometric interpretation
of the hyperbolic functions. Refer to section~\xrefn{sec:hyperbolic functions}
and particularly to~figure~\xrefn{fig:geometric def trigh} and 
exercise~\xrefn{exer:hyperbolic double angle formulas}
in section~\xrefn{sec:hyperbolic functions}.

\begin{exercise}
Compute $\ds \int \sqrt{x^2 -1}\,dx $ using the substitution
$u=\arccosh x$, or $x=\cosh u$; 
use exercise~\xrefn{exer:hyperbolic double angle formulas}
in section ~\xrefn{sec:hyperbolic functions}.

\begin{exercise}  Fix $t>0$.
  Sketch the region $R$ in the right half plane bounded by the curves 
$y=x\tanh t$, $y=-x\tanh t$,  and $\ds x^2-y^2 =1$. Note well: $t$ is fixed,
the plane is the $x$-$y$ plane.

\begin{exercise} Prove that the area of $R$ is $t$.
\fi

\end{exercises}

