\chapter{Limits}

\section{Functions}{}{}

\fbox{This should be one of the last sections filled-in.}


\section{The Basic Ideas of Limits}{}{}

Consider the function:
\[
f(x) = \frac{x^2 - 3x + 2}{x-2}
\]
While $f(x)$ is undefined at $x=2$, we can still plot $f(x)$ at other
values, see Figure~\ref{plot:(x^2 - 3x + 2)/(x-2)}. Examining
Table~\ref{table:(x^2 - 3x + 2)/(x-2)}, we see that as $x$ approaces
$2$, $f(x)$ approaches $1$. We write this: As $x \to 2$, $f(x) \to 1$.
This leads us to the definition of a \textit{limit}.

\begin{marginfigure}
\begin{tikzpicture}
	\begin{axis}[width=190pt,axis x line=middle, axis y line=center, tick align=outside]
	  \addplot+[mark=none,smooth] {(x^2-3*x+2)/(x-2)};
	\end{axis}
\end{tikzpicture}
  \caption{A plot of $f(x)=\protect\frac{x^2 - 3x + 2}{x-2}$.}
  \label{plot:(x^2 - 3x + 2)/(x-2)}
\end{marginfigure}

\begin{margintable}
\[
\begin{array}{c|c}
 x & f(x) \\ \hline
 1.5 &  .5 \\
 1.7 &  .7 \\
 1.9 &  .9 \\
 2.1 &  1.1 \\
 2.3 &  1.3 \\
 2.5 &  1.5 \\
\end{array}
\]
\caption{Values of $f(x)=\protect\frac{x^2 - 3x + 2}{x-2}$.}
\label{table:(x^2 - 3x + 2)/(x-2)}
\end{margintable}


\begin{definition}\label{def:limit} 
The \textbf{limit} of $f(x)$ as $x$ goes to $a$ is $L$,
\[
\lim_{x\to a}f(x)=L
\] 
if for every $\epsilon>0$ there is a $\delta > 0$ so that whenever $0
< |x-a| < \delta$, $|f(x)-L|<\epsilon$. If no such value of $L$ can be found, then we say that $f(x)$ \textbf{diverges} at $x=a$.
\end{definition}

\begin{example} Show that $\ds \lim_{x\to 2} x^2=4$.
 
We want to show that for any given $\epsilon>0$, we can find a
$\delta>0$ such that $\ds |x^2-4|<\epsilon$ whenever $0<|x-2|<\delta$.

Write $\ds |x^2-4|=|(x+2)(x-2)|$. Now when $|x-2|$ is small, part of
$|(x+2)(x-2)|$ is small, namely $(x-2)$. What about $(x+2)$? If $x$ is
close to 2, $(x+2)$ certainly can't be too big, but we need to somehow
be precise about it. Let's recall the ``game'' version of what is
going on here. You get to pick an $\epsilon$ and I have to pick a
$\delta$ that makes things work out. Presumably it is the really tiny
values of $\epsilon$ I need to worry about, but I have to be prepared
for anything, even an apparently ``bad'' move like $\epsilon=1000$.  I
expect that $\epsilon$ is going to be small, and that the
corresponding $\delta$ will be small, certainly less than 1.  If
$\delta\le 1$ then $|x+2|<5$ when $|x-2|<\delta$ (because if $x$ is
within 1 of 2, then $x$ is between 1 and 3 and $x+2$ is between 3 and
5). So then I'd be trying to show that
$|(x+2)(x-2)|<5|x-2|<\epsilon$. So now how can I pick $\delta$ so that
$|x-2|<\delta$ implies $5|x-2|<\epsilon$? This is easy: use
$\delta=\epsilon/5$, so $5|x-2|<5(\epsilon/5) = \epsilon$. But what if
the $\epsilon$ you choose is not small? If you choose $\epsilon=1000$,
should I pick $\delta=200$? No, to keep things ``sane'' I will never
pick a $\delta$ bigger than 1. Here's the final ``game strategy:''
When you pick a value for $\epsilon$ I will pick $\delta=\epsilon/5$
or $\delta=1$, whichever is smaller. Now when $|x-2|<\delta$, I know
both that $|x+2|<5$ and that $|x-2|<\epsilon/5$. Thus
$|(x+2)(x-2)|<5(\epsilon/5) = \epsilon$.

This has been a long discussion, but most of it was explanation and
scratch work. If this were written down as a proof, it would be quite
short, like this:

Proof that $\ds \lim_{x\to 2}x^2=4$. Given any $\epsilon$, pick
$\delta=\epsilon/5$ or $\delta=1$, whichever is smaller. Then when
$|x-2|<\delta$, $|x+2|<5$ and
$|x-2|<\epsilon/5$. Hence $\ds |x^2-4|=|(x+2)(x-2)|<5(\epsilon/5) =
\epsilon$. 
\end{example}







\begin{exercises}

Compute the limits. If a limit does not exist, explain why.

\twocol

\begin{exercise} $\ds \lim_{x\to 3}{x^2+x-12\over x-3}$
\begin{answer} 7
\end{answer}\end{exercise}

\begin{exercise} $\ds \lim_{x\to 1}{x^2+x-12\over x-3}$
\begin{answer} 5
\end{answer}\end{exercise}

\begin{exercise} $\ds \lim_{x\to -4}{x^2+x-12\over x-3}$
\begin{answer} 0
\end{answer}\end{exercise}

\begin{exercise} $\ds \lim_{x\to 2} {x^2+x-12\over x-2}$
\begin{answer} undefined
\end{answer}\end{exercise}

\begin{exercise} $\ds \lim_{x\to 1} {\sqrt{x+8}-3\over x-1}$
\begin{answer} $1/6$
\end{answer}\end{exercise}

\begin{exercise} $\ds \lim_{x\to 0^+} \sqrt{{1\over x}+2} - \sqrt{1\over x}$.
\begin{answer} 0
\end{answer}\end{exercise}

\begin{exercise} $\ds\lim _{x\to 2} 3$
\begin{answer} 3
\end{answer}\end{exercise}

\begin{exercise} $\ds\lim _{x\to 4 } 3x^3 - 5x $
\begin{answer} 172
\end{answer}\end{exercise}

\begin{exercise} $\ds \lim _{x\to 0 } {4x - 5x^2\over x-1}$
\begin{answer} 0
\end{answer}\end{exercise}

\begin{exercise} $\ds\lim _{x\to 1 } {x^2 -1 \over x-1 }$
\begin{answer} 2
\end{answer}\end{exercise}

\begin{exercise} $\ds\lim _{x\to 0^ + } {\sqrt{2-x^2 }\over x}$
\begin{answer} does not exist
\end{answer}\end{exercise}

\begin{exercise} $\ds\lim _{x\to 0^ + } {\sqrt{2-x^2}\over x+1}$
\begin{answer} $\ds \sqrt2$
\end{answer}\end{exercise}

\begin{exercise} $\ds\lim _{x\to a } {x^3 -a^3\over x-a}$
\begin{answer} $\ds 3a^2$
\end{answer}\end{exercise}

\begin{exercise} $\ds\lim _{x\to 2 } (x^2 +4)^3$
\begin{answer} 512
\end{answer}\end{exercise}

\begin{exercise} $\ds\lim _{x\to 1 } \begin{cases}
x-5 & x\neq 1, \\
7 & x=1. \end{cases}$
\begin{answer} $-4$
\end{answer}\end{exercise}

\endtwocol

\msk
\begin{exercise} $\ds\lim _{x\to 0 } x\sin \left( {1\over x}\right)$
(Hint: Use the fact that $|\sin a |< 1 $ for any real number $a$. You
should probably use the definition of a limit here.)
\begin{answer} $0$
\end{answer}\end{exercise}

\begin{exercise} Give an $\epsilon$--$\delta$ proof, similar to
example~\xrefn{exam:epsilon-delta proof},
of the fact that 
$\ds \lim_{x\to 4} (2x-5) = 3$. 
\end{exercise}

\begin{exercise} Evaluate the expressions by reference to this graph:\hfill\break
BADBAD

% BADBAD
%\hbox{\epsfxsize8cm\epsfbox{limit-exercise-graph.eps}}\hfill\break

% \halign{&\indent#\hfill \\
%  (a) $\ds \lim_{x\to 4} f(x)$ & 
%  (b) $\ds \lim_{x\to -3} f(x)$ & 
%  (c) $\ds \lim_{x\to 0} f(x)$  \\
%  (d) $\ds \lim_{x\to 0^-} f(x)$ & 
%  (e) $\ds \lim_{x\to 0^+} f(x)$ & 
%  (f) $\ds f(-2)$  \\
%  (g) $\ds \lim_{x\to 2^-} f(x)$ & 
%  (h) $\ds \lim_{x\to -2^-} f(x)$ & 
%  (i) $\ds \lim_{x\to 0} f(x+1)$  \\
%  (j) $\ds f(0)$ & 
%  (k) $\ds \lim_{x\to 1^-} f(x-4)$ & 
%  (l) $\ds \lim_{x\to 0^+} f(x-2)$  \\}
% \begin{answer} (a) $8$, (b) $6$, (c) dne, (d) $-2$, (e) $-1$, (f) $8$,
%  (g) $7$, (h) $6$, (i) $3$, (j) $-3/2$, (k) $6$, (l) $2$
% \end{answer}
\end{exercise}

\begin{exercise} Use a calculator to estimate $\ds\lim_{x\to 0}
{\sin x\over x}$.
\end{exercise}

\begin{exercise} Use a calculator to estimate $\ds\lim_{x\to 0}
{\tan(3x)\over\tan(5x)}$.
\end{exercise}

\end{exercises}
