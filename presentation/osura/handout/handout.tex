\documentclass[12pt]{amsart}
\usepackage{hyperref}
\newcommand{\mooculus}{\textsf{\textbf{mooc}ulus}}
\usepackage{nopageno}
\usepackage{geometry}
\geometry{left=1.5in,right=1.5in,top=1in,bottom=0.8in}

\title{What is Mooculus?}
\author{OSURA Dinner Series Presentation}

\begin{document}

\maketitle
\pagestyle{empty}
\thispagestyle{empty} 

On January~7, 2013, Ohio State launched its first massive open online
course (MOOC); the course is called ``Calculus One'' and is designed
to cover the same content as our local, in-person sections of
Math~1151.  Our online calculus course is available on Coursera at
\begin{center}
\texttt{https://www.coursera.org/course/calc1}
\end{center}
but since Coursera's platform lacks, for instance, randomly generated
math problems, we built our own MOOC platform here at Ohio State---we
call it \mooculus.  We encourage you to explore the platform we've
built by going to
\begin{center}
\texttt{https://mooculus.osu.edu/}
\end{center}

\subsection*{How many people are taking the course?}

The course has been offered twice: there were 47,227 students enrolled
in the first run, and there are 37,475 enrolled now in its second run.

%There are 32,890 students enrolled; a popular concern about MOOCs is
%the high attrition rate, but we have had 11,133 students engage with
%the course during Week~4, which is quite a healthy showing.  People
%are spending upwards of ten hours a week on our content.

\subsection*{Who are these students?}

It is easier to say who these students are not: they aren't
traditional 18--22 year old college students.  Many of our MOOC
students already have advanced degrees.  A five year old took Calculus
One.  An 11-year-old bested her father on the final exam.  For others,
they took calculus decades ago (even $\geq 50$ years ago) and want a
refresher.  Some are in the hospital.  Some are soldiers, currently
deployed.  Our student population is remarkably international and
incredibly diverse.

\subsection*{How has the course been received?}

Remarkably well; posts on the forum include ``the teaching style is
the best I've seen on Coursera so far'' and ``one of the best
lecturers I've ever seen anywhere, live or online'' and ``this is the
first example I have seen in either EdX or Coursera of using the
medium really well.''  Thousands of people from all over the world are
successfully learning calculus with OSU's help.

\subsection*{Can students get credit for Calculus One?}

Not yet.  Right now, students who complete our MOOC will receive a
certificate signed by the instructor.  In the longer term, we hope to
be able to offer a ``signature track'' version of Calculus One.  With
the signature track, a student's identity is verified, and the
certificate they earn will be verifiable and shareable, making them
more valuable as part of the student's job or school application.

\pagebreak

\subsection*{How does a student engage with our course?}

For the student, \mooculus\ provides new content each week, in the form of
\begin{itemize}
\item another chapter to our free, open-source calculus textbook;
\item a dozen short lecture videos, between 2 and 15 minutes long; and
\item interactive randomly-generated exercises that emphasize both conceptual and computational learning.
\end{itemize}
Since one learns math by doing math, the interactive exercises are
especially important.  Unlike other MOOC platforms, we use
a Hidden Markov Model to estimate student learning, based on how the
student engages with our website.  Once \mooculus\ believes the
student has mastered a particular problem, the
student is provided with a new challenge.

\subsection*{Why build our own platform?}

By building \mooculus, we provide a better student experience than
other MOOC platforms.  For instance, a common complaint with other
systems that don't rely on randomly generated exercises is that a
student eventually ``runs out'' of new exercises to try.  In contrast,
\mooculus\ never runs out of problems.  Additionally, now that we've
built \mooculus, we can use the same backend for other courses, such
as the English department's WexMOOC.

\subsection*{MOOCs help high school teachers ``flip'' the classroom}

A ``flipped classroom'' is a classroom structured so that students
learn new content online through video lectures---watched in their
dorm room or at home, rather than during the assigned lecture
time---and what used to be homework is instead done during the scheduled lecture time. The upside is
that this provides additional time for the teacher to interact with
students.  The downside is that teachers then must prepare video
lectures, but shooting video is expensive.

The content on \mooculus\ can help.  Currently we have six high school
teachers planning to use \mooculus\ in conjunction with their AP
Calculus classes.

\subsection*{MOOCs enhance the in-class experience at OSU}

While some believe that MOOCs are a replacement for in-class experiences, we do
not think this needs to be the case. Like Elizabeth Miller's project
\textit{Flipped and Flexible}, we see MOOCs as a tool for the
instructor to use their abilities to improve education across the
board. It should not be the case that the instructor is ``bound'' to
the textbook, instead the course materials should be dynamic enough
that the instructor can choose which online materials are to be used
and can create additional interactive materials without much programming experience.

\subsection*{The near-term future for MOOCs at OSU}

We will be offering more courses, like Calculus Two and Complex
Analysis, as well as offering Calculus One again.

We will be building a new platform: the Division of Undergraduate
Education at the NSF has funded a Transforming Undergraduate Education
in STEM (TUES) Type 1 award for our proposal DUE--1245433
(``Interactive Textbook'').  Our new platform will make it easier for
other instructors to build engaging online courses.


\end{document}

%%% Local Variables: 
%%% mode: latex
%%% TeX-master: t
%%% End: 
