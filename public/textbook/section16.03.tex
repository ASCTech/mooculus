\section{The Fundamental Theorem of Line Integrals}{}{}

One way to write the Fundamental Theorem of Calculus
(\xrefn{thm:fundamental_theorem_I}) is:
$$\int_a^b f'(x)\,dx = f(b)-f(a).$$
That is, to compute the integral of a derivative $f'$ 
we need only compute the values of $f$ at the endpoints. Something
similar is true for line integrals of a certain form.

\begin{theorem} (Fundamental Theorem of Line Integrals) Suppose a curve $C$ is
given by the vector function ${\bf r}(t)$, with ${\bf a}={\bf r}(a)$
and ${\bf b}={\bf r}(b)$. Then
$$\int_C \nabla f\cdot d{\bf r} = f({\bf b})-f({\bf a}),$$
provided that $\bf r$ is sufficiently nice.
\begin{proof}

We write ${\bf r}=\langle x(t),y(t),z(t)\rangle$, so 
that ${\bf r}'=\langle x'(t),y'(t),z'(t)\rangle$. Also,
we know that $\nabla f=\langle f_x,f_y,f_z\rangle$. Then
$$\int_C \nabla f\cdot d{\bf r} = 
\int_a^b \langle f_x,f_y,f_z\rangle\cdot\langle
x'(t),y'(t),z'(t)\rangle\,dt=
\int_a^b f_x x'+f_y y'+f_z z' \,dt.$$
By the chain rule (see section~\xrefn{sec:multivariable chain rules})
$f_x x'+f_y y'+f_z z'=df/dt$, where $f$ in this context means
$f(x(t),y(t),z(t))$, a function of $t$. In other words, all we have is
$$\int_a^b f'(t)\,dt=f(b)-f(a).$$
In this context,
$f(a)=f(x(a),y(a),z(a))$.
Since ${\bf a}={\bf r}(a)=\langle x(a),y(a),z(a)\rangle$, we can
write $f(a)=f({\bf a})$---this is a bit of a cheat, since we are
simultaneously using $f$ to mean $f(t)$ and $f(x,y,z)$, and since
$f(x(a),y(a),z(a))$ is not technically the same as 
$f(\langle x(a),y(a),z(a)\rangle)$,
but the
concepts are clear and the different uses are compatible. Doing the
same for $b$, we get
$$\int_C \nabla f\cdot d{\bf r} = \int_a^b f'(t)\,dt=f(b)-f(a)=f({\bf
  b})-f({\bf a}).$$ 
\vglue-20pt\end{proof}

This theorem, like the Fundamental Theorem of Calculus, says roughly
that if we integrate a ``derivative-like function'' ($f'$ or $\nabla
f$) the result depends only on the values of the original function ($f$)
at the endpoints.

If a vector field $\bf F$ is the gradient of a function, ${\bf
  F}=\nabla f$, we say that $\bf F$ is a {\dfont conservative
vector field\index{conservative vector field}}. If $\bf F$ is a
conservative force field, then the integral for work,
$\int_C {\bf F}\cdot d{\bf r}$, is in the form required by the
Fundamental Theorem of Line Integrals. This means that in a
conservative force field, the amount of work required to move an
object from point $\bf a$ to point $\bf b$ depends only on those
points, not on the path taken between them.

\begin{example}
An object moves in the force field 
$${\bf F}=
\left
\langle {-x\over (x^2+y^2+z^2)^{3/2}},{-y\over (x^2+y^2+z^2)^{3/2}},{-z\over
  (x^2+y^2+z^2)^{3/2}}\right\rangle,$$
along the curve ${\bf r}=\langle 1+t,t^3,t\cos(\pi t)\rangle$ as $t$
ranges from 0 to 1. Find the work done by the force on the object.

The straightforward way to do this involves substituting the
components of ${\bf r}$ into $\bf F$, forming the dot product ${\bf
  F}\cdot{\bf r}'$, and then trying to compute the integral, but this
integral is extraordinarily messy, perhaps impossible to compute. But
since ${\bf F}=\nabla (1/\sqrt{x^2+y^2+z^2})$ we need only substitute:
$$\int_C {\bf F}\cdot d{\bf r}=
\left.{1\over \sqrt{x^2+y^2+z^2}}\right|_{(1,0,0)}^{(2,1,-1)}=
{1\over\sqrt6}-1.
$$
\end{example}

Another immediate consequence of the Fundamental Theorem involves
{\dfont closed paths\index{closed paths}}. A path $C$ is closed if it
forms a loop, so that traveling over the $C$ curve brings you back to
the starting point. If $C$ is a closed path, we can integrate around
it starting at any point $\bf a$; since the starting and ending points are the
same,
$$\int_C \nabla f\cdot d{\bf r}=f({\bf a})-f({\bf a})=0.$$
For example, in a gravitational field (an inverse square law field)
the amount of work required to move an object around a closed path is
zero. Of course, it's only the {\it net\/} amount of work that is
zero. It may well take a great deal of work to get from point $\bf a$
to point $\bf b$, but then the return trip will ``produce'' work. For
example, it takes work to pump water from a lower to a higher elevation,
but if you then let gravity pull the water back down, you can recover
work by running a water wheel or generator. (In the real world you
won't recover all the work because of various losses along the way.)

To make use of the Fundamental Theorem of Line Integrals, we need to
be able to spot conservative vector fields $\bf F$ and to compute 
$f$ so that ${\bf F}=\nabla f$. Suppose that ${\bf F}=\langle
P,Q\rangle = \nabla f$. Then $P=f_x$ and $Q=f_y$, and provided that
$f$ is sufficiently nice, we know from Clairaut's Theorem
(\xrefn{thm:clairaut}) that $P_y=f_{xy}=f_{yx}=Q_x$. If we compute
$P_y$ and $Q_x$ and find that they are not equal, then $\bf F$ is not
conservative. If $P_y=Q_x$, then, again provided that $\bf F$ is
sufficiently nice, we can be assured that $\bf F$ is conservative.
Ultimately, what's important is that we be able to find $f$; as this
amounts to finding anti-derivatives, we may not always succeed.

\begin{example} Find an $f$ so that $\langle 3+2xy,x^2-3y^2\rangle = \nabla f$.

First, note that 
$${\partial\over\partial y}(3+2xy)=2x\qquad\hbox{and}\qquad
{\partial\over\partial x}(x^2-3y^2)=2x,$$
so the desired $f$ does exist. This means that $f_x=3+2xy$, so that
$f=3x+x^2y+g(y)$; the first two terms are needed to get $3+2xy$, and
the $g(y)$ could be any function of $y$, as it would disappear upon
taking a derivative with respect to $x$. Likewise, since
$f_y=x^2-3y^2$, $f=x^2y-y^3+h(x)$. The question now becomes, is it
possible to find $g(y)$ and $h(x)$ so that
$$3x+x^2y+g(y)=x^2y-y^3+h(x),$$
and of course the answer is yes: $g(y)=-y^3$, $h(x)=3x$. Thus,
$f=3x+x^2y-y^3$.
\end{example}

\pagerdef{page:test for conservative vector field}
We can test a vector field ${\bf F}=\v{P,Q,R}$ in a similar
way. Suppose that $\v{P,Q,R}=\v{f_x,f_y,f_z}$. If we temporarily hold
$z$ constant, then then $f(x,y,z)$ is a function of $x$ and $y$, and
by Clairaut's Theorem $P_y=f_{xy}=f_{yx}=Q_x$.
Likewise, holding $y$ constant implies $P_z=f_{xz}=f_{zx}=R_x$, and
with $x$ constant we get $Q_z=f_{yz}=f_{zy}=R_y$. Conversely, if we
find that $P_y=Q_x$, $P_z=R_x$, and $P_z=R_x$ then $\bf F$ is
conservative.

\begin{exercises}


\begin{exercise} Find an $f$ so that $\nabla f=\langle 2x+y^2,2y+x^2\rangle$, or
explain why there is no such $f$.
\begin{answer} no $f$
\end{answer}\end{exercise}

\begin{exercise} Find an $f$ so that $\nabla f=\langle x^3,-y^4\rangle$, 
or explain why there is no such $f$.
\begin{answer} $x^4/4-y^5/5$
\end{answer}\end{exercise}

\begin{exercise} Find an $f$ so that $\nabla f=\langle xe^y,ye^x \rangle$, 
or explain why there is no such $f$.
\begin{answer}  no $f$
\end{answer}\end{exercise}

\begin{exercise} Find an $f$ so that $\nabla f=\langle y\cos x,y\sin x \rangle$, 
or explain why there is no such $f$.
\begin{answer}  no $f$
\end{answer}\end{exercise}

\begin{exercise} Find an $f$ so that $\nabla f=\langle y\cos x,\sin x\rangle$, 
or explain why there is no such $f$.
\begin{answer} $y\sin x$
\end{answer}\end{exercise}

\begin{exercise} Find an $f$ so that $\nabla f=\langle x^2y^3,xy^4\rangle$, 
or explain why there is no such $f$.
\begin{answer} no $f$
\end{answer}\end{exercise}

\begin{exercise} Find an $f$ so that $\nabla f=\langle yz,xz,xy\rangle$, 
or explain why there is no such $f$.
\begin{answer} $xyz$
\end{answer}\end{exercise}

% Albert
\begin{exercise} Evaluate $\ds\int_C (10x^4 - 2xy^3)\,dx - 3x^2y^2\,dy$ where $C$ is
the part of the curve $x^5-5x^2y^2-7x^2=0$ from $(0,0)$ to
$(3,2)$.
\begin{answer} 414
\end{answer}\end{exercise}

%/Albert

\begin{exercise} Let 
${\bf F}=
\langle yz,xz,xy\rangle$.
Find the work done by this force field on an object that moves from 
$(1,0,2)$ to $(1,2,3)$.
\begin{answer} $6$
\end{answer}\end{exercise}

\begin{exercise} Let 
${\bf F}=
\langle e^y,xe^y+\sin z,y\cos z\rangle$.
Find the work done by this force field on an object that moves from 
$(0,0,0)$ to $(1,-1,3)$.
\begin{answer} $1/e-\sin3$
\end{answer}\end{exercise}

\begin{exercise} Let 
$${\bf F}=
\left
\langle {-x\over (x^2+y^2+z^2)^{3/2}},{-y\over (x^2+y^2+z^2)^{3/2}},{-z\over
  (x^2+y^2+z^2)^{3/2}}\right\rangle.$$
Find the work done by this force field on an object that moves from 
$(1,1,1)$ to $(4,5,6)$.
\begin{answer} $1/\sqrt{77}-1/\sqrt3$
\end{answer}\end{exercise}

\end{exercises}

