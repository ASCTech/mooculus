\chapter{Integrals}

\section{Definite Integrals Compute Signed Area}

Definite integrals, often simply called integrals, compute signed area. 

\begin{definition}\index{integral}\index{definite integral}
The \textbf{definite integral}
\[
\int_a^b f(x) \d x
\]
computes the signed area in the region $[a,b]$ between $f(x)$ and the
$x$-axis. If the region is above the $x$-axis, then the area has
positive sign. If the region is below the $x$-axis, then the area has
negative sign.
\end{definition}

\begin{example}
Compute
\[
\int_0^3 x \d x.
\]
\end{example}
\begin{marginfigure}
\begin{tikzpicture}
  \begin{axis}[
      xmin=0, xmax=3,ymin=0,ymax=3,domain=0:3,
      axis lines =center, xlabel=$x$, ylabel=$y$,
      every axis y label/.style={at=(current axis.above origin),anchor=south},
      every axis x label/.style={at=(current axis.right of origin),anchor=west},
      axis on top,
    ] 
    \addplot [draw=none, fill=fillp] {x} \closedcycle;
    \addplot [penColor,very thick] {x};
  \end{axis}
\end{tikzpicture}
\caption{The integral $\int_0^3 x \d x$ measures the shaded area.}
\label{figure:intfirst}
\end{marginfigure}

\begin{solution}
The definite integral $\int_0^3 x \d x$ measures signed area of the
shaded region shown in figure~\ref{figure:intfirst}. Since this region
is a triangle, we can use the formula for the area of the triangle to
compute
\[
\int_0^2 x \d x = \frac{1}{2} 3\cdot 3 = 9/2. 
\]
\end{solution}

When working with signed area, positive and negative area cancel each
other out.

\begin{example} Compute
\[
\int_{-1}^3 \lfloor x \rfloor \d x.
\]
\end{example}

\begin{marginfigure}[0in]
\begin{tikzpicture}
  \begin{axis}[
      domain=-1:3,
      axis lines =middle, xlabel=$x$, ylabel=$y$,
      every axis y label/.style={at=(current axis.above origin),anchor=south},
      every axis x label/.style={at=(current axis.right of origin),anchor=west},
      clip=false,
      axis on top,
    ]
    \addplot [draw=none, fill=filln, domain=(-1:0)] {-1} \closedcycle;
    \addplot [draw=none, fill=fillp, domain=(0:1)] {0} \closedcycle;
    \addplot [draw=none, fill=fillp, domain=(1:2)] {1} \closedcycle;
    \addplot [draw=none, fill=fillp, domain=(2:3)] {2} \closedcycle;
    \addplot [very thick, penColor, domain=(-1:0)] {-1};
    \addplot [very thick, penColor, domain=(0:1)] {0};
    \addplot [very thick, penColor, domain=(1:2)] {1};
    \addplot [very thick, penColor, domain=(2:3)] {2};
    \addplot[color=penColor,fill=penColor,only marks,mark=*] coordinates{(-1,-1)};  %% closed hole          
    \addplot[color=penColor,fill=penColor,only marks,mark=*] coordinates{(0,0)};  %% closed hole          
    \addplot[color=penColor,fill=penColor,only marks,mark=*] coordinates{(1,1)};  %% closed hole          
    \addplot[color=penColor,fill=penColor,only marks,mark=*] coordinates{(2,2)};  %% closed hole  
    \addplot[color=penColor,fill=penColor,only marks,mark=*] coordinates{(3,3)};  %% closed hole                  
    \addplot[color=penColor,fill=background,only marks,mark=*] coordinates{(0,-1)};  %% open hole
    \addplot[color=penColor,fill=background,only marks,mark=*] coordinates{(1,0)};  %% open hole
    \addplot[color=penColor,fill=background,only marks,mark=*] coordinates{(2,1)};  %% open hole
    \addplot[color=penColor,fill=background,only marks,mark=*] coordinates{(3,2)};  %% open hole
  \end{axis}
\end{tikzpicture}
\caption{The integral $\int_{-1}^3 \lfloor x\rfloor \d x$ measures the
  shaded area. Area above the $x$-axis has positive sign and the area below the $x$-axis has negative sign.}
\label{plot:int-greatist-integer}
\end{marginfigure}

\begin{solution}
The definite integral $\int_{-1}^3 \lfloor x\rfloor \d x$ measures
signed area of the shaded region shown in
figure~\ref{plot:int-greatist-integer}. We see that 
\[
\int_{-1}^3 \lfloor x\rfloor \d x = 
\int_{-1}^0 \lfloor x\rfloor \d x + \int_{0}^1 \lfloor x\rfloor \d x + \int_{1}^2 \lfloor x\rfloor \d x + \int_{2}^3 \lfloor x\rfloor \d x.
\]
So computing each of these areas separately
\begin{align*}
\int_{-1}^3 \lfloor x\rfloor \d x &= -1 + 0 + 1+2 \\
&= 2.
\end{align*}
\end{solution}

Our previous examples hopefully gives us enough insight that this next
theorem is unsurprising.

\begin{mainTheorem}[Properties of Definite Integrals]
\begin{enumerate}
\item $\int_a^b k \d x= kb-ka$, where $k$ is a constant.
\item $\int_a^b \left( f(x) + g(x) \right) \d x = \int_a^b f(x) \d x + \int_a^b
  g(x) \d x$.
\item $\int_a^b k \cdot f(x) \d x = k \int_a^b f(x) \d x$.
\end{enumerate}
\end{mainTheorem}
Each of these properties follows from the notion that definite
integrals compute signed area.

\subsection*{Accumulation Functions}

While the definite integral computes a signed area, which is a fixed
number, there is a way to turn it into a function.
\begin{definition}
The given a function $f(x)$, an \textbf{accumulation function} for
$f(x)$ is given by
\[
F(x) = \int_a^x f(t) \d t.
\]
\end{definition}

One thing that you might note is that an accumulation function seems
to have two variables $x$ and $t$. Let's see if we can explain
this. Consider the following plot:

\begin{tikzpicture}
	\begin{axis}[
            domain=0:6, ymax=2.2,xmax=6,
            axis lines =left, xlabel=$t$, ylabel=$y$,
            every axis y label/.style={at=(current axis.above origin),anchor=south},
            every axis x label/.style={at=(current axis.right of origin),anchor=west},
            xtick={1,5}, ytick={.203,1.679},
            xticklabels={$a$,$x$}, yticklabels={$f(a)$,$f(x)$},
            axis on top,
          ]
          \addplot [draw=none,fill=fillp,domain=(1:5)] {sin(deg((x - 4)/2)) + 1.2} \closedcycle;
          \addplot [very thick,penColor, smooth,domain=(0:6)] {sin(deg((x - 4)/2)) + 1.2};
          \addplot [textColor,dashed] plot coordinates {(0,1.679) (5,1.679)};
          \addplot [textColor,dashed] plot coordinates {(0,.203) (1,.203)};
          \addplot [textColor,dashed] plot coordinates {(5,0) (5,1.679)};
          \addplot [textColor,dashed] plot coordinates {(1,0) (1,.203)};
          \addplot [color=penColor,fill=penColor,only marks,mark=*] coordinates{(1,.203)};  %% closed hole         
          \addplot [color=penColor,fill=penColor,only marks,mark=*] coordinates{(5,1.679)};  %% closed hole       
          \node at (axis cs:3.4,.3) [textColor] {$F(x) = \int_a^x f(t) \d t$};
          \node at (axis cs:3.4,1.1) [penColor] {$f(t)$};
        \end{axis}
\end{tikzpicture}


An accumulation function $F(x)$ is measuring the signed area in the
region $[a,x]$ between $f(t)$ and the $t$-axis. Hence $t$ is playing
the role of a ``place-holder'' and represents numbers where we are
evaluating $f(t)$. On the other hand, $x$ is the specific number that
we are using to bound the region that will determine the area between
$f(t)$ and the $t$-axis.


\begin{example} 
Consider the following accumulation function for $f(x) = x^3$.
\[
F(x) = \int_{-1}^x t^3 \d t.
\]
Considering the interval $[-1,1]$, where is $F(x)$ increasing? Where
is $F(x)$ decreasing? When does $F(x)$ have a local extrema?
\end{example}

\begin{marginfigure}
\begin{tikzpicture}
  \begin{axis}[
      xmin=-1, xmax=1,ymin=-1,ymax=1,domain=-1:1,
      axis lines =center, xlabel=$t$, ylabel=$y$,
      every axis y label/.style={at=(current axis.above origin),anchor=south},
      every axis x label/.style={at=(current axis.right of origin),anchor=west},
      xtick={-1,.8}, 
      xticklabels={$-1$,$x$}, 
      axis on top,
    ] 
    \addplot [draw=none, fill=fillp,domain=0:.8] {x^3} \closedcycle;
    \addplot [draw=none, fill=filln,domain=-1:0] {x^3} \closedcycle;
    \addplot [penColor,very thick] {x^3};
  \end{axis}
\end{tikzpicture}
\caption{The integral $\int_{-1}^x t^3 \d t$ measures the shaded area.}
\label{figure:accumulationeg}
\end{marginfigure}

\begin{solution}
We can see a plot of $f(x)$ along with the signed area measured by the
accumulation function in Figure~\ref{figure:accumulationeg}. The
accumulation function starts off at zero, and then is decreasing as it
accumulates negatively signed area. However when $x>0$, $F(x)$ starts
to accumulate positively signed area, and hence is increasing. Thus
$F(x)$ is increasing on $[0,1]$, decreasing on $[-1,0]$ and hence has
a local minimum at $(0,0)$.
\end{solution}

Working with the accumulation function leads us to a question: What is  
\[
\int_a^x f(x) \d x
\]
when $x< a$? We'll answer this with a general convention
\[
\int_a^b f(x) \d x = -\int_b^a f(x) \d x. 
\]
With this in mind, let's extend the example we did above. 


\begin{example} 
Consider the following accumulation function for $f(x) = x^3$.
\[
F(x) = \int_{-1}^x t^3 \d t.
\]
Where is $F(x)$ increasing? Where is $F(x)$ decreasing? When does
$F(x)$ have a local extrema?
\end{example}

\begin{marginfigure}
\begin{tikzpicture}
  \begin{axis}[
      xmin=-5, xmax=5,ymin=-25,ymax=25,domain=-3:3,
      axis lines =center, xlabel=$t$, ylabel=$y$,
      every axis y label/.style={at=(current axis.above origin),anchor=south},
      every axis x label/.style={at=(current axis.right of origin),anchor=west},
      xtick={-1,.8}, 
      xticklabels={$-1$,$x$}, 
      axis on top,
    ] 
    \addplot [draw=none, fill=fillp,domain=0:.8] {x^3} \closedcycle;
    \addplot [draw=none, fill=filln,domain=-1:0] {x^3} \closedcycle;
    \addplot [penColor,very thick] {x^3};
  \end{axis}
\end{tikzpicture}
\caption{The integral $\int_{-1}^x t^3 \d t$ measures the shaded area.}
\label{figure:accumulationeg}
\end{marginfigure}

\begin{solution}
We can see a plot of $f(x)$ along with the signed area measured by the
accumulation function in Figure~\ref{figure:accumulationegreal}. The
accumulation function starts off at zero, and then is decreasing as it
accumulates negatively signed area. However when $x>0$, $F(x)$ starts
to accumulate positively signed area, and hence is increasing. Thus
$F(x)$ is increasing on $[0,1]$, decreasing on $[-1,0]$ and hence has
a local minimum at $(0,0)$.
\end{solution}



\begin{exercises}

\end{exercises}




\section{Riemann Sums}

\begin{exercises}

\end{exercises}



\section{Euler's Method}

