\chapter{Functions}

\section{For Each Input, Exactly One Output}

Life is complex. Part of this complexity stemms from the fact that
there are many relations between seemingly independent events. Armed
with mathematics we seek to understand the world, and hence we need
tools for talking about these relations.

A \textit{function} is a relation between sets of objects that can be
thought of as a ``mathematical machine.'' This means for each input,
there is exactly one output. Moreover, whenever we talk about
functions, we should try to explicitly state what type of things the
inputs are and what type of things the outputs are.

In calculus, functions often define a relation from (a subset of) the
real numbers to (a subset of) the real numbers.


\marginnote{While the name of the function is technically ``$f$,'' we
  will abuse notation and call the function ``$f(x)$'' to remind the
  reader that it is a function.}
\begin{example}
Consider the function $f$ that maps from the real numbers to the real
numbers by taking a number and mapping it to its cube:
\begin{align*}
1 &\mapsto 1\\
2 &\mapsto 8\\
3 &\mapsto 27
\end{align*}
and so on. This function can be described by the formula $f(x)=x^3$ or
by the plot shown in Figure~\ref{plot:fxn x^3}.
\end{example}

\begin{warning}
A function is a mapping (such that for each input, there is exactly one
output) between sets and should not be confused with either its
formula or its plot.
\begin{itemize}
\item A formula merely describes the mapping using algebra.
\item A plot merely describes the mapping using pictures. 
\end{itemize}
\end{warning}


\begin{marginfigure}[0in]
\begin{tikzpicture}
	\begin{axis}[
            domain=-2:2,
            axis lines =middle, xlabel=$x$, ylabel=$y$,
            every axis y label/.style={at=(current axis.above origin),anchor=south},
            every axis x label/.style={at=(current axis.right of origin),anchor=west},
          ]
	  \addplot [very thick, penColor, smooth] {x^3};
        \end{axis}
\end{tikzpicture}
\caption{A plot of $f(x)=x^3$. Here we can see that for each input---a
  value on the $x$-axis, there is exactly one output---a value on the
  $y$-axis.}
\label{plot:fxn x^3}
\end{marginfigure}



\begin{example}
Consider the function defined by the formula:
\[
f(x) = x^2
\]
see Figure~\ref{} for a plot of this function. Some might be confused
because here we have two inputs that give the same output. However,
this is not a problem. To be a function, we mererly need a that for
each input, there is exactly one output, and this is satisfied.
\end{example}






\section{Inverses}

\section{The Language of Functions}



$f(x) = x+1$ What is $f(f(f(f(1))))$



Which of the following represent the same thing?

$\sin^2 x$, $\sin(x)^2$  $(\sin x)^2$ $\sin(x^2)$ $\sin x^2$ $\sin(x) \sin(x)$ $(\sin x)(\sin x)$


$\arcsin(x)$ $\sin^{-1}(x)$ $\sin(x)^{-1}$ $\frac{1}{\sin(x)}$ $\sin(x^{-1})$  $(\sin x)^{-1}$


$f^{-1}(x)$ vs $f(x)^{-1}$
