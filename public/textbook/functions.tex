\chapter{Functions}

\section{For Each Input, Exactly One Output}

Life is complex. Part of this complexity stems from the fact that
there are many relations between seemingly independent events. Armed
with mathematics we seek to understand the world, and hence we need
tools for talking about these relations.

\marginnote{Something as simple as a dictionary could be thought of as
  a relation, as it connects \textit{words} to
  \textit{definitions}. However, a dictionary is not a function, as
  there are words with multiple definitions. On the other hand, if
  each word only had a single definition, then it would be a
  function.}

A \textit{function} is a relation between sets of objects that can be
thought of as a ``mathematical machine.'' This means for each input,
there is exactly one output. Moreover, whenever we talk about
functions, we should try to explicitly state what type of things the
inputs are and what type of things the outputs are.

In calculus, functions often define a relation from (a subset of) the
real numbers to (a subset of) the real numbers.


\marginnote[.5in]{While the name of the function is technically ``$f$,'' we
  will abuse notation and call the function ``$f(x)$'' to remind the
  reader that it is a function.}
\begin{example}
Consider the function $f$ that maps from the real numbers to the real
numbers by taking a number and mapping it to its cube:
\begin{align*}
1 &\mapsto 1\\
-2 &\mapsto -8\\
1.5 &\mapsto 3.375
\end{align*}
and so on. This function can be described by the formula $f(x)=x^3$ or
by the plot shown in Figure~\ref{plot:fxn x^3}.
\end{example}

\begin{warning}
A function is a mapping (such that for each input, there is exactly one
output) between sets and should not be confused with either its
formula or its plot.
\begin{itemize}
\item A formula merely describes the mapping using algebra.
\item A plot merely describes the mapping using pictures. 
\end{itemize}
\end{warning}


\begin{marginfigure}[0in]
\begin{tikzpicture}
	\begin{axis}[
            domain=-2:2,
            axis lines =middle, xlabel=$x$, ylabel=$y$,
            every axis y label/.style={at=(current axis.above origin),anchor=south},
            every axis x label/.style={at=(current axis.right of origin),anchor=west},
          ]
	  \addplot [very thick, penColor, smooth] {x^3};
        \end{axis}
\end{tikzpicture}
\caption{A plot of $f(x)=x^3$. Here we can see that for each input---a
  value on the $x$-axis, there is exactly one output---a value on the
  $y$-axis.}
\label{plot:fxn x^3}
\end{marginfigure}



\begin{example}
Consider the \textit{greatest integer function}, denoted by
\[
f(x) = \lfloor x \rfloor.
\]
This is the function that maps any real number $x$ to the the greatest
integer less than or equal to $x$. See Figure~\ref{plot:greatest-integer fxn} for a plot of
this function. Some might be confused because here we have multiple
inputs that give the same output. However, this is not a problem. To
be a function, we merely need a that for each input, there is exactly
one output, and this is satisfied.
\end{example}
\begin{marginfigure}[0in]
\begin{tikzpicture}
	\begin{axis}[
            domain=-2:4,
            axis lines =middle, xlabel=$x$, ylabel=$y$,
            every axis y label/.style={at=(current axis.above origin),anchor=south},
            every axis x label/.style={at=(current axis.right of origin),anchor=west},
            clip=false,
            %axis on top,
          ]
          \addplot [textColor, very thin, domain=(0:2.3)] {0}; % puts the axis back, axis on top clobbers our open holes
          \addplot [textColor, very thin] plot coordinates {(0,0) (0,2)}; % puts the axis back, axis on top clobbers our open holes
	  \addplot [very thick, penColor, domain=(-2:-1)] {-2};
          \addplot [very thick, penColor, domain=(-1:0)] {-1};
          \addplot [very thick, penColor, domain=(0:1)] {0};
          \addplot [very thick, penColor, domain=(1:2)] {1};
          \addplot [very thick, penColor, domain=(2:3)] {2};
          \addplot [very thick, penColor, domain=(3:4)] {3};
          \addplot[color=penColor,fill=penColor,only marks,mark=*] coordinates{(-2,-2)};  %% closed hole          
          \addplot[color=penColor,fill=penColor,only marks,mark=*] coordinates{(-1,-1)};  %% closed hole          
          \addplot[color=penColor,fill=penColor,only marks,mark=*] coordinates{(0,0)};  %% closed hole          
          \addplot[color=penColor,fill=penColor,only marks,mark=*] coordinates{(1,1)};  %% closed hole          
          \addplot[color=penColor,fill=penColor,only marks,mark=*] coordinates{(2,2)};  %% closed hole  
          \addplot[color=penColor,fill=penColor,only marks,mark=*] coordinates{(3,3)};  %% closed hole                  
          \addplot[color=penColor,fill=background,only marks,mark=*] coordinates{(-1,-2)};  %% open hole
          \addplot[color=penColor,fill=background,only marks,mark=*] coordinates{(0,-1)};  %% open hole
          \addplot[color=penColor,fill=background,only marks,mark=*] coordinates{(1,0)};  %% open hole
          \addplot[color=penColor,fill=background,only marks,mark=*] coordinates{(2,1)};  %% open hole
          \addplot[color=penColor,fill=background,only marks,mark=*] coordinates{(3,2)};  %% open hole
          \addplot[color=penColor,fill=background,only marks,mark=*] coordinates{(4,3)};  %% open hole
        \end{axis}
\end{tikzpicture}
\caption{A plot of $f(x)=\lfloor x\rfloor$. Here we can see that for each input---a
  value on the $x$-axis, there is exactly one output---a value on the
  $y$-axis.}
\label{plot:greatest-integer fxn}
\end{marginfigure}


Just to remind you, a function maps from one set to another. We call
the set a function is mapping from the \textbf{domain}\index{domain}
or \textit{source} and we call the set a function is mapping to the
\textbf{range}\index{range} or \textit{target}.  In our previous
examples the domain and range have both been the real numbers, denoted
by $\R$. In our next examples we show that this is not always the
case.


\begin{example}
Consider the function that maps non-negative real numbers to their positive square root. This function is denoted by 
\[
f(x) = \sqrt{x}.
\]
Note, since this is a function, and its range consists of the positive real numbers, we have that 
\[
\sqrt{x^2} = |x|.
\]
See Figure~\ref{plot:sqrt fxn} for a plot of this
function.
\end{example}

Finally, we will consider a function whos domain is all real numbers
except for a single point.

\begin{example}
Consider the function defined by 
\[
f(x) = \frac{x^2 - 3x + 2}{x-2}
\]
This function may seem innocent enough; however, it is undefined at
$x=2$. See Figure~\ref{plot:point undfed fxn} for a plot of this function.
\end{example}

\begin{marginfigure}[0in]
\begin{tikzpicture}
	\begin{axis}[
            xmin=-8,xmax=8,
            ymin=-5,ymax=5,
            domain=0:8,
            axis lines =middle, xlabel=$x$, ylabel=$y$,
            every axis y label/.style={at=(current axis.above origin),anchor=south},
            every axis x label/.style={at=(current axis.right of origin),anchor=west},
          ]
	  \addplot [very thick, penColor, smooth,samples=100] {sqrt(x)};
        \end{axis}
\end{tikzpicture}
\caption{A plot of $f(x)=\sqrt{x}$. Here we can see that for each
  input---a non-negative value on the $x$-axis, there is exactly one
  output---a positive value on the $y$-axis.}
\label{plot:sqrt fxn}
\end{marginfigure}


\begin{marginfigure}[0in]
\begin{tikzpicture}
	\begin{axis}[
            domain=-2:4,
            axis lines =middle, xlabel=$x$, ylabel=$y$,
            every axis y label/.style={at=(current axis.above origin),anchor=south},
            every axis x label/.style={at=(current axis.right of origin),anchor=west},
            xtick={-2,...,4},
            ytick={-3,...,3},
          ]
	  \addplot [very thick, penColor, smooth] {x-1};
          \addplot[color=penColor,fill=background,only marks,mark=*] coordinates{(2,1)};  %% open hole
        \end{axis}
\end{tikzpicture}
\caption{A plot of $f(x)=\protect\frac{x^2 - 3x + 2}{x-2}$. Here we
  can see that for each input---any value on the $x$-axis except for
  $x=2$, there is exactly one output---a value on the $y$-axis.}
\label{plot:point undfed fxn}
\end{marginfigure}


\begin{exercises}

\begin{exercise} 
\begin{answer}
\end{answer}
\end{exercise}

\end{exercises}




\section{Inverses}








\section{The Language of Functions}



$f(x) = x+1$ What is $f(f(f(f(1))))$



Which of the following represent the same thing?

$\sin^2 x$, $\sin(x)^2$  $(\sin x)^2$ $\sin(x^2)$ $\sin x^2$ $\sin(x) \sin(x)$ $(\sin x)(\sin x)$


$\arcsin(x)$ $\sin^{-1}(x)$ $\sin(x)^{-1}$ $\frac{1}{\sin(x)}$ $\sin(x^{-1})$  $(\sin x)^{-1}$


$f^{-1}(x)$ vs $f(x)^{-1}$
