\chapter{Antiderivatives}

\section{Basic Antiderivatives}

Computing derivatives is not too difficult. At this point, you should
be able to take the derivative of almost any function you can write
down. However, undoing derivatives is much harder. This process of
undoing a derivative is called taking an \textit{antiderivative}.

\begin{definition}\index{antiderivative}
A function $F(x)$ is called an \textbf{antiderivative} of $f(x)$ on an
interval if
\[
F'(x) = f(x)
\]
for all $x$ in the interval.
\end{definition}

We have special notion for the antiderivative:

\begin{definition}\index{antiderivative!notation}\index{indefinite integral}
The antiderivative is denoted by
\[
\int f(x) \d x = F(x)+C,
\]
where $dx$ identifies $x$ as the variable and $C$ is a constant
indicating that there a many possible antiderivatives, each varying by
the addition of a constant.  This is often called the
\textbf{indefinite integral}.
\end{definition}

Here are the basic antiderivatives. Note each of these examples comes
from knowing our.


\begin{fullwidth}
\begin{mainTheorem}[Basic Antiderivatives of Common Functions]\label{theorem:basicAnti} \hfil
\begin{multicols}{3}
\begin{itemize}
\item $\int k \d x= kx+C$.
\item $\int x^n \d x= \frac{x^{n+1}}{n+1}+C\qquad(n\ne-1)$.
\item $\int e^x \d x= e^x + C$.
\item $\int a^x \d x= \frac{a^x}{\ln(a)}+C$.
\item $\int \frac{1}{x} \d x= \ln|x|+C$.
\item $\int \cos(x) \d x = \sin(x) + C$.
\item $\int \sin(x) \d x = -\cos(x) + C$.  
\item $\int \tan(x) \d x = \ln|\cos(x)| + C$.  
\item $\int \sec^2(x) \d x = \tan(x) + C$. 
\item $\int \csc^2(x) \d x = -\cot(x) + C$.
\item $\int \sec(x)\tan(x) \d x = \sec(x) + C$.
\item $\int \csc(x)\cot(x) \d x = -\csc(x) + C$.
\item $\int \frac{1}{x^2+1}\d x = \arctan(x) + C$.
\item $\int \frac{1}{\sqrt{1-x^2}}\d x= \arcsin(x)+C$.
\end{itemize}
\end{multicols}
\end{mainTheorem}
\end{fullwidth}
It may seem that one could simply memorize these antiderivatives and
antidifferentiating would be as easy as differentiating. This is
\textbf{not} the case. The issue comes up when trying to combine these
functions.  When taking derivatives we have the \textit{product rule}
and the \textit{chain rule}. The analogues of these two rules are much
more difficult to deal with when taking antiderivatives. However, not
all is lost. We have the following analogue of the Sum Rule for
derivatives, Theorem~\ref{theorem:sumRule}.

\begin{mainTheorem}[The Sum Rule for Antiderivatives]\label{theorem:SRA}
Given two functions $f(x)$ and $g(x)$ where $k$ is a constant:
\begin{itemize}
\item $\int k f(x) \d x= kF(x) + C$.
\item $\int \left(f(x) + g(x)\right) \d x = F(x) + G(x) + C$.
\end{itemize}
\end{mainTheorem}

Let's put this rule and our knowledge of basic derivatives to work.

\begin{example}
Compute
\[
\int 3 x^7 \d x.
\]
\end{example}

\begin{solution}
By Theorem~\ref{theorem:basicAnti} and Theorem~\ref{theorem:SRA}, we
see that
\begin{align*}
\int 3 x^7 \d x &= 3 \int x^7 \d x\\
&= 3 \cdot \frac{x^8}{8}+C.
\end{align*}
\end{solution}

The sum rule for antiderivatives, Theorem~\ref{theorem:SRA}, allows us to integrate term-by-term. Let's see an example of this.

\begin{example}
Compute
\[
\int \left(x^4 + 5x^2 - \cos(x)\right) \d x.
\]
\end{example}

\begin{solution} 
Let's start by simplifying the problem using the sum rule for
antiderivatives, Theorem~\ref{theorem:SRA}.
\[
\int \left(x^4 + 5x^2 - \cos(x)\right) \d x = \int x^4 \d x + 5\int x^2 \d x - \int \cos(x) \d x.
\]
Now we may integrate term-by-term to find
\[
\int \left(x^4 + 5x^2 - \cos(x)\right) \d x = \frac{x^5}{4} + \frac{5x^3}{3}  - \sin(x)+C.
\]
\end{solution}


\begin{warning}
While the sum rule for antiderivatives allows us to integrate
term-by-term, we cannot integrate \textit{factor-by-factor}, meaning
that in general
\[
\int f(x)g(x) \d x \ne \int f(x) \d x\cdot \int g(x) \d x.
\]
\end{warning}








\subsection*{Tips for Guessing Antiderivatives}


Unfortunately, we cannot tell you how to compute every antiderivative.
We advise that the mathematician view antiderivatives as a sort of
\textit{puzzle}. Later we will learn a hand-full of techniques for
computing antiderivatives. However, a robust and simple way to compute
antiderivatives is guess-and-check.


\begin{guessingAntiderivatives}\hfil
\begin{enumerate}
\item Make a guess for the antiderivative.
\item Take the derivative of your guess.
\item Note how the above derivative is different from the function
  whose antiderivative you want to find.
\item Change your original guess by \textbf{multiplying} by constants
  or by \textbf{adding} in new functions.
\end{enumerate}
\end{guessingAntiderivatives}

\begin{template}\label{template:powerchain}
If the indefinite integral looks \emph{something} like
\[
\int \mathrm{stuff}' \cdot (\mathrm{stuff})^n \d x \qquad\text{then guess} \qquad \mathrm{stuff}^{n+1}
\]
where $n\ne -1$.
\end{template}

\begin{example} Compute
\[
\int \frac{x^3}{\sqrt{x^4 -6}} \d x.
\]
\end{example}

\begin{solution}
Start by rewriting the definite integral as
\[
\int x^3\left(x^4 -6\right)^{-1/2} \d x.
\]
Now start with a guess of 
\[
\int x^3\left(x^4 -6\right)^{-1/2} \d x \approx \left(x^4 -6\right)^{1/2}.
\]
Take the derivative of your guess to see if it is correct:
\[
\ddx  \left(x^4 -6\right)^{1/2} = (4/2)x^3\left(x^4 -6\right)^{-1/2}.
\]
We're off by a factor of $2/4$, so multiply our guess by this constant
to get the solution,
\[
\int \frac{x^3}{\sqrt{x^4 -6}} \d x = (2/4)\left(x^4 -6\right)^{1/2}+C.
\]
\end{solution}


\begin{template}\label{template:echain}
If the indefinite integral looks \emph{something} like
\[
\int \mathrm{junk}\cdot e^{\mathrm{stuff}} \d x \qquad\text{then
  guess}\qquad e^{\mathrm{stuff}} \text{ \textbf{or} }\mathrm{junk}
\cdot e^{\mathrm{stuff}}.
\]
\end{template}


\begin{example}
Compute
\[
\int xe^{x} \d x.
\]
\end{example}


\begin{solution}
We try to guess the antiderivative. Start with a guess of
\[
\int xe^x \d x \approx xe^x.
\]
Take the derivative of your guess to see if it is correct:
\[
\ddx xe^x = e^x + xe^x.
\]
Ah! So we need only subtract $e^x$ from our original guess.  We now
find
\[
\int xe^x \d x =xe^x - e^x + C.
\]
\end{solution}





\begin{template}\label{template:lnchain}
If the indefinite integral looks \emph{something} like
\[
\int \frac{\mathrm{stuff}'}{\mathrm{stuff}}\d x \qquad\text{then guess}\qquad\ln(\mathrm{stuff}).
\]
\end{template}

\begin{example}
Compute
\[
\int \frac{2x^2}{7x^3 + 3} \d x.
\]
\end{example}

\begin{solution} We'll start with a guess of
\[
\int \frac{2x^2}{7x^3 + 3} \d x \approx \ln(7x^3+3).
\]
Take the derivative of your guess to see if it is correct:
\[
\ddx \ln(7x^3+3) = \frac{21x^2}{7x^3 + 3}.
\]
We are only off by a factor of $2/21$, so we need to multiply our
original guess by this constant to get the solution,
\[
\int \frac{2x^2}{7x^3 + 3} \d x = (2/21)\ln(7x^3+3)+C.
\]
\end{solution}




\begin{template}\label{template:trigchain}
If the indefinite integral looks \emph{something} like
\[
\int \mathrm{junk}\cdot \sin(\mathrm{stuff}) \d x \qquad\text{then
  guess}\qquad \cos(\mathrm{stuff}) \text{ \textbf{or} }\mathrm{junk}
\cdot \cos(\mathrm{stuff}),
\]
likewise if you have 
\[
\int \mathrm{junk}\cdot \cos(\mathrm{stuff}) \d x \qquad\text{then
  guess}\qquad \sin(\mathrm{stuff}) \text{ \textbf{or} }\mathrm{junk}
\cdot \sin(\mathrm{stuff}),
\]
\end{template}



\begin{example}
Compute
\[
\int x^4\sin(3x^5+7) \d x.
\]
\end{example}


\begin{solution}
Here we simply try to guess the antiderivative. Start with a guess of
\[
\int x^4\sin(3x^5+7)\d x \approx \cos(3x^5+7).
\]
To see if your guess is correct, take the derivative of $\cos(3x)$,
\[
\ddx \cos(3x^5+7) = -15x^4\sin(3x^5+7).
\]
We are off by a factor of $-1/15$. Hence we should multiply our
original guess by this constant to find
\[
\int x^4\sin(3x^5+7) \d x = \frac{-\sin(3x)}{15} + C.
\]
\end{solution}





\subsection*{Final Thoughts}
Computing antiderivatives is a place where insight and rote
computation meet. We cannot teach you a method that will always
work. Moreover, merely \emph{understanding} the examples above will
probably not be enough for you to become proficient in computing
antiderivatives. You must practice, practice, practice!



\begin{exercises}
\noindent Compute the following antiderivatives.
\begin{multicols}{2}
\begin{exercise}
$\int 5\d x$
\begin{answer}
\end{answer}
\end{exercise}

\begin{exercise}
$\int \left(-7x^4+8\right)\d x$
\begin{answer}
\end{answer}
\end{exercise}

\begin{exercise}
$\int \left(2e^x -4\right)\d x$
\begin{answer}
\end{answer}
\end{exercise}

\begin{exercise}
$\int \left(7^x - x^7\right)\d x$
\begin{answer}
\end{answer}
\end{exercise}


\begin{exercise}
$\int \left(\frac{15}{x}+x^{15}\right)\d x$
\begin{answer}
\end{answer}
\end{exercise}


\begin{exercise}
$\int \left(-3\sin(x) + \tan(x)\right)\d x$
\begin{answer}
\end{answer}
\end{exercise}

\begin{exercise}
$\int \left(\sec^2(x) -\csc^2(x)\right) \d x$
\begin{answer}
\end{answer}
\end{exercise}


\begin{exercise}
$\int\left(\frac{1}{x} + \frac{1}{x^2} + \frac{1}{\sqrt{x}}\right)\d x$
\begin{answer}
\end{answer}
\end{exercise}

\begin{exercise}
$\int\left(\frac{17}{1 + x^2} +\frac{13}{x}\right)\d x$
\begin{answer}
\end{answer}
\end{exercise}

\begin{exercise}
$\int\left(\frac{\csc(x)\cot(x)}{4} - \frac{4}{\sqrt{1-x^2}}\right)\d x$
\begin{answer}
\end{answer}
\end{exercise}

\end{multicols}



\noindent Use Template~\ref{template:powerchain} to compute the
following antiderivatives:
\begin{multicols}{2}
\begin{exercise}
$\int 2x (x^2+4)^5 \d x$
\begin{answer}
\end{answer}
\end{exercise}

\begin{exercise}
$\int \frac{\ln(x)^4}{x} \d x$ 
\begin{answer}
\end{answer}
\end{exercise}

\begin{exercise}
$\int x\sqrt{4-x^2} \d x$
\begin{answer}
\end{answer}
\end{exercise}

\begin{exercise}
$\int \frac{1}{\sqrt{2x +1}} \d x$ 
\begin{answer}
\end{answer}\end{exercise}

\begin{exercise}
$\int \frac{x}{\sqrt{x^2+1}} \d x$
\begin{answer}
\end{answer}\end{exercise}

\begin{exercise}
$\int \frac{\sqrt{\ln(x)}}{x} \d x$ 
\begin{answer}
\end{answer}\end{exercise}
\end{multicols}

\noindent Use Template~\ref{template:echain} to compute the following
antiderivatives:

\begin{multicols}{2}
\begin{exercise}
$\int 3x^2 e^{x^3-1} \d x$ 
\begin{answer}
\end{answer}\end{exercise}

\begin{exercise}
$\int x e^{3(x^2)} \d x$ 
\begin{answer}
\end{answer}\end{exercise}

\begin{exercise}
$\int 2x e^{-(x^2)} \d x$ 
\begin{answer}
\end{answer}\end{exercise}

\begin{exercise}
$\int \frac{8x}{e^{(x^2)}}\d x$ 
\begin{answer}
\end{answer}\end{exercise}

\begin{exercise}
$\int x e^{5x} \d x$ 
\begin{answer}
\end{answer}\end{exercise}

\begin{exercise}
$\int x e^{-x/2} \d x$ 
\begin{answer}
\end{answer}\end{exercise}
\end{multicols}

\noindent Use Template~\ref{template:lnchain} to compute the following
antiderivatives:

\begin{multicols}{2}
\begin{exercise}
$\int \frac{1}{x} \d x$
\begin{answer}
\end{answer}\end{exercise}

\begin{exercise}
$\int \frac{1}{x\ln(x^2)} \d x$ 
\begin{answer}
\end{answer}\end{exercise}

\begin{exercise}
$\int \frac{x^4}{x^5+1} \d x$
\begin{answer}
\end{answer}\end{exercise}

\begin{exercise}
$\int \frac{x^2}{3-x^3} \d x$  
\begin{answer}
\end{answer}\end{exercise}

\begin{exercise}
$\int \frac{e^x}{1+3e^x} \d x$ 
\begin{answer}
\end{answer}\end{exercise}

\begin{exercise}
$\int \frac{e^{2x}-e^{-2x}}{e^{2x}+e^{-2x}} \d x$ 
\begin{answer}
\end{answer}\end{exercise}
\end{multicols}

\noindent Use Template~\ref{template:trigchain} to compute the
following antiderivatives:


\begin{multicols}{2}
\begin{exercise}
$\int 5x^4 \sin(x^5+3) \d x$ 
\begin{answer}
\end{answer}\end{exercise}

\begin{exercise}
$\int x \cos(-2x^2) \d x$
\begin{answer}
\end{answer}\end{exercise}

\begin{exercise}
$\int x \sin(5x^2) \d x$
\begin{answer}
\end{answer}\end{exercise}

\begin{exercise}
$\int 8x\cos(x^2)\d x$       
\begin{answer}
\end{answer}\end{exercise}

\begin{exercise}
$\int 4x \sin(2x) \d x$   
\begin{answer}
\end{answer}\end{exercise}

\begin{exercise}
$\int x\cos(x/5) \d x$ 
\begin{answer}
\end{answer}\end{exercise}
\end{multicols}


\end{exercises}
