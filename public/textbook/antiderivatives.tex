\chapter{Antiderivatives}

\section{Basic Antiderivatives}

Computing derivatives is not too difficult. At this point, you should
be able to take the derivative of almost any function you can write
down. However, undoing derivatives is much harder. This process of
undoing a derivative is called taking an \textit{antiderivative}.

\begin{definition}\index{antiderivative}
A function $F(x)$ is called an \textbf{antiderivative} of $f(x)$ on an
interval if
\[
F'(x) = f(x)
\]
for all $x$ in the interval.
\end{definition}


\begin{definition}\index{antiderivative!notation}\index{indefinite integral}
The antiderivative is denoted by
\[
\int f(x) \d x = F(x)+C,
\]
where $C$ is a constant indicating that there a many possible
antiderivaties, each varying by the addition of a constant.  This is
often called the \textbf{indefinite integral}.
\end{definition}

Here are the basic antiderivatives. Note they come \textit{directly}
from our derivative rules.


\begin{fullwidth}
\begin{mainTheorem}[Basic Antiderivatives of Common Functions]\label{theorem:basicAnti} \hfil
\begin{multicols}{3}
\begin{itemize}
\item $\int k \d x= kx+C$.
\item $\int x^n \d x= \frac{x^{n+1}}{n+1}+C\qquad(n\ne-1)$.
\item $\int e^x \d x= e^x + C$.
\item $\int a^x \d x= \frac{a^x}{\ln(a)}+C$.
\item $\int \frac{1}{x} \d x= \ln|x|+C$.
\item $\int \cos(x) \d x = \sin(x) + C$.
\item $\int \sin(x) \d x = -\cos(x) + C$.  
\item $\int \tan(x) \d x = \ln|\cos(x)| + C$.  
\item $\int \sec^2(x) \d x = \tan(x) + C$. 
\item $\int \csc^2(x) \d x = -\cot(x) + C$.
\item $\int \sec(x)\tan(x) \d x = \sec(x) + C$.
\item $\int \csc(x)\cot(x) \d x = -\csc(x) + C$.
\item $\int \frac{1}{x^2+1}\d x = \arctan(x) + C$.
\item $\int \frac{1}{\sqrt{1-x^2}}\d x= \arcsin(x)+C$.
\end{itemize}
\end{multicols}
\end{mainTheorem}
\end{fullwidth}
It may seem that one could simply memorize these antiderivatives and
antidifferentiating would be as easy as differentiating. This is
\textbf{not} the case. The issue comes up when trying to combine these
functions.  When taking derivatives we have the \textit{product rule}
and the \textit{chain rule}. The analogues of these two rules are much
more difficult to deal with when taking antiderivatives. However, not
all is lost. We have the following analogue of the Sum Rule for
derivatives, Theorem~\ref{theorem:sumRule}.

\begin{mainTheorem}[The Sum Rule for Antiderivatives]\label{theorem:SRA}
Given two functions $f(x)$ and $g(x)$ where $k$ is a constant:
\begin{itemize}
\item $\int k f(x) \d x= kF(x) + C$.
\item $\int f(x) + g(x) \d x = F(x) + G(x) + C$.
\end{itemize}
\end{mainTheorem}


\begin{example}
Compute
\[
\int 3 x^7 \d x.
\]
\end{example}

\begin{solution}
Checking with our Theorem~\ref{theorem:basicAnti} and Theorem~\ref{theorem:SRA}, we see that 
\begin{align*}
\int 3 x^7 \d x &= 3 \int x^7 \d x\\
&= 3 \cdot \frac{x^8}{8}+C.
\end{align*}

\end{solution}

Now let's consider a slightly harder example.

\begin{example}
Compute
\[
\int \sin(3x) \d x
\]
\end{example}


\begin{solution}
Here we might take a guess, say
\[
\int \sin(3x) \d x \approx \cos(3x).
\]
Now take the derivative of $\cos(3x)$,
\[
\ddx \cos(3x) = -3\sin(3x).
\]
Ah, we are off by a factor of $3$. Hence
\[
\int \sin(3x) \d x = \frac{\sin(3x)}{-3} + C.
\]
\end{solution}





 We advise that the mathematician view antiderivatives
as a sort of \textit{puzzle}. Let's see an example.
