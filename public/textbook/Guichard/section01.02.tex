\section{Distance Between Two Points; Circles}{}{}

Given two points $(x_1,y_1)$ and $(x_2,y_2)$, recall that their
horizontal distance from one another is $\Delta x=x_2-x_1$ and their
vertical distance from one another is $\Delta y=y_2-y_1$.  (Actually,
the word ``distance'' normally denotes ``positive distance''. $\Delta
x$ and $\Delta y$ are {\it signed\/} distances, but this is clear from
context.)  The actual (positive) distance from one point to the other
is the length of the hypotenuse of a right triangle with legs $|\Delta
x|$ and $|\Delta y|$, as shown in figure~\xrefn{fig:distance between
points}.  The Pythagorean theorem then says that the distance between
the two points is the square root of the sum of the squares of the
horizontal and vertical sides:
$$
  \hbox{distance}
    =\sqrt{(\Delta x)^2+(\Delta y)^2}=\sqrt{(x_2-x_1)^2+ (y_2-y_1)^2}.
$$
For example, the distance between points $A=(2,1)$ and $B=(3,3)$ is
$\sqrt{(3-2)^2+(3-1)^2}=\sqrt{5}$.

% BADBAD
% \figure
% \vbox{\beginpicture
% \normalgraphs
% \ninepoint
% \setcoordinatesystem units <0.5truein,0.5truein>
% \setplotarea x from 0 to 3, y from 0 to 2
% \putrule from 0 0 to 3 0
% \putrule from 3 0 to 3 2
% \plot 0 0 3 2 /
% \put {$(x_1,y_1)$} [r] <-5pt,0pt> at 0 0
% \put {$(x_2,y_2)$} [l] <5pt,0pt> at 3 2
% \put {$\Delta x$} [t] <0pt,-5pt> at 1.5 0
% \put {$\Delta y$} [l] <5pt,0pt> at 3 1
% \endpicture}
% \figrdef{fig:distance between points}
% \endfigure{Distance between two points, $\Delta x$ and $\Delta y$ positive.}

As a special case of the distance formula, suppose we want to know the
distance of a point $(x,y)$ to the origin.  According to the distance
formula, this is $\sqrt{(x-0)^2+(y-0)^2}=\sqrt{x^2+y^2}$.

A point $(x,y)$ is at a distance $r$ from the origin if and only if
$\sqrt{x^2+y^2}=r$, or, if we square both sides: $x^2+y^2=r^2$.  This is
the equation of the circle\index{circle!equation of} 
of radius $r$ centered at the origin.
The special case $r=1$ is called the unit 
circle\index{unit circle}\index{circle!unit}; 
its equation is
$x^2+y^2=1$.

Similarly, if $C(h,k)$ is any fixed point, then a point $(x,y)$ is at a
distance $r$ from the point $C$ if and only if $\sqrt{(x-h)^2+(y-k)^2}=r$,
i.e., if and only if 
$$
(x-h)^2+(y-k)^2=r^2.
$$
This is the equation of the circle\index{circle!equation of} 
of radius $r$ centered at the
point $(h,k)$.  For example, the circle of radius 5 centered at the
point $(0,-6)$ has equation $(x-0)^2+(y--6)^2=25$, or
$x^2+(y+6)^2=25$.  If we
expand this we get $x^2+y^2+12y+36=25$ or 
$x^2+y^2+12y+11=0$, but the original form is usually more useful.

\begin{example} Graph the circle $x^2-2x+y^2+4y-11=0$. With a little thought
we convert this to $(x-1)^2+(y+2)^2-16=0$ or $(x-1)^2+(y+2)^2=16$.
Now we see that this is the circle with radius 4 and center $(1,-2)$,
which is easy to graph.
\end{example}

\begin{exercises}

\begin{exercise}
Find the equation of the circle of radius 3 centered at: 
\begin{multicols}{2}
\begin{enumerate} 
\item $(0,0)$
\item $(5,6)$  
\item $(-5,-6)$
\columnbreak  
\item $(0,3)$
\item $(0,-3)$
\item $(3,0)$
\end{enumerate}
\end{multicols}
\begin{answer} (a) $x^2+y^2=9$
{} (b) $(x-5)^2+(y-6)^2=9$
{} (c) $(x+5)^2+(y+6)^2=9$
%\item{} (c) $x^2+(y-3)^2=9$
%\item{} (c) $x^2+(y+3)^2=9$
%\item{} (c) $(x-3)^2+y^2=9$
\end{answer}\end{exercise}

\begin{exercise}
For each pair of points $A=(x_1,y_1)$ and $B=(x_2,y_2)$ find (i) $\Delta x$
and $\Delta y$ in going from $A$ to $B$, (ii) the slope of the line joining
$A$ and $B$, (iii) the equation of the line joining $A$ and $B$ in the form
$y=mx+b$, (iv) the distance from $A$ to $B$, and (v) an equation of
the circle with center at $A$ that goes through $B$.

\begin{multicols}{2}
\begin{enumerate} 
\item $A = (2,0)$, $B = (4,3)$ 
\item $A=(1,-1)$, $B=(0,2)$
\item $A=(0,0)$, $B=(-2,-2)$ 
\columnbreak
\item $A=(-2,3)$, $B=(4,3)$
\item $A=(-3,-2)$, $B=(0,0)$ 
\item $A=(0.01,-0.01)$, $B=(-0.01,0.05)$
\end{enumerate}
\end{multicols}
\begin{answer} (a) $\Delta x=2$, $\Delta y = 3$, $m=3/2$, $y=(3/2)x-3$, $\sqrt{13}$

{}(b) $\Delta x=-1$, $\Delta y = 3$, $m=-3$, $y=-3x+2$, $\sqrt{10}$

{}(c) $\Delta x=-2$, $\Delta y = -2$, $m=1$, $y=x$, $\sqrt{8}$
\end{answer}\end{exercise}

\begin{exercise}
Graph the circle $x^2+y^2+10y=0$.
\end{exercise}

\begin{exercise}
Graph the circle $x^2-10x+y^2=24$.
\end{exercise}

\begin{exercise}
Graph the circle $x^2-6x+y^2-8y=0$.
\end{exercise}

\begin{exercise} Find the standard equation of the circle passing through $(-2,1)$
 and tangent to the line $3x-2y =6$ at the point $(4,3)$.  Sketch. 
 (Hint: The line through the center of the circle and the point of tangency
 is perpendicular to the tangent line.)
\begin{answer} $(x+2/7)^2+(y-41/7)^2=1300/49$
\end{answer}\end{exercise}

\end{exercises}
