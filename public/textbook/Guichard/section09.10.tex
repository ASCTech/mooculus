% Adapted from Mike Wills material.
\section{Surface Area}{}{}
\label{sec:surface area}
\nobreak
Another geometric question that arises naturally is: ``What is the
surface area of a volume?'' For example, what is the surface area of a
sphere? More advanced techniques are required to approach this
question in general, but we can compute the areas of some volumes
generated by revolution.

As usual, the question is: how might we approximate the surface area?
For a surface obtained by rotating a curve around an axis, we can take
a polygonal approximation to the curve, as in the last section, and
rotate it around the same axis. This gives a surface composed of many
``truncated cones;'' a truncated cone is called a {\dfont
  frustum\index{frustum}\/} of a cone. Figure~\xrefn{fig:approximating
  surface area} illustrates this approximation. 

\figure
\vbox{\centerline{\beginpicture
\normalgraphs
\ninepoint
\setcoordinatesystem units <0.25truecm,0.25truecm>
\setplotarea x from 0 to 30, y from -15 to 15
\put {\hbox{\epsfxsize6cm\epsfbox{surface_area_approx.eps}}} at 30 0
\put {\hbox{\epsfxsize6cm\epsfbox{surface_area.eps}}} at 0 0
\endpicture}}
\figrdef{fig:approximating surface area}
\endfigure{Approximating a surface (left) by portions of cones (right).}

So we need to be able to compute the area of a frustum of a cone.
Since the frustum can be formed by removing a small cone from the top
of a larger one, we can compute the desired area if we know the
surface area of a cone.
Suppose a right circular cone has base radius $r$ and slant height
$h$. If we cut the cone from the vertex to the base circle and
flatten it out, we obtain a sector of a circle with radius $h$ and arc
length $2\pi r$, as in figure~\xrefn{fig:area of cone}. The angle at
the center, in radians, is then $2\pi r/h$, and the area of the cone
is equal to the area of the sector of the circle. Let $A$ be the area
of the sector; since the area of the entire circle is $\ds \pi h^2$, we
have
$$
  \eqalign{{A\over\pi h^2}&={2\pi r/h\over 2\pi} \\
  A &= \pi r h. \\}
$$

\figure
\vbox{\beginpicture
\normalgraphs
\sevenpoint
\setcoordinatesystem units <1truecm,1truecm> point at 0 0
\setplotarea x from -1.5 to 1.5, y from -1 to 5.3
\ellipticalarc  axes ratio 3:1  180 degrees from -1.5 0 center at 0 0
\plot 1.5 0 0 5 -1.5 0 /
\putrule from -1.5 0 to 0 0
\put {$r$} [t] <0pt,-3pt> at -0.75 0
\put {$h$} [bl] <2pt,2pt> at 0.75 2.5 
\setdashes
\ellipticalarc  axes ratio 3:1  180 degrees from 1.5 0 center at 0 0
\setsolid
\setcoordinatesystem units <1truecm,1truecm> point at -4 0
\circulararc 104 degrees from 5.22 0 center at 0 0
\circulararc 104 degrees from 1 0 center at 0 0
\plot 5.22 0 0 0 -1.26 5.065 /
\put {$h$} [t] <0pt,-3pt> at 2.61 0
\put {$2\pi r$} [bl] <2pt,2pt> at 3.69 3.69  
\put {$2\pi r/h$} [bl] <2pt,2pt> at 0.71 0.71
\endpicture}
\figrdef{fig:area of cone}
\endfigure{The area of a cone.}

Now suppose we have a frustum of a cone with slant height $h$ and
radii $\ds r_0$ and $\ds r_1$, as in figure~\xrefn{fig:frustum}. The area of
the entire cone is $\ds \pi r_1(h_0+h)$, and the area of the small cone is
$\ds \pi r_0 h_0$; thus, the area of the frustum is $\ds \pi r_1(h_0+h)-\pi
r_0 h_0=\pi((r_1-r_0)h_0+r_1h)$. By similar triangles, 
$${h_0\over r_0}={h_0+h\over r_1}.$$
With a bit of algebra this becomes $\ds (r_1-r_0)h_0= r_0h$; substitution
into the area gives
$$
  \pi((r_1-r_0)h_0+r_1h)=\pi(r_0h+r_1h)=\pi h(r_0+r_1)=2\pi
  {r_0+r_1\over2} h = 2\pi r h.
$$
The final form is particularly easy to remember, with $r$ equal to the
average of $\ds r_0$ and $\ds r_1$, as it is also the formula for the area of
a cylinder. (Think of a cylinder of radius $r$ and height $h$ as the
frustum of a cone of infinite height.)

\figure
\vbox{\beginpicture
\normalgraphs
\sevenpoint
\setcoordinatesystem units <1truecm,1truecm> point at 0 0
\setplotarea x from -1.5 to 1.5, y from -1 to 5
\ellipticalarc  axes ratio 3:1  180 degrees from -1.5 0 center at 0 0
\plot 1.5 0 0 5 -1.5 0 /
\putrule from -1.5 0 to 0 0
\put {$r_1$} [t] <0pt,-3pt> at -0.75 0
\put {$h_0$} [bl] <2pt,2pt> at 0.45 3.5 
\ellipticalarc  axes ratio 3:1  180 degrees from -0.9 2 center at 0 2
\putrule from -0.9 2 to 0 2
\put {$r_0$} [b] <0pt,3pt> at -0.45 2
\put {$h$} [bl] <2pt,2pt> at 1.2 1
\setdashes
\ellipticalarc  axes ratio 3:1  180 degrees from 1.5 0 center at 0 0
\ellipticalarc  axes ratio 3:1  110 degrees from 0.9 2 center at 0 2
\endpicture}
\figrdef{fig:frustum}
\endfigure{The area of a frustum.}

Now we are ready to approximate the area of a surface of
revolution. On one subinterval, the situation is as shown in
figure~\xrefn{fig:surface subinterval}. When the line joining two
points on the curve is rotated around the $x$-axis, it forms a frustum
of a cone. The area is
$$
  2\pi r h= 2\pi {f(x_i)+f(x_{i+1})\over2}
    \sqrt{1+(f'(t_i))^2}\,\Delta x.
$$
Here
$\ds \sqrt{1+(f'(t_i))^2}\,\Delta x$ is the length of the line segment, 
as we found in the previous section. Assuming $f$ is a continuous
function, there must be some $\ds x_i^*$ in $\ds [x_i,x_{i+1}]$
such that
$\ds (f(x_i)+f(x_{i+1}))/2 = f(x_i^*)$, so
the approximation for the
surface area is
$$\sum_{i=0}^{n-1} 2\pi f(x_i^*)\sqrt{1+(f'(t_i))^2}\,\Delta x.$$
This is not quite the sort of sum we have seen before, as it contains
two different values in the interval $\ds [x_i,x_{i+1}]$, namely
$\ds x_i^*$ and $\ds t_i$. Nevertheless, using more advanced techniques
than we have available here, it turns out that
$$\lim_{n\to\infty} 
\sum_{i=0}^{n-1} 2\pi f(x_i^*)\sqrt{1+(f'(t_i))^2}\,\Delta x=
\int_a^b 2\pi f(x)\sqrt{1+(f'(x))^2}\,dx$$ 
is the surface area we seek. (Roughly speaking, this is because while
$\ds x_i^*$ and $\ds t_i$ are distinct values in $\ds[x_i,x_{i+1}]$,
they get closer and closer to each other as the length of the interval
shrinks.) 

\figure
\vbox{\beginpicture
\normalgraphs
\sevenpoint
\setcoordinatesystem units <1.3truecm,1.3truecm> point at 0 0
\setplotarea x from 0 to 5, y from 0 to 3
\axis left /
\axis bottom ticks short withvalues {$x_i$}
  {$x_i^*$} {$x_{i+1}$} / at 2.5 3 4 / /
\plot 2.5 1.5 4 2.5 /
\setquadratic
\plot 2.5 1.5 3 2 4 2.5 /
\put {$(x_i,f(x_i))$} [r] <-3pt,0pt> at 2.5 1.5
\put {$(x_{i+1},f(x_{i+1}))$} [l] <3pt,0pt> at 4 2.5
%\putrule from 3.25 0 to 3.25 2
\setdashes <2pt>
\putrule from 3 0 to 3 2
\putrule from 3 2 to 3.25 2
\setdashes
\putrule from 2.5 0 to 2.5 1.5
\putrule from 4 0 to 4 2.5
\endpicture}
\figrdef{fig:surface subinterval}
\endfigure{One subinterval.}

\begin{example} We compute the surface area of a sphere of radius $r$.
The sphere can be obtained by rotating the graph of
  $\ds f(x)=\sqrt{r^2 - x^2}$ about the $x$-axis.
The derivative $f'$ is $\ds -x/\sqrt{r^2-x^2}$, so the surface area is
given by
$$\eqalign{
A&=2\pi \int_{-r }^r \sqrt{r^2 - x^2}\sqrt{1+{x^2\over r^2-x^2}}\,dx \\
&=2\pi \int_{-r }^r \sqrt{r^2 - x^2}\sqrt{r^2\over r^2-x^2}\,dx \\
&=2\pi \int_{-r }^r r\,dx=2\pi r\int_{-r }^r 1\,dx=4\pi r^2 \\}$$
\vskip-10pt\end{example}

If the curve is rotated around the $y$ axis, the formula is nearly
identical, because the length of the line segment we use to
approximate a portion of the curve doesn't change. Instead of the
radius $\ds f(x_i^*)$, we use the new radius $\ds \bar x_i=
(x_i+x_{i+1})/2$, and the surface area integral becomes
$$\int_a^b 2\pi x\sqrt{1+(f'(x))^2}\,dx.$$

\begin{example} \relax
\label{exam:surface around y axis}
Compute the area of the surface formed when $\ds f(x)
=x^2$ between $0$ and $2$ is rotated around the $y$-axis.

We compute $f'(x)= 2x$, and then
$$2\pi\int_0^2 x\sqrt{1+4x^2}\,dx={\pi\over6}(17^{3/2}-1),$$
by a simple substitution.
\end{example}

\begin{exercises}

\begin{exercise} Compute the area of the surface formed when $\ds f(x)=2\sqrt{1-x}$
between $-1$ and $0$ is rotated around the $x$-axis.
\begin{answer} $\ds 8\pi\sqrt3-{16\pi\sqrt2\over 3}$
\end{answer}\end{exercise}

\begin{exercise} Compute the surface area of example~\xrefn{exam:surface
  around y axis} by rotating $\ds f(x)=\sqrt x$ around the $x$-axis.

\begin{exercise} Compute the area of the surface formed when 
$\ds f(x)=x^3$ between $1$ and $3$ is rotated around the $x$-axis.
\begin{answer} $\ds {730\pi\sqrt{730}\over27}-{10\pi\sqrt{10}\over 27}$
\end{answer}\end{exercise}

\begin{exercise} Compute the area of the surface formed when 
$\ds f(x)=2 +\cosh (x)$ between $0$ and $1$ is rotated around the
  $x$-axis.
\begin{answer} $\ds \pi +2\pi e+ {1\over4}\pi e^2-{\pi\over4e^2}-{2\pi\over e}$
\end{answer}\end{exercise}

\begin{exercise} Consider the surface obtained by rotating the graph of $\ds
f(x)=1/x$, $x\geq 1$, around the $x$-axis. This surface is called
{\dfont Gabriel's horn\index{Gabriel's horn}\/} or {\dfont Toricelli's
  trumpet\index{Toricelli's trumpet}}.  
In exercise~\xrefn{exer:gabriels horn} in 
section~\xrefn{sec:improper integrals} we saw that Gabriel's horn has
finite volume. 
Show that Gabriel's horn has
infinite surface area.

\begin{exercise} Consider the circle $\ds (x-2)^2+y^2 = 1$. Sketch the
surface obtained by rotating this circle about the $y$-axis. (The
surface is called a {\dfont torus\index{torus}}.) What is the surface area?
\begin{answer} $8\pi^2$
\end{answer}\end{exercise}

\begin{exercise} Consider the ellipse with equation $\ds x^2/4+y^2 = 1$.
If the ellipse is rotated around the $x$-axis it forms 
an {\dfont ellipsoid\index{ellipsoid}}.
Compute the surface area.
\begin{answer} $\ds 2\pi+{8\pi^2\over 3\sqrt{3}}$
\end{answer}\end{exercise}

\begin{exercise} Generalize the preceding result: rotate the ellipse
given by $\ds x^2/a^2+y^2/b^2=1$ about the
$x$-axis and find the surface area of the resulting ellipsoid. You
should consider two cases, when $a>b$ and when $a<b$. Compare to the
area of a sphere.
\begin{answer} $a>b$: $\ds 2\pi b^2+$\hfill\break
\hbox{\hskip1cm}$\ds {2\pi a^2b\over\sqrt{a^2-b^2}}
  \arcsin(\sqrt{a^2-b^2}/a)$,\hfill\break
$a<b$: $\ds 2\pi b^2+ $\hfill\break
\hbox{\hskip1cm}$\ds {2\pi a^2b\over\sqrt{b^2-a^2}}
  \ln\left({b\over a}+{\sqrt{b^2-a^2}\over a}\right)$
\end{answer}\end{exercise}

\end{exercises}
