\section{Vectors}{}{}
\nobreak A vector\index{vector} is a quantity consisting of a
non-negative magnitude and a direction. We could represent a vector in
two dimensions as $(m,\theta)$, where $m$ is the magnitude and
$\theta$ is the direction, measured as an angle from some agreed upon
direction. For example, we might think of the vector $\ds
(5,45^\circ)$ as representing ``5 km toward the northeast''; that is,
this vector might be a {\dfont displacement
vector\index{vector!displacement}\index{displacement vector}\/},
indicating, say, that your grandfather walked 5 kilometers toward the
northeast to school in the snow. On the other hand, the same vector
could represent a velocity, indicating that your grandfather walked at
5 km/hr toward the northeast. What the vector does not indicate is
where this walk occurred: a vector represents a magnitude and a
direction, but not a location. Pictorially it is useful to represent a
vector as an arrow; the direction of the vector, naturally, is the
direction in which the arrow points; the magnitude of the vector is
reflected in the length of the arrow.

It turns out that many, many quantities behave as vectors, e.g.,
displacement, velocity, acceleration, force. Already we can get some
idea of their usefulness using displacement vectors. Suppose that your
grandfather walked 5 km NE and then 2 km SSE; if the terrain allows,
and perhaps armed with a compass, how could your grandfather have
walked directly to his destination? We can use vectors (and a bit of
geometry) to answer this question. We begin by noting that since
vectors do not include a specification of position, we can ``place''
them anywhere that is convenient. So we can picture your grandfather's
journey as two displacement vectors drawn head to tail:
$$\vbox{\beginpicture
\normalgraphs
\ninepoint
\setcoordinatesystem units <7truemm,7truemm>
\setplotarea x from 0 to 5, y from 0 to 3.6
\arrow <4pt> [0.35, 1] from 0 0 to 3.54 3.54
\arrow <4pt> [0.35, 1] from 3.54 3.54 to 4.3 1.69
\setdashes
\arrow <4pt> [0.35, 1] from 0 0 to 4.3 1.69
\endpicture
}$$
The displacement vector for the shortcut route is the vector drawn
with a dashed line, from the tail of the first to the head of the
second. With a little trigonometry, we can compute that the third
vector has magnitude approximately 4.62 and direction $\ds
21.43^\circ$, so walking 4.62 km in the direction $\ds 21.43^\circ$
north of east (approximately ENE) would get your grandfather to
school. This sort of calculation is so common, we dignify it with a
name: we say that the third vector is the {\dfont sum\index{sum!of
vectors}\index{vector!sum}\/} of the other two vectors. There is
another common way to picture the sum of two vectors. Put the vectors
tail to tail and then complete the parallelogram they indicate; the
sum of the two vectors is the diagonal of the
parallelogram\index{parallelogram!and vector sum}:
$$\vbox{\beginpicture
\normalgraphs
\ninepoint
\setcoordinatesystem units <7truemm,7truemm>
\setplotarea x from 0 to 5, y from -2 to 3.6
\arrow <4pt> [0.35, 1] from 0 0 to 3.54 3.54
\arrow <4pt> [0.35, 1] from 0 0 to 0.77 -1.85
\setdashes
\arrow <4pt> [0.35, 1] from 0 0 to 4.3 1.69
\setdots
\plot 3.54 3.54 4.3 1.69 0.77 -1.85 /
\endpicture
}$$
This is a more natural representation in some circumstances. For
example, if the two original vectors represent forces acting on an
object, the sum of the two vectors is the net or effective force on
the object, and it is nice to draw all three with their tails at the
location of the object.

We also define {\dfont scalar multiplication\index{scalar
multiplication}\index{vector!scalar multiplication}\/} for
vectors: if $\bf A$ is a vector $(m,\theta)$ and $a\ge 0$ is a real
number, the vector $a\bf A$ is $(am,\theta)$, namely, it points in
the same direction but has $a$ times the magnitude. If $a<0$, $a\bf
A$ is $(|a|m,\theta+\pi)$, with $|a|$ times the magnitude and
pointing in the opposite direction (unless we specify otherwise,
angles are measured in radians).

Now we can understand subtraction of vectors: 
${\bf A}-{\bf B}={\bf A}+(-1){\bf B}$:
$$\vbox{\beginpicture
\normalgraphs
\ninepoint
\setcoordinatesystem units <7truemm,7truemm>
\setplotarea x from 0 to 5, y from 0 to 3.6
\altarrow <4pt,5pt> [0.35, 1] from 0 0 to 3.54 3.54
\arrow <4pt> [0.35, 1] from 0 0 to 4.3 1.69
\altarrow <4pt,5pt> [0.35, 1] from 4.3 1.69 to 3.54 3.54
\put {$\bf A$} [br] <-2pt,2pt> at 1.72 1.72
\put {$\bf B$} [tl] <2pt,-2pt> at 2.15 0.84
\put {${\bf A}-\bf B$} [bl] <2pt,2pt> at 3.92 2.615
\setcoordinatesystem units <7truemm,7truemm> point at -6 0
\setplotarea x from 0 to 5, y from 0 to 3.6
\altarrow <4pt,5pt> [0.35, 1] from 0 0 to 3.54 3.54
\altarrow <4pt,10pt> [0.35, 1] from 4.3 1.69 to 0 0
\altarrow <4pt,5pt> [0.35, 1] from 4.3 1.69 to 3.54 3.54
\put {$\bf A$} [br] <-2pt,2pt> at 1.72 1.72
\put {$-\bf B$} [tl] <2pt,-2pt> at 2.15 0.84
\put {${\bf A}-\bf B$} [bl] <2pt,2pt> at 3.92 2.615
\endpicture}$$
Note that as you would expect, ${\bf B} + ({\bf A}-{\bf B}) = {\bf A}$.

We can represent a vector in ways other than $(m,\theta)$, and in fact
$(m,\theta)$ is not generally used at all. How else could we describe
a particular vector? Consider again the vector $\ds (5,45^\circ)$. Let's
draw it again, but impose a coordinate system. If we put the tail of
the arrow at the origin, the head of the arrow ends up at
the point $\ds (5/\sqrt2,5/\sqrt2)\approx(3.54, 3.54)$.
$$\vbox{\beginpicture
\normalgraphs
\ninepoint
\setcoordinatesystem units <7truemm,7truemm>
\setplotarea x from 0 to 5, y from 0 to 4
\axis left /
\axis bottom /
\arrow <4pt> [0.35, 1] from 0 0 to 3.54 3.54
\put {$(3.54, 3.54)$} [l] <3pt,0pt> at 3.54 3.54
\circulararc 45 degrees from 1 0 center at 0 0
\put {$5$} [br] <-2pt,2pt> at 1.72 1.72
\put {$45^\circ$} [bl] <2pt,2pt> at 1 0.3
\endpicture
}$$
In this picture the coordinates $(3.54,3.54)$ identify the head of the
arrow, provided we know that the tail of the arrow has been placed at
$(0,0)$. Then in fact the vector can always be identified as
$(3.54,3.54)$, no matter where it is placed; we just have to remember
that the numbers 3.54 must be interpreted as a {\it change\/} from the
position of the tail, not as the actual coordinates of the arrow head;
to emphasize this we will write $\langle 3.54,3.54\rangle$ to mean the
vector and $(3.54,3.54)$ to mean the point. Then if the vector
$\langle 3.54,3.54\rangle$ is drawn with its tail at $(1,2)$ it looks
like this:
$$\vbox{\beginpicture
\normalgraphs
\ninepoint
\setcoordinatesystem units <7truemm,7truemm>
\setplotarea x from 0 to 5, y from 0 to 6
\axis left /
\axis bottom /
\arrow <4pt> [0.35, 1] from 1 2 to 4.54 5.54
\setdashes
\plot 1 2 4.54 2 4.54 5.54 /
\put {$3.54$} [t] <0pt,-3pt> at 2.72 2
\put {$3.54$} [l] <3pt,0pt> at 4.54 3.72
\put {$(4.54,5.54)$} [l] <3pt,0pt> at 4.54 5.54
\endpicture
}$$ 
Consider again the two part trip: 5 km NE and then 2 km SSE. The
vector representing the first part of the trip is $\ds \langle
5/\sqrt2,5/\sqrt2\rangle$, and the second part of the trip is
represented by $\langle 2\cos(-3\pi/8),2\sin(-3\pi/8)\rangle
\approx\langle 0.77,-1.85 \rangle$.  We can represent the sum of these
with the usual head to tail picture:
$$\vbox{\beginpicture
\normalgraphs
\ninepoint
\setcoordinatesystem units <9truemm,9truemm>
\setplotarea x from 0 to 5, y from 0 to 5
\axis bottom /
\axis left /
\arrow <4pt> [0.35, 1] from 0 0 to 3.54 3.54
\arrow <4pt> [0.35, 1] from 3.54 3.54 to 4.3 1.69
\setdashes
\arrow <4pt> [0.35, 1] from 0 0 to 4.3 1.69
\put {$3.54$} [b] <0pt,3pt> at 1.7 3.54
\put {$0.77$} [b] <0pt,3pt> at 3.9 3.54
\put {$-1.85$} [l] <3pt,0pt> at 4.3 2.69
\put {$(4.3,1.69)$} [l] <3pt,0pt> at 4.3 1.69
\plot 0 3.54 4.3 3.54 4.3 1.69 /
\endpicture
}$$
It is clear from the picture that the coordinates of the destination
point are $\ds (5/\sqrt2+2\cos(-3\pi/8),5/\sqrt2+2\sin(-3\pi/8))$ or
approximately $(4.3,1.69)$, so the sum of the two vectors is $\ds
\langle 5/\sqrt2+2\cos(-3\pi/8),5/\sqrt2+2\sin(-3\pi/8)\rangle \approx
\langle 4.3,1.69\rangle$. Adding the two vectors is easier in this
form than in the $(m,\theta)$ form, provided that we're willing to
have the answer in this form as well.

It is easy to see that scalar multiplication and vector subtraction
are also easy to compute in this form: $a\langle v,w\rangle=\langle
av,aw\rangle$ and $\ds \langle v_1,w_1\rangle - \langle v_2,w_2\rangle
=\langle v_1-v_2,w_1-w_2\rangle$. What about the magnitude? The
magnitude of the vector $\langle v,w\rangle$ is still the length of
the corresponding arrow representation; this is the distance from the
origin to the point $(v,w)$, namely, the distance from the tail to the
head of the arrow. We know how to compute distances, so the magnitude of
the vector is simply $\ds \sqrt{v^2+w^2}$, which we also denote with
absolute value bars: $\ds |\langle v,w\rangle|=\sqrt{v^2+w^2}$.

In three dimensions, vectors are still quantities consisting of a
magnitude and a direction, but of course there are many more possible
directions. It's not clear how we might represent the direction
explicitly, but the coordinate version of vectors makes just as much
sense in three dimensions as in two. By $\langle 1,2,3\rangle$ we mean
the vector whose head is at $(1,2,3)$ if its tail is at the origin. As
before, we can place the vector anywhere we want; if it has its tail
at $(4,5,6)$ then its head is at $(5,7,9)$. It remains true that
arithmetic is easy to do with vectors in this form:
$$\eqalign{
  &a\langle v_1,v_2,v_3\rangle=\langle av_1,av_2,av_3\rangle \\
  &\langle v_1,v_2,v_3\rangle + \langle w_1,w_2,w_3\rangle
  =\langle v_1+w_1,v_2+w_2,v_3+w_3\rangle \\
  &\langle v_1,v_2,v_3\rangle - \langle w_1,w_2,w_3\rangle
  =\langle v_1-w_1,v_2-w_2,v_3-w_3\rangle \\}
$$
The magnitude of the vector is again the distance from the origin to
the head of the arrow, or 
$\ds |\langle v_1,v_2,v_3\rangle|=\sqrt{v_1^2+v_2^2+v_3^2}$.

\figure
\vbox{\beginpicture
\normalgraphs
\ninepoint
\setcoordinatesystem units <6truemm,6truemm>
\setplotarea x from 0 to 5, y from 0 to 6
\axis left /
\axis bottom /
\plot 0 0 -3 -3 /
\altarrow <4pt,10pt> [0.35, 1] from 0 0 to 3 4
\setdashes
\plot 4 0 3 -1 /
\plot 3 -1 -1 -1 /
\plot 3 -1 3 4 /
\setdots
\plot 0 5 4 5 3 4 -1 4 0 5 /
\plot -1 4 -1 -1 /
\plot 4 5 4 0 /
\put {$\bullet$} at 3 4
%\put {$(2,4,5)$} [tr] <-4pt,-4pt> at 3 4
\put {$y$} [l] <4pt,0pt> at 5 0
\put {$z$} [b] <0pt,4pt> at 0 6
\put {$x$} [tr] <-4pt,-4pt> at -3 -3
\put {$4$} [b] <0pt,3pt> at 4 0
\put {$2$} [br] <-2pt,2pt> at -1 -1
\put {$5$} [r] <-3pt,0pt> at 0 5
\endpicture}
\figrdef{fig:3d vector}
\endfigure{The vector $\langle 2,4,5\rangle$ with its tail at the origin.}

Three particularly simple vectors turn out to be quite useful: 
${\bf i}=\langle1,0,0\rangle$, ${\bf j}=\langle0,1,0\rangle$, and 
${\bf k}=\langle0,0,1\rangle$. These play much the same role for
vectors that the axes play for points. In particular, notice that
$$\eqalign{
  \langle v_1,v_2,v_3\rangle &= \langle v_1,0,0\rangle + \langle
  0,v_2,0\rangle + \langle 0,0,v_3\rangle \\
  &=v_1\langle1,0,0\rangle + v_2\langle0,1,0\rangle + v_3\langle0,0,1\rangle \\
  &= v_1{\bf i} + v_2{\bf j} + v_3{\bf k} \\
}$$

We will frequently want to produce a vector that points from one point
to another. That is, if $P$ and $Q$ are points, we seek the vector
$\bf x$ such that when the tail of $\bf x$ is placed at $P$, its head
is at $Q$; we refer to this vector as $\ds \overrightarrow{\vrule
height8pt width 0pt PQ}$. If we know the coordinates of $P$ and $Q$,
the coordinates of the vector are easy to find.

\begin{example}
Suppose $P=(1,-2,4)$ and $Q=(-2,1,3)$. The vector
$\ds \overrightarrow{\vrule height8pt width 0pt PQ}$ is
$\langle -2-1,1--2,3-4\rangle=\langle -3,3,-1\rangle$ and
$\ds \overrightarrow{\vrule height8pt width 0pt QP}=\langle 3,-3,1\rangle$.
\end{example}

\begin{exercises}

\begin{exercise} Draw the vector $\langle 3,-1\rangle$ with its tail at the
origin. 

\begin{exercise} Draw the vector $\langle 3,-1,2\rangle$ with its tail at the
origin. 

\begin{exercise} Let ${\bf A}$ be the vector with tail at the origin and head
at $(1,2)$; let ${\bf B}$ be the vector with tail at the origin and head
at $(3,1)$. Draw ${\bf A}$ and ${\bf B}$ and a vector ${\bf C}$ with 
tail at $(1,2)$ and head at $(3,1)$. Draw $\bf C$ with its tail at the origin.

\begin{exercise} Let ${\bf A}$ be the vector with tail at the origin and head
at $(-1,2)$; let ${\bf B}$ be the vector with tail at the origin and head
at $(3,3)$. Draw ${\bf A}$ and ${\bf B}$ and a vector ${\bf C}$ with 
tail at $(-1,2)$ and head at $(3,3)$. Draw $\bf C$ with its tail at the origin.

\begin{exercise} Let ${\bf A}$ be the vector with tail at the origin and head
at $(5,2)$; let ${\bf B}$ be the vector with tail at the origin and head
at $(1,5)$. Draw ${\bf A}$ and ${\bf B}$ and a vector ${\bf C}$ with 
tail at $(5,2)$ and head at $(1,5)$. Draw $\bf C$ with its tail at the origin.

\begin{exercise} Find $|{\bf v}|$, ${\bf v}+{\bf w}$, ${\bf v}-{\bf w}$,
$|{\bf v}+{\bf w}|$, $|{\bf v}-{\bf w}|$ and $-2{\bf v}$ for
${\bf v} = \langle 1,3\rangle$ and ${\bf w} = \langle -1,-5\rangle$.
\begin{answer} $\ds \sqrt{10}$, $\langle 0,-2\rangle$, $\langle 2,8\rangle$
2, $\ds 2\sqrt{17}$, $\langle -2,-6\rangle$
\end{answer}\end{exercise}

\begin{exercise} Find $|{\bf v}|$, ${\bf v}+{\bf w}$, ${\bf v}-{\bf w}$,
$|{\bf v}+{\bf w}|$, $|{\bf v}-{\bf w}|$ and $-2{\bf v}$ for
${\bf v} = \langle 1,2,3\rangle$ and ${\bf w} = \langle -1,2,-3\rangle$.
\begin{answer} $\ds \sqrt{14}$, $\langle 0,4,0\rangle$, $\langle 2,0,6\rangle$
4, $\ds 2\sqrt{10}$, $\langle -2,-4,-6\rangle$
\end{answer}\end{exercise}

\begin{exercise} Find $|{\bf v}|$, ${\bf v}+{\bf w}$, ${\bf v}-{\bf w}$,
$|{\bf v}+{\bf w}|$, $|{\bf v}-{\bf w}|$ and $-2{\bf v}$ for
${\bf v} = \langle 1,0,1\rangle$ and ${\bf w} = \langle -1,-2,2 \rangle$.
\begin{answer} $\ds \sqrt{2}$, $\langle 0,-2,3\rangle$, $\langle 2,2,-1\rangle$
$\ds \sqrt{13}$, $3$, $\langle -2, 0, -2\rangle$
\end{answer}\end{exercise}

\begin{exercise} Find $|{\bf v}|$, ${\bf v}+{\bf w}$, ${\bf v}-{\bf w}$,
$|{\bf v}+{\bf w}|$, $|{\bf v}-{\bf w}|$ and $-2{\bf v}$ for
${\bf v} = \langle 1,-1,1\rangle$ and ${\bf w} = \langle 0,0,3\rangle$.
\begin{answer} $\ds \sqrt{3}$, $\langle 1,-1,4\rangle$, $\langle 1,-1,-2\rangle$
$\ds 3\sqrt{2}$, $\ds \sqrt{6}$, $\langle -2, 2, -2\rangle$
\end{answer}\end{exercise}

\begin{exercise} Find $|{\bf v}|$, ${\bf v}+{\bf w}$, ${\bf v}-{\bf w}$,
$|{\bf v}+{\bf w}|$, $|{\bf v}-{\bf w}|$ and $-2{\bf v}$ for
${\bf v} = \langle 3,2,1\rangle$ and ${\bf w} = \langle -1,-1,-1\rangle$.
\begin{answer} $\ds \sqrt{14}$, $\langle 2,1,0\rangle$, $\langle 4,3,2\rangle$
$\ds \sqrt{5}$, $\ds \sqrt{29}$, $\langle -6,-4, -2\rangle$
\end{answer}\end{exercise}

\begin{exercise} Let $P=(4,5,6)$, $Q=(1,2,-5)$. Find 
$\ds \overrightarrow{\vrule height8pt width 0pt PQ}$. Find a vector with
the same direction as $\ds \overrightarrow{\vrule height8pt width 0pt PQ}$
but with length 1. Find a vector with
the same direction as $\ds \overrightarrow{\vrule height8pt width 0pt PQ}$
but with length 4.
\begin{answer} $\langle -3, -3, -11\rangle$,
$\langle -3/\sqrt{139},-3/\sqrt{139},-11/\sqrt{139}\rangle$
$\langle -12/\sqrt{139},-12/\sqrt{139},-44/\sqrt{139}\rangle$
\end{answer}\end{exercise}

\begin{exercise} If $A, B$, and $C$ are three points, find
$\ds \overrightarrow{\vrule height8pt width 0pt AB}+
\overrightarrow{\vrule height8pt width 0pt BC}+
\overrightarrow{\vrule height8pt width 0pt CA}$.
\begin{answer} $\langle 0,0,0\rangle$
\end{answer}\end{exercise}

\begin{exercise} Consider the 12 vectors that have their tails at the center of a
clock and their respective heads at each of the 12 digits.  What is
the sum of these vectors?  What if we remove the vector corresponding
to 4 o'clock?  What if, instead, all vectors have their
tails at 12 o'clock, and their heads on the remaining digits?
\begin{answer} $\bf 0$; $\langle -r\sqrt3/2,r/2\rangle$; $\langle
10r,0\rangle$; where $r$ is the radius of the clock
\end{answer}\end{exercise}

\begin{exercise} Let $\bf a$ and $\bf b$ be nonzero vectors in two dimensions
that are not parallel or anti-parallel.  Show, algebraically, that if
$\bf c$ is any two dimensional vector, there are scalars $s$ and $t$
such that ${\bf c}=s{\bf a}+t{\bf b}$.

\begin{exercise} Does the statement in the previous exercise hold if the vectors
$\bf a$, $\bf b$, and $\bf c$ are three dimensional vectors? Explain.

\end{exercises}
