\documentclass[14pt]{beamer}

\usepackage{xcolor}
\usepackage{colortbl}
\usepackage{pgf}
\usepackage{amsmath}
\usepackage{amssymb}
\usepackage{latexsym}
\usepackage{tikz}
\usepackage{pgfplots}
\usepackage{pdfpages}
\usepackage{ulem}

\definecolor{shaded}{RGB}{210,210,210}
\usecolortheme[named=shaded]{structure}

\definecolor{stressed}{RGB}{150,40,40}
\setbeamercolor{alerted_text}{fg=stressed}


\setbeamertemplate{navigation symbols}{}
\setbeamersize{text margin left=3mm} 
\setbeamersize{text margin right=3mm} 

%\usepackage{fontspec,xunicode,xltxtra}
%\setmainfont{Helvetica Neue Light}
%\setsansfont{Helvetica Neue Light}

%\usepackage[T1]{fontenc}
%\usepackage{tgheros}
%\renewcommand*\familydefault{\sfdefault} %% Only if the base font of the document is to be sans serif

\setbeamertemplate{sidebar right}{default}{}

\makeatletter
\define@key{beamerframe}{nofills}[true]{% top
  \beamer@frametopskip=0pt\relax%
  \beamer@framebottomskip=0pt\relax%
  \beamer@frametopskipautobreak=\beamer@frametopskip\relax%
  \beamer@framebottomskipautobreak=\beamer@framebottomskip\relax%
  \def\beamer@initfirstlineunskip{%
    \def\beamer@firstlineitemizeunskip{%
      \vskip-\partopsep\vskip-\topsep\vskip-\parskip%
      \global\let\beamer@firstlineitemizeunskip=\relax}%
    \everypar{\global\let\beamer@firstlineitemizeunskip=\relax}}
}
\makeatother

\newcommand{\setbackgroundpicturewhite}[1]{%
\usebackgroundtemplate{%
\begin{tikzpicture}%
\draw[fill=white] (current page.north west) rectangle (current page.south east);%
\node[draw,minimum width=\paperwidth,minimum height=\paperheight] [anchor=south west] (mynode) {\includegraphics[width=\paperwidth]{#1}};%
\end{tikzpicture}%
}%
%\begin{pgfpicture}{0in}{0in}{\paperwidth}{\paperheight}
%\pgfputat{\pgfxy(0,0)}{\includegraphics[width=\paperwidth]{#1}}
%\color{white}
%\pgfsetfillopacity{0.0}
%\pgfrect[fill]{\pgfxy(0,0)}{\pgfpoint{\paperwidth}{\paperheight}}
%\end{pgfpicture}
%}
}


\newcommand{\setbackgroundpictureblack}[1]{%
\usebackgroundtemplate{%
\begin{tikzpicture}%
\draw[fill=black] (current page.north west) rectangle (current page.south east);%
\node[draw,minimum width=\paperwidth,minimum height=\paperheight] [anchor=south west] (mynode) {\includegraphics[width=\paperwidth]{#1}};%
\end{tikzpicture}%
}%
%\begin{pgfpicture}{0in}{0in}{\paperwidth}{\paperheight}
%\pgfputat{\pgfxy(0,0)}{\includegraphics[width=\paperwidth]{#1}}
%\color{white}
%\pgfsetfillopacity{0.0}
%\pgfrect[fill]{\pgfxy(0,0)}{\pgfpoint{\paperwidth}{\paperheight}}
%\end{pgfpicture}
%}
}


\newcommand{\setdarkbackgroundpictureblack}[1]{%
\usebackgroundtemplate{%
\begin{tikzpicture}%
\draw[fill=black] (current page.north west) rectangle (current page.south east);%
\node[draw,very thick,minimum width=\paperwidth-\pgflinewidth,minimum height=\paperheight-\pgflinewidth] [anchor=south west] (mynode) {\includegraphics[width=\paperwidth]{#1}};%
\draw[fill=black,opacity=0.75] (current page.north west) rectangle (current page.south east);%
\end{tikzpicture}%
}}%


\newcommand{\setdarkbackgroundpicturewhite}[1]{%
\usebackgroundtemplate{%
\begin{tikzpicture}%
\draw[fill=white] (current page.north west) rectangle (current page.south east);%
\node[draw,very thick,minimum width=\paperwidth-\pgflinewidth,minimum height=\paperheight-\pgflinewidth] [anchor=south west] (mynode) {\includegraphics[width=\paperwidth]{#1}};%
\draw[fill=white,opacity=0.75] (current page.north west) rectangle (current page.south east);%
\end{tikzpicture}%
}}%


\newcommand{\clearbackgroundpicture}{\usebackgroundtemplate{}}

\begin{document}
\setbeamercolor{background canvas}{bg=white,fg=black}
\usebeamercolor[fg]{background canvas}

%%%%%%%%%%%%%%%%%%%%%%%%%%%%%%%%%%%%%%%%%%%%%%%%%%%%%%%%%%%%%%%%
\clearbackgroundpicture
\begin{frame}[nofills]
  \includegraphics[width=\textwidth]{cow.pdf}

  \vfill
  \Large
  \textsf{\textbf{Jim Fowler}} \\
  \textsf{OSU Mathematics Department}
\end{frame}

%%%%%%%%%%%%%%%%%%%%%%%%%%%%%%%%%%%%%%%%%%%%%%%%%%%%%%%%%%%%%%%%
% Coursera
\setbackgroundpicturewhite{coursera-landing-page.pdf}
\begin{frame}
\end{frame}

\setdarkbackgroundpicturewhite{coursera-landing-page.pdf}
\begin{frame}
\huge\textsf{\textbf{35k students enrolled}}
\end{frame}

\setbackgroundpictureblack{coursera-video.jpg}
\begin{frame}
\end{frame}

\setbackgroundpicturewhite{mooculus-textbook-page.pdf}
\begin{frame}
\end{frame}

\setbackgroundpicturewhite{coursera-quiz.png}
\begin{frame}
\end{frame}

\setdarkbackgroundpicturewhite{coursera-quiz.png}
\begin{frame}
\huge\textsf{\textbf{Ten questions per quiz.}}

\vspace{1cm}\pause

\huge\textsf{It's a paper quiz, but online.}
\end{frame}

\begin{frame}[nofills]
  \vfill
  \huge\textsf{How many questions \textit{should} \\
\quad be on a quiz?}
 \vfill\pause
  \huge\textsf{\textbf{Depends on the student!}}
\vfill
\end{frame}

%%%%%%%%%%%%%%%%%%%%%%%%%%%%%%%%%%%%%%%%%%%%%%%%%%%%%%%%%%%%%%%%
% Mooculus
\setbackgroundpictureblack{mooculus-landing-page.png}
\begin{frame}
\end{frame}

%%%%%%%%%%%%%%%%%%%%%%%%%%%%%%%%%%%%%%%%%%%%%%%%%%%%%%%%%%%%%%%%
\clearbackgroundpicture
\begin{frame}[nofills]
  \vfill
  \scalebox{3}{\textsf{What is}}
  \includegraphics[width=0.8\textwidth]{cow.pdf}\raisebox{0.25in}{\scalebox{5}{?}}
  \vfill
  \pause
  \huge{Online homework, \pause with \\\quad a hidden Markov model.}
  \vfill
\end{frame}

%%%%%%%%%%%%%%%%%%%%%%%%%%%%%%%%%%%%%%%%%%%%%%%%%%%%%%%%%%%%%%%%
% Mooculus as an exercise platform
\setbackgroundpictureblack{mooculus-exercise-1.png}
\begin{frame}
\end{frame}
\setbackgroundpictureblack{mooculus-exercise-1-highlight.png}
\begin{frame}[nofills]
\pause
\vfill
\vspace{1.5cm}
\huge\color{red!75!black}
We've built a \\
\quad computer algebra \\
\quad\quad system \\
\quad in JavaScript.
\vfill
\end{frame}
\setbackgroundpictureblack{mooculus-exercise-1.png}
\begin{frame}
\end{frame}
\setbackgroundpictureblack{mooculus-exercise-hint.png}
\begin{frame}
\end{frame}
\setbackgroundpictureblack{mooculus-exercise-1.png}
\begin{frame}
\end{frame}
\setbackgroundpictureblack{mooculus-exercise-2.png}
\begin{frame}
\end{frame}
\setbackgroundpictureblack{mooculus-exercise-3.png}
\begin{frame}
\end{frame}
\setbackgroundpictureblack{mooculus-exercise-4.png}
\begin{frame}
\end{frame}
\setbackgroundpictureblack{mooculus-exercise-5.png}
\begin{frame}
\end{frame}
\setbackgroundpictureblack{mooculus-exercise-progress.png}
\begin{frame}
\end{frame}

\setbackgroundpictureblack{mooculus-exercise-progress.png}
\begin{frame}[nofills]
\vfill\vfill
\Huge
Student understanding\\
\quad is invisible.
\vfill\pause
Hints and answers \\
\quad are visible.
\vfill
\vfill
\end{frame}

\setbackgroundpictureblack{mooculus-exercise-index.png}
\begin{frame}
\end{frame}

\clearbackgroundpicture
\begin{frame}[nofills]
\Huge\textbf{What's the benefit \\
\quad of a MOOC?}

\vfill
\uncover<2->{\hfill\alt<1-2>{Cheaper?}{\textcolor{gray}{\sout{Cheaper?}}}%
\hfill\uncover<3->{\textcolor{red!50!black}{Better!}}\hfill\null}
\vfill
 
\end{frame}

\setbackgroundpictureblack{Math_lecture_at_TKK.jpg}
\begin{frame}
\end{frame}

\setdarkbackgroundpictureblack{Math_lecture_at_TKK.jpg}
\begin{frame}[nofill]
\vfill
\Huge\color{white}
Doing mathematics \\
\vfill
\quad is better than \\
\vfill
watching mathematics.
\vfill
\end{frame}

%%% THIS SEGUE IS NOT SO CLEAN

\clearbackgroundpicture
\begin{frame}[nofills]
\Huge\textbf{What's \textit{massive} \\\quad about a MOOC?}

\vfill
\uncover<2->{\hfill\alt<1-2>{Enrollment?}{\textcolor{gray}{\sout{Enrollment?}}}%
\hfill\uncover<3->{\textcolor{red!50!black}{Data!}}\hfill\null}
\vfill
 
\end{frame}



\begin{frame}[nofills]
INCLUDE DATA
\end{frame}


\begin{frame}[nofills]
\Huge
\vfill
  Evolution requires \\
  \quad heritable variation.
\vfill
  Unlike an offline course, \\
  \quad MOOCs offer this.
\vfill
\end{frame}


\setbackgroundpicturewhite{brownian-motion/brownian-motion-1.pdf}
\begin{frame}[nofills]
\null\vspace{1ex}
\Huge An analogy
\end{frame}
\setbackgroundpicturewhite{brownian-motion/brownian-motion-2.pdf}
\begin{frame}[nofills]
\end{frame}
\setbackgroundpicturewhite{brownian-motion/brownian-motion-3.pdf}
\begin{frame}[nofills]
\end{frame}
\setbackgroundpicturewhite{brownian-motion/brownian-motion-more-1.pdf}
\begin{frame}[nofills]
\end{frame}
\setbackgroundpicturewhite{brownian-motion/brownian-motion-more-2.pdf}
\begin{frame}[nofills]
\end{frame}
\setbackgroundpicturewhite{brownian-motion/brownian-motion-more-3.pdf}
\begin{frame}[nofills]
\end{frame}
\setbackgroundpicturewhite{brownian-motion/brownian-motion-more-4.pdf}
\begin{frame}[nofills]
\end{frame}
\setbackgroundpicturewhite{brownian-motion/brownian-motion-more-5.pdf}
\begin{frame}[nofills]
\end{frame}
\setbackgroundpicturewhite{brownian-motion/brownian-motion-more-5.pdf}
\begin{frame}[nofills]
\Huge Teaching without data \\
\quad is a random walk.
\end{frame}

\setbackgroundpicturewhite{brownian-motion/brownian-motion-weighted-1.pdf}
\begin{frame}[nofills]
\null\vspace{1ex}
\Huge But with feedback?
\end{frame}
\setbackgroundpicturewhite{brownian-motion/brownian-motion-weighted-2.pdf}
\begin{frame}[nofills]
\end{frame}
\setbackgroundpicturewhite{brownian-motion/brownian-motion-weighted-3.pdf}
\begin{frame}[nofills]
\end{frame}
\setbackgroundpicturewhite{brownian-motion/brownian-motion-weighted-more-1.pdf}
\begin{frame}[nofills]
\end{frame}
\setbackgroundpicturewhite{brownian-motion/brownian-motion-weighted-more-2.pdf}
\begin{frame}[nofills]
\end{frame}
\setbackgroundpicturewhite{brownian-motion/brownian-motion-weighted-more-3.pdf}
\begin{frame}[nofills]
\end{frame}
\setbackgroundpicturewhite{brownian-motion/brownian-motion-weighted-more-4.pdf}
\begin{frame}[nofills]
\end{frame}
\setbackgroundpicturewhite{brownian-motion/brownian-motion-weighted-more-5.pdf}
\begin{frame}[nofills]
\end{frame}
\setbackgroundpicturewhite{brownian-motion/brownian-motion-weighted-more-5.pdf}
\begin{frame}[nofills]
\Huge MOOCs are really about data. \\
\end{frame}

\setbackgroundpicturewhite{brownian-motion/brownian-motion-together.pdf}
\begin{frame}[nofills]
\Huge \uncover<2->{\color{blue}{MOOCs will improve.}}
\vfill
\uncover<3->{\color{red}{Will the University?}}
\end{frame}

\begin{frame}[nofills]
Data transformed medicine from anecdote to science.
\end{frame}



% \begin{frame}<1-2>[nofills,label=overview]
%   \fontsize{28}{28}
%   \fontspec{Helvetica Neue Bold}Three projects

%   \vfill

%   \fontsize{20}{20}
%   \fontspec{Helvetica Neue}

%   \alert<2>{\quad Using mobile phones as clickers}

%   \vfill

%   \alert<3>{\quad An online calculus course}

%   \vfill

%   \alert<4>{\quad Optical mark recognition for exams}

%   \vfill

% \end{frame}

% %%%%%%%%%%%%%%%%%%%%%%%%%%%%%%%%%%%%%%%%%%%%%%%%%%%%%%%%%%%%%%%%
% \setbackgroundpicture{clicker.png}
% \begin{frame}
% \end{frame}

% \setdarkbackgroundpicture{clicker.png}
% \begin{frame}
%   \fontsize{32}{32}
%   \fontspec{Helvetica Neue Bold}
%   \color{white}

% No expensive clickers.

% \vfill

% Students use their \\
% mobile phone.

% \end{frame}

% \setbackgroundpicture{poll-everywhere.png}
% \begin{frame}
% \end{frame}

% \setbackgroundpicture{top-hat-monocle.png}
% \begin{frame}
% \end{frame}

% \setdarkbackgroundpicture{top-hat-monocle.png}
% \begin{frame}
%   \fontsize{32}{32}
%   \fontspec{Helvetica Neue Bold}
%   \color{white}

% Online in March 2010

% \end{frame}

% \setbackgroundpicture{top-hat-eight-million.png}
% \begin{frame}
% \end{frame}

% \setdarkbackgroundpicture{top-hat-eight-million.png}
% \begin{frame}
%   \fontsize{32}{32}
%   \fontspec{Helvetica Neue Bold}
%   \color{white}

%    100,000 students

%    \vfill

%    150 universities

%    \vfill

%    \$9.56 million in VC

%    \vfill
%    \pause

%    Could've been OSU

% \end{frame}



% %%%%%%%%%%%%%%%%%%%%%%%%%%%%%%%%%%%%%%%%%%%%%%%%%%%%%%%%%%%%%%%%
% \clearbackgroundpicture
% \againframe<3>{overview}

% %%%%%%%%%%%%%%%%%%%%%%%%%%%%%%%%%%%%%%%%%%%%%%%%%%%%%%%%%%%%%%%%
% \clearbackgroundpicture
% \begin{frame}[nofills]

% \vfill

% \makebox[\textwidth][c]{\includegraphics[width=\paperwidth]{mooculus-logo.pdf}}

% \vfill

% \fontsize{16}{16}
% \fontspec{Helvetica Neue}

% Bart Snapp \hfill Jim Fowler \hfill Roman Holowinsky \\
% \vspace{1ex}
% \hfill Ryan Kowalick \hfill Steve Gubkin \hfill\null \\

% %\vfill

% %Replace community theater with television

% %\vfill

% %Assessment more important than videos

% \vfill

% \end{frame}

% \setbackgroundpicture{coursera.png}
% \begin{frame}
% \end{frame}

% \clearbackgroundpicture
% \begin{frame}
% \makebox[\textwidth][c]{\includegraphics[height=\paperheight]{enrollments.pdf}}
% \end{frame}

% %%%%%%%%%%%%%%%%%%%%%%%%%%%%%%%%%%%%%%%%%%%%%%%%%%%%%%%%%%%%%%%%
% \clearbackgroundpicture
% \begin{frame}[nofills]

% \vfill

% \fontsize{24}{24}
% \fontspec{Helvetica Neue Bold}
% Videos \\
% \fontsize{24}{24}
% \fontspec{Helvetica Neue}
% From community theater to TV

% \vfill

% \fontsize{24}{24}
% \fontspec{Helvetica Neue Bold}
% Assessment \\
% \fontsize{24}{24}
% \fontspec{Helvetica Neue}
% Iterative development

% \vfill

% \end{frame}

% %%%%%%%%%%%%%%%%%%%%%%%%%%%%%%%%%%%%%%%%%%%%%%%%%%%%%%%%%%%%%%%%
% \clearbackgroundpicture
% \againframe<4>{overview}

% \clearbackgroundpicture
% \begin{frame}[nofills]

% \fontsize{24}{24}
% \fontspec{Helvetica Neue Bold}

% Process

% \vfill

% \fontsize{24}{24}
% \fontspec{Helvetica Neue}

% \quad Student writes on paper.

% \vfill

% \quad The paper is scanned.

% \vfill

% \quad Grader types comments.

% \vfill

% \quad Graded work is emailed back.

% \vfill

% \end{frame}

% \clearbackgroundpicture
% \begin{frame}
% \makebox[\textwidth][c]{\includegraphics[width=\paperwidth]{example-worksheet.pdf}}
% \end{frame}

% \clearbackgroundpicture
% \begin{frame}
% \makebox[\textwidth][c]{\includegraphics[width=\paperwidth]{example-worksheet-complete.pdf}}
% \end{frame}


% %%%%%%%%%%%%%%%%%%%%%%%%%%%%%%%%%%%%%%%%%%%%%%%%%%%%%%%%%%%%%%%%
% \clearbackgroundpicture
% \begin{frame}[nofills]
%   \vfill
%   \begin{center}
%   \fontsize{60}{60}
%   \fontspec{Helvetica Neue Bold} Thank You
%   \end{center}

%   \vfill
%   \vfill
%   \includegraphics[width=1in]{cc-logo.pdf}\\
%   \fontsize{8}{8}\fontspec{Helvetica Neue Light} licensed for reuse under a Creative Commons BY-NC-ND License
%   \null
% \end{frame}

\end{document}
